% !TEX root = sga4.5.tex

\chapter{Cat\'egories dérivées}\label{VIII}










\section{Cat\'egories triangul\'ees: d\'efinitions et exemples}\label{VIII:1}





\subsection{D\'efinition des cat\'egories triangul\'ees}\label{VIII:1-1}


\addtocounter{subsubsection}{-1}
\subsubsection{}\label{VIII:1-1-0}

Soient $\cA$ un cat\'egorie additive, $T$ un automorphisme additif de $\cA$. 
$X$ et $Y$ \'etant deux objets de $\cA$, nous poserons 
\[
  \hom^i(X,Y) = \hom_\cA(X,T^i(Y)) \qquad i\in \dZ 
\]
Soient $X$, $Y$, $Z$ trois objets de $\cA$, $\alpha\in \hom^i(X,Y)$, 
$\beta\in \hom^j(X,Y)$. Nous d\'efinirons le compos\'e $\beta\circ\alpha$ 
appartenant \`a $\hom^{i+j}(X,Z)$ par: 
\[
  \beta\circ \alpha = \beta\circ T^i(\alpha) \text{.}
\]
On v\'erifie qu'on obtient ainsi une nouvelle cat\'egorie dont les groupes de 
morphismes entre les objets sont gradu\'es. Cette nouvelle cat\'egorie sera 
appel\'ee, par abus de langage: la cat\'egorie $\cA$, gradu\'ee par le foncteur 
de translation $T$. 

Soit $\cA$ une cat\'egorie gradu\'ee par un foncteur de translation $T$. 
Lorsqu'on nommera ou notera un morphisme sans sp\'ecifier le degr\'e, il 
s'agira toujours d'un morphisme de degr\'e z\'ero. 

Soient $\cA$ et $\cA'$ deux cat\'egories additives gradu\'ees par les foncteurs
$T$ et $T'$, $F$ un foncteur additif de $\cA$ dans $\cA'$. On dira que $F$ 
\emph{est gradu\'e} si l'on s'est donn\'e un isomorphisme de 
foncteurs\footnote{Texte modifi\'e en 1976: le texte original demandait une 
\'egalit\'e $F\circ T=T'\circ F$. Le cas des foncteurs \`a plusieurs variables 
est discut\'e dans \cite[XVIII 0.2]{sga4}. Il pose un probl\`eme de signes.} 
\[
  F\circ T = T'\circ F 
\]

La d\'efinition des morphismes de foncteurs gradu\'es est donn\'ee au 
\ref{VIII:4-1-1}. 

$\cA$ \'etant une categorie additive gradu\'ee par le foncteur $T$, on 
appellers triangle, un ensemble de trois objets $X$, $Y$, $Z$ et de trois 
morphismes $u:X\to Y$, $v:Y\to Z$, $w:Z\to T(X)$ ($\deg w=1$). Pour d\'esigner 
les triangles on utilisera la notation: $(X,Y,Z,u,v,w)$ on bien le diagramme: 
\[\xymatrix{
  & Z \ar[dl]_-w \\
  X \ar[rr]^-u 
    & & Y \ar[lu]_-v 
}\qquad \deg(w)=1
\]
Soient $(X,Y,Z,u,v,w)$, $(X',Y',Z',u',v',w')$ deux triangles de $\cA$; un 
morphisme du premier triangle dans le second est un ensemble de trois 
morphismes $f:X\to X'$, $g:Y\to Y'$, $h:Z\to Z'$, tels que le diagramme 
suivant soit commutatif: 
\[\xymatrix{
  X \ar[r]^-u \ar[d]^-f 
    & Y \ar[r]^-v \ar[d]^-g 
    & Z \ar[r]^-w \ar[d]^-h 
    & T(X) \ar[d]^-{T(f)} \\
  X' \ar[r]^-{u'} 
    & Y' \ar[r]^-{v'} 
    & Z' \ar[r]^-{w'} 
    & T(X') 
}\]





\subsubsection{}\label{VIII:1-1-1}

On appelle \emph{cat\'egorie triangul\'ee} une cat\'egorie additive gradu\'ee 
par un foncteur de translation, munie d'une famille de triangles qu'on appelle 
famille des triangles distingu\'es. Cette famille doit v\'erifier de plus les 
axiomes: 


\paragraph{TR1}
\phantomsection\hypertarget{VIII:TR1}{}

Tout triangle isomorphe \`a triangle distingu\'e est distingu\'e. Tout 
morphisme $u:X\to Y$ est contenu dans un triangle distingu\'e 
$(X,Y,Z,u,v,w)$. Le triangle $(X,X,0,\operatorname{id}_{X'},0,0)$ est 
distingu\'e. 


\paragraph{TR2}
\phantomsection\hypertarget{VIII:TR2}{}

Pour que $(X,Y,Z,u,v,w)$ soit distingu\'e, il faut et il suffit que 
$(Y,Z,T(X),v,w,-T(u))$ soit distingu\'e. 


\paragraph{TR3}
\phantomsection\hypertarget{VIII:TR3}{}

$(X,Y,Z,u,v,w)$, $(X',Y',Z',u',v',w')$ \'etant distingu\'es, pour tout 
morphisme $(f,g):u\to u'$, il existe un morphisme $h:Z\to Z'$ tel que $(f,g,h)$ 
soit un morphisme de triangles. 


\paragraph{TR4}
\phantomsection\hypertarget{VIII:TR4}{}

\[\xymatrix{
  & Z' \ar[dl]|{\deg=1} \\
  X \ar[rr]^-u 
    & & Y \ar[ul]_-i 
}\quad
\xymatrix{
  & X' \ar[dl]|{\deg(j)=1} \\
  Y \ar[rr]^-v 
    & & Z \ar[ul] 
}\quad
\xymatrix{
  & Y' \ar[dl]|{\deg=1} \\
  X \ar[rr]^-w 
    & & Z \ar[ul] 
}\]
\'etant trois triangles distingu\'es tels que $w=v\circ u$, il existe deux 
morphismes $f:Z'\to Y'$, $g:Y' \to X'$ tels que 
\begin{enumerate}
  \item $(\operatorname{id}_X, v, f)$ soit un morphisme de triangle 
  \item $(u,\operatorname{id}_Z,g)$ soit un morphisme de triangle 
  \item $(Z',Y',X',f,g,T(i) j)$ soit un triangle distingu\'e
\end{enumerate}

$\cC$ et $\cC'$ \'etant deux cat\'egories triangul\'ees, on appelle 
\emph{foncteur exact} de $\cC$ dans $\cC'$, tout foncteur additif, gradu\'e, 
transformant les triangles distingu\'es en triangles distingu\'es. Les 
morphismes de foncteurs exacts sont les morphismes de foncteurs gradu\'es. 

On d\'eduit imm\'ediatement dex axiomes \hyperlink{VIII:TR1}{TR1}, 
\hyperlink{VIII:TR2}{TR2}, \hyperlink{VIII:TR3}{TR3} la propri\'et\'e suivante: 





\begin{proposition}\label{VIII:1-1-2}
Soient $(X,Y,Z,u,v,w)$ un triangle distingu\'e et $M$ un objet d'une 
cat\'egorie triangul\'ee. On a alors les suites exactes: 
\[\xymatrix{
  \cdots \ar[r] 
    & \hom^i(M,X) \ar[r]^-{\hom(M,u)} 
    & \hom^i(M,Y) \ar[r]^-{\hom(M,v)} 
    & \hom^i(M,Z) \ar[r]^-{\hom(M,w)} 
    & \hom^{i+1}(M,X) \ar[r] 
    & \cdots
}\]
\[\xymatrix{
  \cdots \ar[r] 
    & \hom^i(Z,M) \ar[r]^-{\hom(v,M)} 
    & \hom^i(Y,M) \ar[r]^-{\hom(u,M)} 
    & \hom^i(X,M) \ar[r]^-{\hom(w,M)} 
    & \hom^{i+1}(Z,M) \ar[r] 
    & \cdots
}\]
\end{proposition}

On d\'eduit alors de cette proposition des \'enonc\'es du genre: 
\begin{itemize}
  \item $(f,g,h)$ \'etant un morphisme de triangle distingu\'e et $f,g$ des 
    isomorphismes. 
  \item Les triangles distingu\'es construits sur un morphisme 
    (\hyperlink{VIII:TR1}{TR1}), sont tour isomorphismes (\emph{mais les 
    isomorphismes ne sont pas, en g\'en\'eral, uniquoment d\'etermin\'es}). 
  \item Lorsque dans un triangle distingu\'e, un des morphismes admet un noyau 
    ou un conoyau, celui-ci se scinde i.e. il est de la forme: 
    \[\xymatrix{
      & Z+T(X) \ar[dl] \\
      X+Y \ar[rr] 
        & & Y+Z \ar[ul]
    }\]
  \item La somme de deux triangles distingu\'es est distingu\'ee. (On utilise 
    l'axiome \hyperlink{VIII:TR4}{TR4}). 
\end{itemize}










\subsection{Exemples}\label{VIII:1-2}





\subsubsection{}\label{VIII:1-2-1}

Nous allons d'abord fixer quelques notations. Soit $\cA$ une cat\'egorie 
additive $\eC(\cA)$ d\'esignera la cat\'egorie suivante: 
\begin{itemize}
  \item Les objets de $\eC(\cA)$ seront les complexes de $\cA$, sans limitation 
    de degr\'e, \`a diff\'erentielle de degr\'e $+1$. 
  \item Les morphismes de $\eC(\cA)$ seront les morphismes de complexes (qui 
    commutent avec la diff\'erentielle) conservant le degr\'e. 
\end{itemize}

Soit $T$ le foncteur suivant: pour tout objet $X^\bullet$ de $\eC(\cA)$ 
\begin{align*}
  T(X^\bullet)_i &= (X^\bullet)_{i+1} \\
  d_i(T(X^\bullet)) &= -d_{i+1}(X^\bullet)  \text{.}
\end{align*}
Sur les morphismes le foncteur $T$ agit de la mani\`ere suivante: pour tout 
morphisme $f:X^\bullet \to Y^\bullet$, le morphisme 
$T(f):T(X^\bullet) \to T(Y^\bullet)$ est d\'efini par 
\[
  T(f)_i = f_{i+1} \text{.}
\]
On v\'erifie imm\'ediatement que $T$ est un automorphisme additif de 
$\eC(\cA)$. Les morphismes de degr\'e $n$, d\'efinis \`a l'aide de ce foncteur 
de translation, sont alors les morphismes qui augmentent le degr\'e de $n$, et 
qui commutent ou anticommutent avec la diff\'erentielle suivant la parit\'e de 
$n$. On retrouve ainsi la d\'efinition g\'en\'eralement adopt\'ee. 

$\sC$ d\'esignera la famille de triangles suivante: un triangle 
$(X^\bullet,Y^\bullet,Z^\bullet,u,v,w)$ est un \'el\'ement de la famille 
si\footnote{Texte modifi\'e en 1976, pour assurer la validit\'e de 
\ref{VIII:1-2-4}.}
\begin{itemize}
  \item $Z^\bullet$ est le complexe simple associ\'e au complexe double 
    \[\xymatrix{
      \cdots \ar[r] 
        & 0 \ar[r] 
        & X^\bullet \ar[r]^-u 
        & Y^\bullet \ar[r] 
        & 0 \ar[r] 
        & \cdots 
    }\]
    o'\`u les objets de $Y^\bullet$ sont les objets de premier degr\'e z\'ero, 
    et les objets de $X^\bullet$, les objets de premier degr\'e $-1$. 
  \item $v$ est l'image par le foncteur: complexe double $\mapsto$ complexe 
    simple, du morphisme de doubles complexes: 
    \[\xymatrix{
      \cdots \ar[r] 
        & 0 \ar[r] \ar[d] 
        & 0^\bullet \ar[r] \ar[d] 
        & Y^\bullet \ar[r] \ar[d] 
        & 0 \ar[r] \ar[d] 
        & \cdots \\
      \cdots \ar[r] 
        & 0 \ar[r] 
        & X^\bullet \ar[r] 
        & Y^\bullet \ar[r] 
        & 0 \ar[r] 
        & \cdots 
    }\]
  \item $w$ est l'oppos\'e de l'image, par le m\^eme foncteur, du morphisme de 
    doubles complexes: 
    \[\xymatrix{
      \cdots \ar[r] 
        & 0 \ar[r] 
        & X^\bullet \ar[r] 
        & 0 \ar[r] 
        & 0 \ar[r] 
        & \cdots \\
      \cdots  \ar[r] 
        & 0 \ar[r] \ar[u] 
        & X^\bullet \ar[r] \ar[u] 
        & Y^\bullet \ar[r] \ar[u] 
        & 0 \ar[r] \ar[u] 
        & \cdots 
    }\]
\end{itemize}





\subsubsection{}\label{VIII:1-2-2}

L'ensemble des morphismes homotopes \`a z\'ero est un id\'eal bilat\`ere (si 
$f$ et $g$ sont composables et si $f$ ou $g$ homotope \`a z\'ero, la compos\'e 
est homotope \`a z\'ero. Si $f$ et $g$ sont homotopes \`a z\'ero, le morphisme 
$f\oplus g$ est homotope \`a z\'ero). Pour tout couple d'objet $X^\bullet$, 
$Y^\bullet$ on d\'esigne par $\operatorname{Htp}(X^\bullet,Y^\bullet)$ le 
sous-groupe de $\hom_{\eC(\cA)}(X^\bullet,Y^\bullet)$ des morphismes homotopes 
\`a z\'ero. $\eK(\cA)$ sera alors la cat\'egorie dont les objets sont les 
objets de $\eC(\cA)$ et les morphismes de source $X^\bullet$ et de but 
$Y^\bullet$, le groupe: 
\[
  \hom_{\eC(\cA)}(X^\bullet,Y^\bullet) / \operatorname{Htp}(X^\bullet,Y^\bullet) \text{.}
\]
La loi de composition sur les morphismes de $\eK(\cA)$ se d\'efinit \`a partir 
de la loi de composition sur les morphismes de $\eC(\cA)$ en passant au 
quotient. $\eK(\cA)$ est une cat\'egorie additive. Le foncteur de translation 
$T$ sur $\eC(\cA)$ passe au quotient (le translat\'e d'un morphisme homotope 
\`a z\'ero, est homotope \`a z\'ero). Le foncteur obtenu est un automorphisme 
additif de $\eK(\cA)$ que nous noterons encore $T$. Par abus de notation 
$\eK(\cA)$ d\'esignera par la suite la cat\'egorie $\eK(\cA)$ gradu\'ee par le 
foncteur $T$. 

On appellera triangle distingu\'e dans $\eK(\cA)$, tout triangle 
\emph{isomorphe} \`a un triangle provenant de $\sC$. La famille des triangles 
distingu\'es v\'erifie les axioms \hyperlink{VIII:TR1}{TR1}, 
\hyperlink{VIII:TR2}{TR2}, \hyperlink{VIII:TR3}{TR3} et 
\hyperlink{VIII:TR4}{TR4}. Par abus de notation $\eK(\cA)$ d\'esignera la 
cat\'egorie $\eK(\cA)$ triangul\'ee par la famille de triangles d\'efinie 
ci-dessus. 





\subsubsection{}\label{VIII:1-2-3}

On d\'efinit de plus tout un arsenal de sous-cat\'egories: 
\begin{itemize}
  \item Un complexe est dit born\'e inf\'erieurement si tous ses objets de 
    degr\'e n\'egatif, sauf au plus un nombre fini, sont nuls. On d\'efinit de 
    m\^eme les complexes born\'es sup\'erieurement, les complexes born\'es. On 
    d\'esignera par $\eK^+(\cA)$, $\eK^-(\cA)$, $\eK^b(\cA)$ les 
    sous-cat\'egories pleines de $\eK(\cA)$ engendr\'ees par ses familles 
    d'objets. Ces sous-cat\'egories sont des ``\emph{sous-cat\'egories 
    triangul\'ees}'' de $\eK(\cA)$: tout triangle distingu\'e dont deux des 
    objets sont des objets de la sous-cat\'egorie, est isomorphe \`a un 
    triangle dont les trois objets sont des objets de la sous-cat\'egorie. 
  \item Supposons que $\cA$ soit une cat\'egorie ab\'elienne. Un complexe est 
    dit cohomologiquement born\'e inf\'erieurement si tous ses objets de 
    cohomologie de degr\'e n\'egatif sont nuls sauf un nombre fini d'entre 
    eux. On d\'efinit de mani\`ere analogue les complexes cohomologiquement 
    born\'es sup\'erieurement, les complexes cohomologiquement born\'es, les 
    complexes acycliques. $\eK^{\infty,+}(\cA)$, $\eK^{\infty,-}(\cA)$, 
    $\eK^{\infty,b}(\cA)$, $\eK^{\infty,\varnothing}(\cA)$ d\'esigneront les 
    sous-cat\'egories pleines engendr\'ees par ces familles d'objets. Ce sont 
    des sous-cat\'egories triangul\'ees. 
    
    On paiera aussi des sous-cat\'egories triangul\'ees $\eK^{+,b}(\cA)$, 
    $\eK^{+,\varnothing}(\cA)$, \ldots etc \ldots (le premier signe en 
    exposant donne des renseignements sur les objets des complexes, le 
    deuxi\`eme signe sur les objets de cohomologie). 
  \item $\cA$ n'\'etant plus n\'ecessairement ab\'elienne, soit $O$ un 
    ensemble d'objets de $\cA$ stable par isomorphisme et par somme directe. 
    $O(\cA)$ d\'esignera la sous-cat\'egorie pleine de $\eC(\cA)$ engendr\'ee 
    par les complexes dont les objets en tout degr\'e sont des objets de $O$. 
    On notera $\underline O(\cA)$ la sous-cat\'egorie triangul\'ee 
    correspondante dans $\eK(\cA)$. $\cA$ \'etant ab\'elienne, on pourra 
    prendre pour ensemble $O$, l'ensemble $I$ des injectifs ou l'ensemble $P$ 
    des projectifs. 
\end{itemize}





\subsubsection{}\label{VIII:1-2-4}

$\eK(\cA)$ est uniquement d\'etermin\'ee par la donn\'ee de $\eC(\cA)$ et de la 
famille $\sC$. De mani\`ere pr\'ecise: tout foncteur additif gradu\'e de 
$\eC(\cA)$ dans une cat\'egorie triangul\'ee, transformant tout triangle de 
$\sC$ en un triangle distingu\'e, se factorise de mani\`ere unique par 
$\eK(\cA)$, et le foncteur obtenu est exact. 

Pour tout complexe $X^\bullet$ de $\eC(\cA)$, $\bar X^\bullet$ d\'esignera le 
complexe obtenu \`a partir de $X^\bullet$ en annulant la diff\'erentielle. Une 
suite \`a trois termes: 
\[\xymatrix{
  0 \ar[r] 
    & X^\bullet \ar[r] 
    & Y^\bullet \ar[r] 
    & Z^\bullet \ar[r] 
    & 0 
}\]
sera dite suite semi-scind\'ee si la suite 
\[\xymatrix{
  0 \ar[r] 
    & \bar X^\bullet \ar[r] 
    & \bar Y^\bullet \ar[r] 
    & \bar Z^\bullet \ar[r] 
    & 0 
}\]
est scind\'ee; i.e. si elle est isomorphe \`a la suite: 
\[\xymatrix{
  0 \ar[r] 
    & \bar X^\bullet \ar[r] 
    & \bar X^\bullet + \bar Z^\bullet \ar[r] 
    & \bar Z^\bullet \ar[r] 
    & 0 \text{.}
}\]

Soit $0 \to X^\bullet \to Y^\bullet \to Z^\bullet \to 0$ une suite 
semi-scind\'ee. Choisissons un scindage i.e. pour tout $n$ un isomorphisme 
\[
  (Y^\bullet)^n \simeq (X^\bullet)^n + (Z^\bullet)^n \text{.}
\]
La matrice de la diff\'erentielle de $Y^\bullet$, dans la d\'ecomposition en 
somme directe si-dessus est de la forme: 
\[
  \begin{pmatrix}
    d_{X^\bullet} & \varphi_n \\
    0 & d_{Y^\bullet} 
  \end{pmatrix}
\]
o\`u $\varphi_n$ est un morphisme de $(Z^\bullet)^n$ dans 
$(X^\bullet)^{n+1}$. Les $\varphi_n$ d\'eterminent un morphisme de complexes 
\[
  \varphi:Z^\bullet \to T(X^\bullet) \text{.}
\]
Lorsqu'on change le scindage, la classe modulo homotopie de $\varphi$ ne 
change pas. De plus, en notant $\underline u$, $\underline v$, 
$\underline\varphi$ les images dans $\eK(\cA)$ des morphismes $u$, $v$, 
$\varphi$, le triangle 
$(X^\bullet,Y^\bullet,Z^\bullet,\underline u,\underline v,\underline\varphi)$ 
est distingu\'e. D\'esignons par $\mathsf{SSS}(\cA)$ la cat\'egorie des 
suites semi-scind\'ees de $\eC(\cA)$, et par $\mathsf{TrK}(\cA)$ la 
cat\'egorie des triangles distingu\'es de $\eK(\cA)$. Ce qui pr\'ec\`ede 
permet de d\'efinir un foncteur: 
\[
  \rho:\mathsf{SSS}(\cA) \to \mathsf{TrK}(\cA) \text{.}
\]
Ce foncteur est essentiellement surjectif. 










\subsection{Exemples de foncteurs cohomologiques et de foncteurs exacts}\label{VIII:1-3}





\begin{definition}
Soient $\cA$ une cat\'egorie triangul\'ee, $\cB$ une cat\'egorie ab\'elienne. 
Un foncteur additif $F:\cA\to \cB$ est die \emph{foncteur cohomologique} si 
pour tout triangle distingu\'e $(X,Y,Z,u,v,w)$ la suite 
\[\xymatrix{
  F(X) \ar[r]^-{F(u)} 
    & F(Y) \ar[r]^-{F(v)} 
    & F(Z) 
}\]
est exacte. 
\end{definition}

Le foncteur $F\circ T^i$ sera souvent not\'e $F^i$. En vertu de l'axiome 
\hyperlink{VIII:TR2}{TR2} des cat\'egories triangul\'ees, on a la suite 
exacte illimit\'ee: 
\[\xymatrix{
  \cdots \ar[r] 
    & F^i(X) \ar[r] 
    & F^i(Y) \ar[r] 
    & F^i(Z) \ar[r] 
    & F^{i+-}(X) \ar[r] 
    & \cdots 
}\]





\subsubsection{Exemples}\label{VIII:1-3-2}

\begin{enumerate}
  \item Soit $\cA$ une cat\'egorie triangul\'ee. Pour tout objet $X$ de $\cA$ 
    le foncteur $\hom_\cA(X,-)$ est un foncteur cohomologique. Le foncteur 
    $\hom_\cA(-,X)$ est un foncteur cohomologique $\cA^\circ\to\mathsf{Ab}$. 
  \item Soient $\cA$ une cat\'egorie ab\'elienne, $\eC(\cA)$ la cat\'egorie des 
    complexes de $\cA$. On d\'esigne par $\h^0:\eC(\cA)\to\cA$ le foncteur 
    suivant: Soit $X^\bullet$ un objet de $\eC(\cA)$. On pose 
    \[
      \h^0(X^\bullet) = \ker d^0 / \im{d^{-1}} \text{.}
    \]
    $\h^0$ annule les morphismes homotopes \`a z\'ero, donc se factorise de 
    mani\`ere unique par $\eK(\cA)$. On d\'esignera encore par 
    $\h^0:\eK(\cA) \to \cA$ le foncteur obtenu. 
\end{enumerate}





\subsubsection{Exemple de bi-foncteur exact}\label{VIII:1-3-3}

Soient $\cA$, $\cA'$, $\cA''$ trois cat\'egories additives, 
\[
  P:\cA\times \cA'\to \cA''
\]
un foncteur bilin\'eaire (i.e. additif par rapport \`a chacun des arguments 
s\'epar\'ement). On en d\'eduit alors le foncteur bilin\'eaire: 
\[
  P^\ast:\eC(\cA)\times \eC(\cA') \to \eC(\cA'') 
\]
de la mani\`ere suivante: 

Soient $X^\bullet$ un objet de $\eC(\cA)$ et $Y^\bullet$ un objet de 
$\eC(\cA')$. $P(X^\bullet,Y^\bullet)$ est un complexe double de $\cA''$. On 
pose alors: $P^\ast(X^\bullet,Y^\bullet) = $ complexe simple associ\'e \`a 
$P(X^\bullet,Y^\bullet)$. 

Soient $f$ un morphisme de $\eC(\cA)$ (resp. $\eC(\cA')$) homotope \`a z\'ero 
et $Z^\bullet$ un objet de $\eC(\cA')$ (resp. $\eC(\cA)$). Le morphisme 
$P^\ast(f,Z^\bullet)$ (resp. $P^\bullet(Z^\bullet,f)$) est alors homotope \`a 
z\'ero. On en d\'eduit que $P^\ast$ d\'efinit d'une mani\`ere unique un 
foncteur: 
\[
  P^\bullet:\eK(\cA) \times \eK(\cA') \to \eK(\cA'') \text{.}
\]
$P^\bullet$ est un bi-foncteur exact. 

En particulier, soit $\cA$ un cat\'egorie additive. On peut prendre pour $P$ le 
foncteur: 
\begin{align*}
  \cA^\circ\times \cA &\to \mathsf{Ab} \\
  (X,Y) &\mapsto \hom(X,Y) 
\end{align*}

On obtient alors par la construction pr\'ec\'edente un foncteur 
\[
  \hom^\bullet:\eK(\cA)^\circ \times \eK(\cA) \to \eK(\mathsf{Ab}) 
\]
qui, compos\'e avec le foncteur $\h^0:\eK(\mathsf{Ab})\to \mathsf{Ab}$, 
redonne \'evidemment le foncteur $\hom_{\eK(\cA)}$. 




















\section{Les cat\'egories quotients}\label{VIII:2}










\subsection{Sous-cat\'egories \'epaisses. Syst\`emes multiplicatifs de morphismes}\label{VIII:2-1}

Nous commencerons par donner deux d\'efinitions. 





\begin{definition}\label{VIII:2-1-1}
Une sous-cat\'egorie $\cB$ d'une cat\'egorie triangul\'ee $\cA$ est dite 
\emph{\'epaisse} si $\cB$ est une sous-cat\'egorie triangul\'ee pleine de $\cA$ 
et si de plus $\cB$ poss\`ede la propri\'et\'e suivante: 

Pour tout morphisme $f:X\to Y$, se factorisant par un objet de $\cB$ et contenu 
dans un triangle distingu\'e $(X,Y,Z,f,g,h)$, o\`u $Z$ est un objet de $\cB$, 
la source de $f$ et le but de $f$ sont des objets de $\cB$. 
\end{definition}

L'ensemble des sous-cat\'egories \'epaisses de $\cA$ sera not\'e $\sN$. 

L'intersection d'une famille quelconque de sous-cat\'egories \'epaisses est 
une sous-cat\'egorie \'epaisse. 





\begin{definition}
Soit $\cA$ un cat\'egorie triangul\'ee. Un ensemble $S$ de morphismes de $\cA$ 
est appel\'e \emph{syst\`eme multiplicatif de morphismes} si'il pos\`ede les 
cinq propri\'et\'es suivantes: 

\paragraph{FR1}
\phantomsection\hypertarget{VIII:FR1}{}
Si $f,g\in S$ et si $f$ et $g$ sont composables, $f\circ g\in S$. Pout tout 
objet $X$ de $\cA$, le morphisme identique de $X$ est un \'el\'ement de $S$. 

\paragraph{FR2}
\phantomsection\hypertarget{VIII:FR2}{}
Dans la cat\'egorie $\cA$, tout diagramme 
\[\xymatrix{
  & Y \ar[d]^-{s\in S} \\
  Z \ar[r]^-f 
    & X 
}\]
se compl\`ete en un diagramme commutatif 
\[\xymatrix{
  P \ar[r]^-g \ar[d]^-{t\in S} 
    & Y \ar[d]^-{s\in S} \\
  Z \ar[r]^-f 
    & X 
}\]
De plus la propri\'et\'e sym\'etrique est vraie. 

\paragraph{FR3}
\phantomsection\hypertarget{VIII:FR3}{}
Si $f$ et $g$ sont des morphismes, les propri\'et\'es suivantes sont 
\'equivalentes: 
\begin{enumerate}[\indent i)]
  \item il existe un $s\in S$ tel que $s\circ f=s\circ g$ 
  \item il existe un $t\in S$ tel que $f\circ t=g\circ t$
\end{enumerate}

\paragraph{FR4}
\phantomsection\hypertarget{VIII:FR4}{}
Soit $T$ le foncteur de translation de $\cA$. Par tout \'el\'ement $s$ de $S$, 
$T(s)\in S$. 

\paragraph{FR5}
\phantomsection\hypertarget{VIII:FR5}{}
$(X,Y,Z,u,v,w)$, $(X',Y',Z',u',v',w')$ \'etant deux triangles distingu\'ees, 
$f$ et $g$ deux \'el\'ements de $S$ tels que le couple $(f,g)$ soit un 
morphisme de $u$ dans $u'$, il existe un morphisme $h\in S$, tel que 
$(f,g,h)$ soit un morphisme de triangle. 
\end{definition}

Un syst\`eme multiplicatif de morphismes est dit \emph{satur\'e} s'il poss\`ede 
la propri\'et\'e suivante: 

Un morphisme $f$ appartient \`a $S$ si et seulement s'il existe deux 
morphismes $g$ et $g'$ tels que $g\circ f\in S$ et $f\circ g'\in S$. 

L'intersection d'une famille quelconque de syst\`emes multiplicatifs satur\'es 
un syst\`eme multiplicatif satur\'e. 

L'ensemble des syst\`emes multiplicatifs satur\'es sera not\'e $\sS$. 

Soit $\cB$ une sous-cat\'egorie \'epaisse; on d\'esigne par $\varphi(\cB)$ 
l'ensemble des morphismes $f$ qui sont contenus dans un triangle distingu\'e 
$(X,Y,Z,f,g,h)$ o\`u $Z$ est un objet de $\cB$. $\varphi(\cB)$ est un 
syst\`eme multiplicatif satur\'e. 

Soit $S$ un syst\`eme multiplicatif satur\'e; on d\'esigne par $\psi(S)$ la 
sous-cat\'egorie pleine engendr\'ee par les objets $Z$ contenus dans un 
triangle distingu\'e $(X,Y,Z,f,g,h)$ o\`u $f$ est un \'el\'ement de $S$. La 
sous-cat\'egorie $\psi(S)$ est une sous-cat\'egorie \'epaisse. 

Le r\'esultat principal de ce num\'ero est alors le suivant: $\varphi$ est un 
isomorphisme (conservant la relation d'ordre d\'efinie par l'inclusion) de 
$\sN$ sur $\sS$. L'isomorphisme inverse n'est autre que $\psi$. 










\subsection{Probl\`emes universels}\label{VIII:2-2}

Soient $\cA$ et $\cA'$ deux cat\'egories triangul\'ees et $F$ un foncteur 
exacte de $\cA$ dans $\cA'$. Soient $S(F)$ l'ensemble des morphismes de $\cA$ 
qui sont transform\'es par $F$ en isomorphismes et $\cB(F)$ la sous-cat\'egorie 
pleine engendr\'ee par les objets de $\cA$ qui sont transform\'es par $F$ en 
objets nuls de $\cA'$. $S(F)$ est un syst\`eme multiplicatif satur\'e et 
$\cB(F)$ est une sous-cat\'egorie \'epaisse. De plus $S(F) = \varphi(\cB(F))$. 
Soient alors $\cA$ une cat\'egorie triangul\'ee, $\cB$ une sous-cat\'egorie 
\'epaisse et $S=\varphi(\cB)$ le syst\`eme multiplicatif satur\'e 
correspondant. Les deux probl\`emes universels ci-dessous sont \'equivalents. 


\paragraph{Probl\`eme 1}
\phantomsection\hypertarget{VIII:prob1}{}
Trouver une cat\'egorie triangul\'ee $\cA/\cB$ et un foncteur exact 
$Q:\cA\to \cA/\cB$ tels que tout foncteur exact de $\cA$ dans une cat\'egorie 
triangul\'ee $\cA'$, transformant les objets de $\cB$ en objets nuls de 
$\cA'$, admette, de mani\`ere unique, une factorisation de la forme $G\circ Q$ 
o\`u $G$ est un foncteur exact. 


\paragraph{Probl\`eme 2}
\phantomsection\hypertarget{VIII:prob2}{}
Trouver une cat\'egorie triangul\'ee $\cA_S$ et un foncteur exact 
$Q:\cA\to \cA_S$ tels que tout foncteur exact de $\cA$ dans une cat\'egorie 
triangul\'ee $\cA'$, transformant les morphismes de $S$ en isomorphismes de 
$\cA'$, admette, de mani\`ere unique, une factorisation de la forme 
$G\circ Q$ o\`u $G$ est un foncteur exact. 

(Le r\'edacteur prie le lecteur de bien vouloir l'excuser pour l'emploi de ce 
langage d\'esuet et r\'etrograde.) Nous allons montrer, dans le num\'ero 
suivant, que le \hyperlink{VIII:prob2}{probl\`eme 2} admet une solution. Donc 
le \hyperlink{VIII:prob1}{probl\`eme 1} correspondant admet la m\^eme solution. 










\subsection{Calcul de fraction}\label{VIII:2-3}





\subsubsection{}\label{VIII:2-3-1}

Soient $\cA$ un cat\'egorie triangul\'ee, $S$ un syst\`eme multiplicatif 
satur\'e. D\'esignons par $\cA[S^{-1}]$ la cat\'egorie de fraction de $\cA$ 
pour l'ensemble de morphismes $S$ et $Q$ le foncteur canonique de $\cA$ dans 
$\cA[S^{-1}]$. Le couple $(\cA[S^{-1}],Q)$ r\'esout le probl\`eme suivant: Tout 
foncteur de $\cA$ dans une cat\'egorie quelconque (non n\'ecessairement 
additive) transformant morphisme de $S$ en isomorphisme, se factorise, d'une 
mani\`ere unique, par $(\cA[S^{-1}],Q)$. Un tel couple existe toujours, sans 
hypoth\`ese sur l'ensemble de morphismes $S$ \cite[1.1]{gz67}\footnote{The 
original text references [C.G.G] here. Most likely, this is the lost work 
\emph{Cat\'egories et foncteurs} by Chevalley, Gabriel and Grothendieck, 
referenced in\cite{sc72}. }. Cependant l'ensemble 
de morphismes $S$, \'etant un syst\`me multiplicatif satur\'e, poss\`ede les 
propri\'et\'es \hyperlink{VIII:FR1}{FR1}, \hyperlink{VIII:FR2}{FR2}, 
\hyperlink{VIII:FR3}{FR3}. Par suite, d'apr\`es \cite{gz67}, la cat\'egorie 
$\cA[S^{-1}]$ peut s'obtenir par un calcul de fractions, \`a droite ou \`a 
gauche. Nous allons rappeler comment on obtient $\cA[S^{-1}]$ par un calcul de 
fractions \`a droite. (Le calcul de fractions \`a gauche s'obtient par le 
proc\'ed\'e du renversement des fl\`eches.) 





\subsubsection{}\label{VIII:2-3-2}

Les objets de $\cA[S^{-1}]$ sont le objets de $\cA$. 
\begin{itemize}
  \item Pour tout objet $X$ de $\cA$, d\'esignons par $S_X$ la cat\'egorie des 
    fl\`eches appartenant \`a $S$, ayant pour but $X$. A tout objet $Y$ de 
    $\cA$, on associe, fonctoriellement en $Y$, le foncteur suivant sur 
    $S_X^\circ$ \`a valeur dans la cat\'egorie des ensembles: 
    ($S_X^\circ$ d\'esigne la cat\'egorie oppos\'ee) 
    \[
      H_Y:s\mapsto \hom(\operatorname{source}(s),Y) \text{.}
    \]
    On pose alors: 
    \[
      \hom_{\cA[S^{-1}]}(X,Y) = \varinjlim_{S_X^\circ} H_Y \text{.}
    \]
    Pour tout ce qui concerne les limites inductives, on se r\'ef\'erera \`a 
    \cite{ar62}. 
    
    On constatera alors, avec plaisir en utilisant \hyperlink{VIII:FR1}{FR1}, 
    \hyperlink{VIII:FR2}{FR2}, \hyperlink{VIII:FR3}{FR3} que la cat\'egorie 
    $S_X^\circ$ poss\`ede les propri\'et\'es L1, L2, L3. Les limites inductives 
    poss\`edent donc toutes les bonnes propri\'et\'es des limites inductives 
    sur les ensembles ordonn\'es filtrants. 
  \item Soient $X$, $Y$, $Z$, trois objets de $\cA$, $a$ un \'el\'ement de 
    $\hom_{\cA[S^{-1}]}(X,Y)$ et $b$ un \'el\'ement de 
    $\hom_{\cA[S^{-1}]}(Y,Z)$. Soient $s$ un objet de $S_X$, $\underline a$ un 
    \'el\'ement de $\hom(\operatorname{source}(s),Y)$ dont l'image est $a$ et 
    un objet de $S_y$, $\underline b$ un \'el\'ement de 
    $\hom(\operatorname{source}(t),Z)$ dont l'image est $b$. On a alors un 
    diagramme: 
    \begin{equation*}\tag{1}\label{VIII:eq:2-3-2-1}
    \xymatrix{
      & \bullet \ar[dl]_-s \ar[dr]^-{\underline a} 
        & 
        & \bullet \ar[dl]_-t \ar[dr]^-{\underline b} \\
      X 
        & 
        & Y 
        & 
        & Z 
    }
    \end{equation*}
    que, d'apr\`es \hyperlink{VIII:FR2}{FR2}, on peut compl\'eter en un 
    diagramme: 
    \[\xymatrix{
      & & \bullet \ar[dl]_-{t'} \ar[dr]^-{\underline c} \\
      & \bullet \ar[dl]_-s \ar[dr]^-{\underline a} 
        & & \bullet \ar[dl]_-t \ar[dr]^-{\underline b} \\
      X 
        & & Y 
        & & Z 
    }\]
    Notons $b\circ a$ l'image dans $\hom_{\cA[S^{-1}]}(X,Z)$ du morphisme 
    $\underline{b\circ c}$. On v\'erifie que gr\^ace aux propri\'et\'es 
    \hyperlink{VIII:FR1}{FR1}, \hyperlink{VIII:FR2}{FR2}, 
    \hyperlink{VIII:FR3}{FR3}, $b\circ a$ ne d\'epend pas des repr\'esentants 
    $\underline a$ et $\underline b$ choisis et qu'il ne d\'epend pas non plus 
    de la mani\`ere de compl\'eter le diagramme \eqref{VIII:eq:2-3-2-1}. On 
    v\'erifie de plus qu'on a ainsi d\'efini une cat\'egorie $\cA[S^{-1}]$. 
\end{itemize}

Nous noterons $Q$ le foncteur \'evident: $\cA\to \cA[S^{-1}]$. 

$\cA$ \'etant une cat\'egorie additive, $\cA[S^{-1}]$ est une cat\'egorie 
additive. De plus, on d\'emontre en utilisant \hyperlink{VIII:FR4}{FR4} qu'il 
existe un et un seul foncteur de translation sur $\cA[S^{-1}]$, que nous 
noterons $T$, v\'erifiant la relation: 
\[
  Q\circ T = T\circ Q \text{.}
\]
Enfin, \`a l'aide de \hyperlink{VIII:FR5}{FR5}, on d\'emontre qu'il existe une 
et une seule structure triangul\'ee sur $\cA[S^{-1}]$ telle que le foncteur $Q$ 
soit exact. Les triangles distingu\'es de $\cA[S^{-1}]$ ne sont qutre que les 
triangles isomorphes aux images, par $Q$, des triangles distingu\'es de $\cA$. 
En d\'esignant par $\cA_S$, la cat\'egorie $\cA[S^{-1}]$ muni de cette 
structure triangul\'ee, on d\'emontre sans difficult\'e que le couple 
$(\cA_S,Q)$ est une solution au \hyperlink{VIII:prob2}{probl\`eme 2}. 





\begin{definition}\label{VIII:2-3-3}
Soient $\cA$ un cat\'egorie triangul\'ee, $\cB$ une sous-cat\'egorie \'epaisse 
de $\cA$, $(Q,\cA/\cB)$ la solution du \hyperlink{VIII:prob1}{probl\`eme 1} 
(\S \ref{VIII:2-2}). $\cA/\cB$ sera appel\'ee la \emph{cat\'egorie quotient} de 
$\cA$ par $\cB$. $Q$ sera appel\'e le foncteur canonique de passage au 
quotient. Plus g\'en\'eralement soit $\cN$ une sous-cat\'egorie triangul\'ee de 
$\cA$ et soit $\underline\cN$ la plus petit sous-cat\'egorie \'epaisse 
contenant $\cN$. La cat\'egorie quotient de $\cA$ par $\cN$: $\cA/\cN$ sera par 
d\'efinition $\cA/\underline\cN$. 
\end{definition}










\subsection{Propri\'et\'es des cat\'egories quotients}\label{VIII:2-4}





\subsubsection{}\label{VIII:2-4-1}

Soient $\cA$ une cat\'egorie triangul\'ee et $H$ un foncteur cohomologique \`a 
valeur dans une cat\'egorie ab\'elienne $\cG$. Soient $\cB_H$ la 
sous-cat\'egorie pleine engendr\'ee par les objets dont tous les translat\'es 
sont transform\'es par $H$ en objets nuls de $\cG$ et $S_H$ l'ensemble des 
morphismes dont tous les translat\'es sont transform\'es par $H$ en 
isomorphismes de $\cG$. $\cB_H$ est une sous-cat\'egorie \'epaisse de $\cA$; 
$S_H$ est un syst\`eme multiplicatif satur\'e. De plus $S_H=\varphi(\cB_H)$ 
(\S \ref{VIII:2-1}). Le foncteur $H$ se factorise d'un mani\`ere unique par 
$(Q,\cA/\cB_H)$. 





\begin{theorem}\label{VIII:2-4-2}
Soient $\cA$ une cat\'egorie triangul\'ee, $\cB$ un sous-cat\'egorie 
triangul\'ee, $\cN$ une sous-cat\'egorie \'epaisse de $\cA$, $S=\varphi(\cN)$ 
le syst\`eme multiplicatif satur\'e correspondant. 
\begin{enumerate}[(a)]
  \item La cat\'egorie $\cN\cap\cB$ est une sous-cat\'egorie \'epaisse de 
    $\cB$. Le syst\`eme multiplicatif correspondant est $S\cap \cB$. 
  \item Les deux propri\'et\'es suivantes sont \'equivalentes: 
    \begin{enumerate}[(i)]
      \item Pour tout objet $X$ de $\cB$ et tout morphisme $s:R\to X$ o\`u $s$ 
        est un \'el\'ement de $S$, il existe un morphisme $s':R'\to R$ o\`u 
        $s\circ'$ est un \'el\'ement de $S\cap \cB$; et 
        $R'\in\operatorname{Ob}\cB$. 
      \item Tout morphisme d'un objet $X$ de $\cB$ dans un objet $Y$ de $\cN$ 
        se factorise par un objet de $\cN\cap \cB$. 
    \end{enumerate}
  \item Les propri\'et\'es (i)' et (ii)' obtenues \`a partir de (i) et (ii) en 
    renversant les fl\`eches sont \'equivalents. 
  \item Si les propriet\'es (i) et (ii) sont v\'erifi\'ees ou bien si les 
    propri\'et\'es (i)' et (ii)' sont v\'erifi\'ees, le foncteur canonique 
    \[\xymatrix{
      \cB/\cN\cap \cB \ar[r] & \cA/\cN
    }\]
    est fid\`ele. 
    
    Comme ce foncteur est injectif sur les objets, il r\'ealise 
    $\cB/\cN\cap\cB$ comme sous-cat\'egorie de $\cA/\cN$. 
  \item Si de plus $\cB$ est une sous-cat\'egorie pleine de $\cA$, 
    $\cB/\cN\cap \cB$ est une sous-cat\'egorie pleine de $\cA/\cN$. 
\end{enumerate}
\end{theorem}





\begin{corollary}\label{VIII:2-4-3}
Soient $\cA\subset \cB\subset \cC$ trois cat\'egories triangul\'ees, $\cA$ sous 
cat\'egorie \'epaisse de $\cB$, $\cB$ sous-cat\'egorie \'epaisse de $\cC$. 
Alors $\cA$ est une sous-cat\'egorie \'epaisse de $\cC$, $\cB/\cA$ est une 
sous-cat\'egorie \'epaisse de $\cC/\cA$. Le foncteur canonique 
$\cC/\cB \to (\cC/\cA)/(\cB/\cA)$ est un isomorphisme. 
\end{corollary}










\subsection{Propri\'et\'es du foncteur de passage au quotient}\label{VIII:2-5}

Soient $\cA$ un cat\'egorie triangul\'ee, $\cB$ un sous-cat\'egorie \'epaisse, 
$S$ le syst\`eme multiplicatif correspondant, $Q:\cA\to \cA/\cB$ le foncteur 
canonique de passage au quotient. 





\begin{proposition}\label{VIII:2-5-1}
$X$ et $Y$ \'etant des objets de $\cA$, les assertions suivantes sont 
\'equivalentes: 
\begin{enumerate}[(a)]
  \item Tout morphisme $f:X\to Y$, tel qu'il existe un morphisme $s:Z\to X$, de 
    $S$, v\'erifiant $f\circ s=0$, est un morphisme nul.
  \item Tout morphisme $f:X\to Y$, tel qu'il existe un morphisme $s:Y\to Z$, 
    $s\in S$, v\'erifiant $s\circ f=0$, est un morphisme nul. 
  \item Tout morphisme $f:X\to Y$ qui se factorise par un objet de $\cB$ est 
    nul. 
  \item L'application canonique: 
    \[
      \hom_\cA(X,Y) \to \hom_{\cA/\cB}(Q(X),Q(Y))
    \]
    est injective. 
\end{enumerate}
\end{proposition}





\begin{proposition}\label{VIII:2-5-2}
Soient $X$ et $Y$ deux objets de $\cA$. Les assertions suivantes sont 
\'equivalentes: 
\begin{enumerate}[(a)]
  \item Tout diagramme ($s\in S$): 
    \[\xymatrix{
      & Z \ar[dl]_-s \ar[dr]^-f \\
      X 
        & & Y 
    }\]
    se compl`ete en un diagramme: $(s,t\in S$): 
    \[\xymatrix{
      & Z' \ar[d]_-t \\
      & Z \ar[dl]_-s \ar[dr]^-f \\
      X 
        & & Y 
    }\]
    o\`u $f\circ t$ se factorise par $(X,s\circ t)$. 
  \item Assertion obtenue en renversant le sens de fl\`eches dans (a) et en 
    permutant $X$ et $Y$. 
  \item Tout diagramme: 
    \[\xymatrix{
      X \ar[r]^-f 
        & N \ar[r]^-g 
        & Y 
    }\]
    o\`u $g\circ f=0$ et o\`u $N$ est un objet de $\cB$, se compl\`ete en un 
    diagramme: 
    \[\xymatrix{
      & N' \ar[d]^-i \\
      X \ar[ur]^-h \ar[r]^-f 
        & N \ar[r]^-g 
        & Y 
    }\]
    o\`u $f=i\circ h$ et o\`u $g\circ i=0$. 
  \item Assertion obtenue en changeant le sens des fl\`eches dans (c) et en 
    permutant $X$ et $Y$. 
  \item L'application canonique: 
    \[\xymatrix{
      \hom_\cA(X,Y) \ar[r]^-Q 
        & \hom_{\cA/\cB}(Q(X),Q(Y)) 
    }\]
    est surjective. 
\end{enumerate}
\end{proposition}





\begin{proposition}\label{VIII:2-5-3}
Soit $X$ un objet de $\cA$. Les assertions suivantes sont \'equivalentes: 
\begin{enumerate}[(a)]
  \item Pour tout objet $Y$ de $\cA$, l'application canonique 
    \[\xymatrix{
      \hom_\cA(X,Y) \ar[r]^-Q 
        & \hom_{\cA/\cB}(Q(X),Q(Y)) 
    }\]
    est un isomorphisme. 
  \item Tout morphisme de $X$ dans un objet de $\cB$ est nul. 
\end{enumerate}
De m\^eme, les deux assertions suivantes sont \'equivalentes: 
\begin{enumerate}[(a)']
  \item Pour tout objet $Y$ de $\cA$, l'application canonique: 
    \[\xymatrix{
      \hom_\cA(X,Y) \ar[r]^-Q 
        & \hom_{\cA/\cB}(Q(X),Q(Y)) 
    }\]
    est un isomorphisme. 
  \item Tout morphisme d'un objet de $\cB$ dans $X$ est nul. 
\end{enumerate}
\end{proposition}





\begin{definition}\label{VIII:2-5-4}
Tout objet poss\'edant les propri\'et\'es \'equivalentes (a) et (b) de la 
proposition pr\'ec\'edente, sera appel\'e: \emph{objet libre \`a gauche sur 
$\cA/\cB$} ou bien encore objet \emph{$Q$-libre \`a gauche}. De m\^eme, tout 
objet poss\'edant les propri\'et\'es \'equivalentes (a)' et (b)' sera appel\'e 
\emph{objet libre \`a droite sur $\cA/\cB$} ou encore objet 
\emph{$Q$-libre \`a droite}. Il est d\'efini \`a isomorphisme pr\`es par la 
connaissance de $Q(X)\in\operatorname{Ob}{\cA/\cB}$. 
\end{definition}





\subsection{Foncteurs adjoints au foncteur de passage au quotient}\label{VIII:2-6}





\begin{definition}\label{VIII:2-6-1}
Deux sous-cat\'egories triangul\'ees $\cN$ et $\cN'$ d'une cat\'egorie 
triangul\'ee $\cA$ sont dites \emph{orthagonales} si pour tout objet $X$ de 
$\cN$ et tout objet $Y$ de $\cN'$, on a: 
\[
  \hom_\cA(X,Y) = 0 
\]
$\cN'$ sera dite alors orthogonale \`a droite \`a $\cN$ et $\cN$ orthogonale 
\`a gauche \`a $\cN'$. 
\end{definition}





\begin{proposition}\label{VIII:2-6-2}
Soit $\cN$ une sous-cat\'egorie triangul\'ee d'une cat\'egorie triangul\'ee 
$\cA$. La cat\'egorie pleine $\cN^\bot$ (resp. $^\bot\cN$) engendr\'ee par les 
objets $X$ de $\cA$ tels que pour tout objet $Y$ de $\cN$ on ait 
$\hom_\cA(Y,X)=0$ (resp. $\hom_\cA(X,Y)=0$), est une sous-cat\'egorie \'epaisse 
de $\cA$. La cat\'egorie $\cN^\bot$ est appel\'ee par abus de langage, 
l'orthogonale \`a droite de $\cN$. 
\end{proposition}

Cette proposition nous permet de traduire la proposition \ref{VIII:2-5-3}. 





\begin{proposition}\label{VIII:2-6-3}
Soit $\cB\subset\cA$, une sous-cat\'egorie \'epaisse d'une cat\'egorie 
triangul\'ee $\cA$. La cat\'egorie pleine engendr\'ee par les objets libre \`a 
droite sur $\cA/\cB$ (resp. \`a gauche), n'est autre que l'orthogonale \`a 
droite (resp. \`a gauche) de $\cB$. 
\end{proposition}





\begin{proposition}\label{VIII:2-6-4}
Soient $\cA$ une cat\'egorie triangul\'ee, $\cB$ une sous-cat\'egorie 
\'epaisse de $\cA$, $\cB^\bot$ l'orthogonale \`a droite de $\cB$, $S(\cB)$, 
$s(\cB^\bot)$ les syst\`emes multiplicatifs correspondants, $Q$ et $Q^\bot$ les 
foncteurs canoniques de passage au quotient. Consid\'erons pour un objet $x$ de 
$\cA$, les propri\'et\'es suivantes: 
\begin{enumerate}[(i)]
  \item $X$ s'envoie par un morphisme de $S(\cB)$ dans un objet de $\cB^\bot$. 
  \item $X$ reçoit par un morphisme de $S(\cB^\bot)$ un objet de $\cB$. 
  \item La cat\'egorie $S_X(\cB)$ des fl\`eches de $S(\cB)$ de source $X$, 
    admet un objet final. 
  \item La cat\'egorie $S^X(\cB^\bot)$ des fl\`eches de $S(\cB^\bot)$ de but 
    $X$, admet un objet initial. 
  \item La cat\'egorie $\cB/X$ des objets de $\cB$ au-dessus de $X$, admet un 
    objet final. 
  \item La cat\'egorie $\cB^\bot/X$ des objets de $\cB^\bot$ au-dessous de $X$ 
    admet un objet initial. 
\end{enumerate}
On a alors 
\[\xymatrix@=0.5cm{
  & & & & (iv) \\
  (i) \ar@{<=>}[r] 
    & (ii) \ar@{<=>}[r] 
    & (iii) \ar@{<=>}[r] 
    & (v) \ar@{=>}[ur] \ar@{=>}[dr] \\
  & & & & (vi) 
}\]

Si de plus $^\bot(\cB^\bot)=\cB$ alors toutes les propri\'et\'es sont 
\'equivalentes. 
\end{proposition}

Soient $\cB$ une sous-cat\'egorie \'epaisse d'une cat\'egorie triangul\'ee 
$\cA$, le foncteur d'injection $i$ de $\cB$ dans $\cA$ et $Q$ le foncteur de 
passage au quotient de $\cA$ dans $\cA/\cB$, $\cB^\bot$ l'orthogonale \`a 
droite de $\cB$, les foncteurs correspondant $i^\bot$ et $Q^\bot$. On d\'eduit 
imm\'ediatement de la proposition \ref{VIII:2-6-4} les propositions suivantes: 





\begin{proposition}\label{VIII:2-6-5}
Les deux propri\'et\'es suivantes sont \'equivalentes: 
\begin{enumerate}[(i)]
  \item Le foncteur $i$ admet un adjoint \`a droite. 
  \item Le foncteur $Q$ admet un adjoint \`a droite. 
\end{enumerate}
\end{proposition}

La proposition est encore vraie lorsqu'on remplace le mot droite par le omt 
gauche. 





\begin{proposition}\label{VIII:2-6-6}
Les deux propri\'et\'es suivantes sont \'equivalentes: 
\begin{enumerate}[(i)]
  \item Le foncteur $i$ admet un adjoint \`a droite. 
  \item Le foncteur $i^\bot$ admet un adjoint \`a gauche. L'orthogonale \`a 
    gauche de la cat\'egorie $\cB^\bot$ est \'egale \`a $\cB$. 
\end{enumerate}
\end{proposition}





\begin{proposition}\label{VIII:2-6-7}
Supposons v\'erifi\'ees les propri\'et\'es des propositions \ref{VIII:2-6-5} et 
\ref{VIII:2-6-6}. Soient $i^\ast$ et $Q^\ast$ les adjoints \`a droite de $i$ et 
$Q$. De m\^eme soient $^\ast i^\bot$ et $^\ast Q^\bot$ les adjoints \`a gauche 
de $i^\bot$ et de $Q^\bot$. Tous ces foncteurs sont exactes. De plus: 

Il existe un isomorphisme fonctoriel 
$Q^\ast\circ Q\iso i^\bot\circ ^\ast i^\bot$ tel que le diagramme ci-apr\`es 
soit commutatif: 
\[\xymatrix{
  & \operatorname{id} \ar[dl] \ar[dr] \\
  Q^\ast \circ Q \ar[rr]^\sim 
    & & i^\bot\circ ^\ast i^\bot 
}\]
(les fl\`eches obliques sont d\'efinies par les propri\'et\'es d'adjonction).

Il existe de m\^eme un isomorphisme 
$^\ast Q^\bot \circ Q^\bot \iso i\circ i^\ast$ tel que le diagramme suivant soit 
commutatif: 
\[\xymatrix{
  ^\ast Q^\bot\circ Q^\bot \ar[rr]^-\sim \ar[dr] 
    & & i\circ i^\ast \ar[dl] \\
  & \operatorname{id}{}
}\]

Enfin, il existe un morphisme fonctoriel de degr\'e $1$: 
$\partial:i^\bot\circ ^\ast i^\bot \to i\circ i^\ast$ tel que le le triangle 
suivant soit distingu\'e $(\deg\partial=1$): 
\[\xymatrix{
  i^\bot \circ ^\ast i^\bot \ar[rr]^-\partial 
    & & i\circ i^\ast \ar[dl] \\
  & \operatorname{id}{} \ar[ul] 
}\]

Un tel morphisme $\delta$ est unique et v\'erifie 
$\delta(T X_0) = - T \delta(X)$ ($X$ objet $q\circ q$, $T$ translation). 
\end{proposition}















\section{Les cat\'egories d\'eriv\'ees d'une cat\'egorie ab\'elienne}\label{VIII:3}





\subsection{Les cat\'egories d\'eriv\'ees}\label{VIII:3-1}

On utilise les notations du \ref{VIII:1-2-3}. On utilisera de plus les 
notations suivantes: 

\subsubsection{Notations}\label{VIII:3-1-1}

Soit $\cA$ une cat\'egorie ab\'elienne. On pose: 
\begin{align*}
  \D(\cA) &= \eK^{\infty,\infty}(\cA)/\eK^{\infty,\varnothing}(\cA) \\
  \D^b(\cA) &= \eK^{b,b}(\cA)/\eK^{b,\varnothing}(\cA) \\
  \D^+(\cA) &= \eK^{+,+}(\cA) / \eK^{+,\varnothing}(\cA) \\
  \D^-(\cA) &= \eK^{-,-}(\cA) / \eK^{-,\varnothing}(\cA) \text{.}
\end{align*}





\begin{proposition}\label{VIII:3-1-2}
Les foncteurs naturels qui figurent dans le diagramme suivant: 
\[\xymatrix@=0.5cm{
  & \D(\cA) \\
  \D^+(\cA) \ar[ur] 
    & & \D^-(\cA) \ar[ul] \\
  & \D^b(\cA) \ar[ul] \ar[ur] 
}\]
sont pleinement fid\`eles. 

Les foncteurs naturels: 
\begin{align*}
  \D^b(\cA) &= \eK^{b,b}(\cA)/\eK^{b,\varnothing}(\cA) \to \eK^{+,b}(\cA)/\eK^{+,\varnothing}(\cA) \\
  \D^b(\cA) &= \eK^{b,b}(\cA)/\eK^{b,\varnothing}(\cA) \to \eK^{-,b}(\cA) / \eK^{-,\varnothing}(\cA) 
\end{align*}
sont des \'equivalences de cat\'egories. 

Les foncteurs du diagramme: 
\[\xymatrix@=0.5cm{
  & \D(\cA) \\
  \D^+(\cA) \ar[ur] 
    & & \D^-(\cA) \ar[ul] \\
  & \D^b(\cA) \ar[ul] \ar[ur]
}\]
\'etant injectifs sur les objets, r\'ealisent les cat\'egories $\D^b(\cA)$, 
$\D^+(\cA)$, $\D^-(\cA)$ comme sous-cat\'egories pleines de $\D(\cA)$. 
\end{proposition}

La preuve de cette proposition se trouve au th\'eor\`eme \ref{VIII:2-4-2}. 





\begin{definition}\label{VIII:3-1-3}
La cat\'egorie $\D(\cA)$ sera appel\'ee la \emph{cat\'egorie d\'eriv\'ee} de la 
cat\'egorie $\cA$. Les objets de $\D(\cA)$ isomorphes aux objets de $\D^b(\cA)$ 
seront appel\'es les objets \emph{born\'es} de $\D(\cA)$. Les objets de 
$\D(\cA)$ isomorphes aux objets de $\D^+(\cA)$ seront appel\'es les objets 
\emph{limit\'es \`a gauche}, les objets de $\D(\cA)$ isomorphes aux objets de 
$\D^-(\cA)$, seront appel\'es les objets \emph{limit\'es \`a droite}. 
\end{definition}

Nous d\'esignerons par $D$ le foncteur canonique: 
\[
  D : \eC(\cA) \to \D(\cA) \text{.}
\]
La ``restriction'' de $D$ \`a la sous-cat\'egorie pleine $\cA$ de $\eC(\cA)$ 
sera encore not\'e $D$. Le foncteur $D$ restrient \`a $\cA$ est pleinement 
fid\`ele. Le foncteur $D$ se factorise d'une mani\`ere unique par $\eK(\cA)$. 
Soient $X$ et $Y$ deux objets de $\cA$. 

On v\'erifie sans difficult\'e que pour tout $m\geqslant 0$, les groupes: 
\[
  \hom_{\D(\cA)}(D(X),T^{-m} D(Y)) \qquad \text{($T$ est le foncteur translation)} 
\]
sont nuls. La dimension cohomologique de $\cA$ sera le plus petit des entiers 
$n$ tels que pour tout $m>n$ on ait: 
\[
  \hom_{\D(\cA)}(D(X),T^m D(Y)) = 0 \qquad\text{pout tout couple $X,Y$ d'objets de $\cA$.} 
\]
S'il n'y a pas de tels entiers, on dira que la dimension cohomologique de $\cA$ 
est infinie. 

D\'esignons par $\eI^+(\cA)$ (resp. $\eI(\cA)$) la sous-cat\'egorie triangul\'ee 
pleine de $\eK^+(\cA)$ (resp. de $\eK(\cA)$) d\'efinie par les complexes dont 
les objets sont injectifs en tout degr\'e. 





\begin{proposition}\label{VIII:3-1-4}
Supposons que la cat\'egorie $\cA$ poss\`ede suffisamment d'injectifs. Le 
foncteur naturel: 
\[
  Q^+:\eK^+(\cA) \to \eD^+(\cA) 
\]
induit une \'equivalence de cat\'egorie: 
\[
  \eI^+(\cA) \to \eD^+(\cA)  \text{.}
\]
Le foncteur quasi-inverse est un adjoint \`a droite de $Q^+$. 
\end{proposition}

Si de plus $\cA$ est de dimension cohomologique finie, l'assertion 
pr\'ec\'edente est encore vraie lorsqu'on y supprime les exposants $+$. 

On a \'enonc\'e analogue concernant les projectifs. 

La d\'emonstration de la proposition \ref{VIII:3-1-4} s'appuie sur 
\ref{VIII:2-6-4}. 

La structure triangul\'ee de $\eD(\cA)$, est pr\'ecis\'ee par la proposition 
suivante. 

Soit $\eE(\cA)$ la cat\'egorie des suites exactes \`a trois termes de 
$\eC(\cA)$. La cat\'egorie $\mathsf{SSS}(\cA)$ des suites exactes 
semi-scind\'ees (\S\ref{VIII:1-2-4}) est une sous-cat\'egorie pleine de 
$\eE(\cA)$. On a d\'efini (Loc. cit.) un foncteur: 
\[
  \rho:\mathsf{SSS}(\cA) \to \mathsf{TrK}(\cA) 
\]
o\`u $\mathsf{TrK}(\cA)$ d\'esigne la cat\'egorie des triangles distingu\'es 
de $\eK(\cA)$. On en d\'eduit un foncteur 
\[
  \sigma : \mathsf{SSS}(\cA) \to \mathsf{TrD}(\cA) 
\]
o\`u $\mathsf{TrD}(\cA)$ d\'esigne la cat\'egorie des triangles distingu\'es 
de $\eD(\cA)$. 





\begin{proposition}\label{VIII:3-1-5}
Il existe un et un seul foncteur: 
\[
  \Pi:\eE(\cA) \to \mathsf{TrD}(\cA) 
\]
de la forme ($\deg \delta(S)=1$): 
\[
(S=0 \to X^\bullet \to Y^\bullet \to Z^\bullet \to 0) 
\mapsto 
\xymatrix{
  & D(Z^\bullet) \ar[dl]_-{\delta(S)} \\
  D(X^\bullet) \ar[rr]^-{D(u)} 
    & & D(Y^\bullet) \ar[ul]_-{D(v)} 
}
\]
dont la restriction \`a $\mathsf{SSS}(\cA)$ soit $\sigma$. \emph{Ce foncteur 
est essentiellement surjectif.}
\end{proposition}





\begin{proposition}\label{VIII:3-1-6}
Le foncteur cohomologique canonique: 
\[
  \h^0:\eK(\cA) \to \cA 
\]
se factorise d'une mani\`ere unique par un foncteur que nous noterns encore 
$\h^0$ 
\[
  \h^0:\eD(\cA) \to \cA \text{.}
\]
Sur $\eD(\cA)$ le foncteur $\h^0$ et ses translat\'es $\h^i$ forment un 
syst\`eme conservatif, i.e. un morphisme $f:X^\bullet \to Y^\bullet$ et un 
isomorphisme si et seulement si pour tout entier $i$ 
\[
  \h^i(f):\h^i(X) \to \h^i(Y) 
\]
est un isomorphisme. 
\end{proposition}





\begin{definition}\label{VIII:3-1-7}
Un morphisme $f:X^\bullet \to Y^\bullet$ de $\eC(\cA)$ (resp. de $\eK(\cA)$) 
est appel\'e un \emph{quasi-isomorphisme} lorsque pour tout entier $i$ 
\[
  \h^i(f):\h^i(X) \to \h^i(Y) 
\]
est un isomorphisme. 
\end{definition}

D'apr\`es la proposition pr\'ec\'edente les quasi-isomorphismes sont les 
morphismes qui deviennent des isomorphismes dans $\eD(\cA)$. 





\begin{definition}\label{VIII:3-1-8}
Soient $\sigma$ un ensemble d'objets de $\cA$, $X^\bullet$ un objet de 
$\eC(\cA)$ (resp. $\eK(\cA)$). Une re\'solution droite (reps. gauche) de type 
$\sigma$ de $X^\bullet$, est un quasi-isomorphisme $X^\bullet \to V^\bullet$ 
(resp. $V^\bullet \to X^\bullet$) o\`u $V^\bullet$ est un complexe dont tous 
les objets appartiennent \`a $\sigma$. On dira \emph{r\'esolution injective} 
(resp. projective) au lieu de r\'esolution droite (resp. gauche) de type 
injectif (resp. projectif). 
\end{definition}

De mani\`ere g\'en\'erale, on se permettra de supprimer les mots ``droite'' ou 
``gauche'' lorsqu'aucune ambiguit\'e n'en r\'esultera. 





\begin{proposition}\label{VIII:3-1-9}
\begin{enumerate}
  \item Supposons que tout objet de $\cA$ s'injecte dans un objet de $\sigma$ 
    (resp. soit quotient d'un objet de $\sigma$). Alors tout objet de 
    $\eC^+(\cA)$ (resp. de $\eC^-(\cA)$) admet une r\'esolution droite 
    (resp. gauche) de type $\sigma$ dans $\eC^+(\cA)$ (resp. dans 
    $\eC^-(\cA)$). 
  \item Supposons de plus qu'il existe un entier $n$ tel que pour toute suite 
    exacte: 
    \begin{align*}
      Y^0 &\to Y^1 \to \cdots \to Y^{n-1} \to Y^n \to 0 \\
      \text{(resp. } 0 &\to Y^n \to Y^{n-1} \to \cdots \to Y^1 \to Y^0 \text{)} 
    \end{align*}
    d'objets de $\cA$ ou pout tout $i$, $0\leqslant i\leqslant n-1$, $Y^i$ est 
    un objet de $\sigma$, l'objet $Y^n$ soit un objet de $\sigma$. Alors tout 
    objet de $\eC(\cA)$ admet une r\'esolution droite (resp. gauche) de type 
    $\sigma$. 
\end{enumerate}
\end{proposition}










\subsection{Etude des $\ext$}\label{VIII:3-2}

Soient $X^\bullet$ et $Y^\bullet$ deux objets de $\eC(\cA)$ (resp. $\eK(\cA)$) 
et $D:\eC(\cA) \to \eD(\cA)$ (resp. $Q:\eK(\cA) \to \eD(\cA)$) le foncteur 
canonique. 





\begin{definition}\label{VIII:3-2-1}
On appelle \emph{$i$-\`eme hyper-ext} et on note: 
\[
  (X^\bullet,Y^\bullet)\mapsto \ext^i(X^\bullet,Y^\bullet) 
\]
le foncteur: $\hom_{\eD(\cA)}(D(X^\bullet),T^i D(Y^\bullet))$ (resp. 
$\hom_{\eD(\cA)}(Q(X^\bullet),T^i Q(Y^\bullet))$). 
\end{definition}





\begin{proposition}\label{VIII:3-2-2}
Soit $0 \to X^\bullet \to Y^\bullet \to Z^\bullet \to 0$ une suite exacte de 
$\eC(\cA)$. Soit $V^\bullet$ un objet de $\eC(\cA)$. On a les suites exactes 
illimit\'ees: 
\[\xymatrix@=0.5cm{
  \cdots \ar[r] 
    & \ext^i(V^\bullet,X^\bullet) \ar[r] 
    & \ext^i(V^\bullet,Y^\bullet) \ar[r] 
    & \ext^i(V^\bullet,Z^\bullet) \ar[r]^-\delta 
    & \ext^{i+1}(V^\bullet,X^\bullet) \ar[r] 
    & \cdots \\
  \cdots \ar[r] 
    & \ext^i(Z^\bullet,V^\bullet) \ar[r] 
    & \ext^i(Y^\bullet,V^\bullet) \ar[r] 
    & \ext^i(X^\bullet,V^\bullet) \ar[r]^-\delta 
    & \ext^{i+1}(Z^\bullet,V^\bullet) \ar[r] 
    & \cdots
}\]
\end{proposition}

Soient $X^\bullet$ et $Y^\bullet$ deux objets de $\eK(\cA)$. On d\'esigne par 
$\mathsf{Qis}/X^\bullet$ (resp. $\mathsf{Qis}^+/X^\bullet$, 
$\mathsf{Qis}^-/X^\bullet$, $\mathsf{Qis}^b/X^\bullet$) la cat\'egorie des 
quasi-isomorphismes de but $X^\bullet$ et de source dans $\eK(\cA)$ (resp. 
$\eK^+(\cA)$, $\eK^-(\cA)$, $\eK^b(\cA)$). De m\^eme on d\'efinit les 
cat\'egories $Y^\bullet/\mathsf{Qis}$, $Y^\bullet/\mathsf{Qis}^+$,\ldots 
(quasi-isomorphismes de source $Y^\bullet)$. 

On se propose de r\'esumer dans la proposition suivante les diff\'erentes 
d\'efinitions \'equivalentes des $\ext^i$. 





\begin{proposition}\label{VIII:3-2-3}
1. Il existe des isomorphismes de foncteurs: 
\[\xymatrix{
  \ext^0(X^\bullet,Y^\bullet) = \hom_{\eD(\cA)}(Q(X^\bullet),Q(Y^\bullet)) \ar[r]^-\sim 
    & \varinjlim_{(\mathsf{Qis}/X^\bullet)^\circ} \hom_{\eK(\cA)}(-,Y^\bullet) \ar[r]^-\sim 
    & \\
  \varinjlim_{Y^\bullet/\mathsf{Qis}} \hom_{\eK(\cA)}(X^\bullet,-) \ar[r]^-\sim 
    & \varinjlim_{(\mathsf{Qis}/X^\bullet)^\circ \times Y^\bullet/\mathsf{Qis}} \hom_{\eK(\cA)}(-,-) \text{.}
}\]
Si $X^\bullet$ est un objet de $\eK^-(\cA)$: 
\[
  \ext^0(X^\bullet,Y^\bullet) \iso \varinjlim_{(\mathsf{Qis}^-/X^\bullet)^\circ} \hom_{\eK(\cA)}(-,Y^\bullet) 
\]
Si $Y^\bullet$ est un objet de $\eK^+(\cA)$: 
\[
  \ext^0(X^\bullet,Y^\bullet) \iso \varinjlim_{Y^\bullet/\mathsf{Qis}} \hom_{\eK(\cA)}(X^\bullet,-) 
\]
Si $X^\bullet$ est un objet de $\eK^-(\cA)$ et si $Y^\bullet$ est un 
objet de $\eK^b(\cA)$: 
\[
  \ext^0(X^\bullet,Y^\bullet) \iso \varinjlim_{Y^\bullet/\mathsf{Qis}^b} \hom_{\eK(\cA)}(X^\bullet,-) 
\]
Si $X^\bullet$ est un objet de $\eK^b(\cA)$ et $Y^\bullet$ un objet de 
$\eK^+(\cA)$: 
\[
  \ext^0(X^\bullet,Y^\bullet) \iso \varinjlim_{(\mathsf{Qis}^b/X^\bullet)^\circ} \hom_{\eK(\cA)}(-,Y^\bullet) 
\]
Si la cat\'egorie $\cA$ poss\`ede suffisamment d'injectif et si 
$Y^\bullet$ est un objet de $\eK^+(\cA)$, $Y^\bullet$ admet une 
r\'esolution injective et une seule (\`a isomorphisme pr\`es dans 
$\eK^+(\cA)$): 
\[
  Y^\bullet \to I(Y^\bullet) 
\]
et on a: 
\[
  \ext^0(X^\bullet,Y^\bullet) \iso \hom_{\eK(\cA)}(X^\bullet,I(Y^\bullet)) \text{.}
\]
On a de m\^eme un \'enonc\'e analogue pour les projectifs. 
Si la cat\'egorie $\cA$ admet suffisamment d'injectifs et si $\cA$ est 
de \emph{dimension cohomologique finie}, tout objet $Y^\bullet$ de 
$\eK(\cA)$ admet une r\'esolution injective et une seule: 
\[
  Y^\bullet \to I(Y^\bullet)
\]
et on a: 
\[
  \ext^0(X^\bullet,Y^\bullet) \iso \hom_{\eK(\cA)}(X^\bullet,I(Y^\bullet)) 
\]
Enonc\'e analogue pour les projectifs. 
\end{proposition}

\paragraph{Remarque:}
En explicitant l'isomorphisme de la proposition \ref{VIII:3-2-3}(3), on 
retrouve la d\'efinition de Yoneda. 










\section{Les foncteurs d\'eriv\'es}\label{VIII:4}










\subsection{D\'efinition des foncteurs d\'eriv\'es}\label{VIII:4-1}





\begin{definition}\label{VIII:4-1-1}
Soient $\cC$ et $\cC'$ deux cat\'egories gradu\'ees (on d\'esigne par $T$ le 
foncteur de translation de $\cC$ et de $\cC'$), $F$ et $G$ deux foncteurs 
gradu\'es de $\cC$ dans $\cC'$. Un \emph{morphisme de foncteurs gradu\'es} et 
un morphisme de foncteurs: 
\[
  u:F\to G 
\]
qui poss\`ede la propri\'et\'e suivante: 

Pour tout objet $X$ de $\cC$ le diagramme suivant est commutatif: 
\[\xymatrix{
  F(T X) \ar[r]^-{u(T X)} 
    & G(T X) \\
  T F(X) \ar[r]^-{T u(X)} \ar[u]_-\wr 
    & T G(X) \ar[u]_-\wr 
}\]
\end{definition}

Soient $\cC$ et $\cC'$ deux cat\'egories triangul\'ees. On d\'esigne par 
$\mathsf{Fex}(\cC,\cC')$ la cat\'egorie des foncteurs exacts de $\cC$ dans 
$\cC'$, le morphismes entre deux foncteurs \'etant les morphismes de foncteurs 
gradu\'es. 

Soient $\cA$ et $\cB$ deux cat\'egories ab\'eliennes de 
$\Phi:\eK^\ast(\cA)\to \eK^{\ast'}(\cB)$ un foncteur exact ($\ast$ et $\ast'$ 
d\'esignent l'un des signes $+$, $-$, $b$, ou $v$ ``vide''). Le foncteur 
canonique $Q:\eK^\ast(\cA)\to \eD^\ast(\cA)$ nous donne, par composition, un 
foncteur 
\[\xymatrix{
  \mathsf{Fex}(\eD^\ast(\cA),\eD^{\ast'}(\cB)) \ar[r] 
    & \mathsf{Fex}(\eK^\ast(\cA),\eD^{\ast'}(\cB)) 
}\]
d'o\`u (en d\'esignant aussi par $Q'$ le foncteur canonique 
$\eK^{\ast'}(\cB) \to \eD^{\ast'}(\cB)$) un foncteur: 
$\%$ (resp. $\%'$): 
$\mathsf{Fex}(\eD^\ast(\cA),\eD^{\ast'}(\cB)) \to \mathsf{Ab}$: 
\begin{align*}
  \Psi  &\mapsto \hom(Q'\circ \Phi,\Psi\circ Q) \\
  (\text{resp. }\Psi &\mapsto \hom(\Psi\circ Q,Q'\circ \Phi)\text{ ).} 
\end{align*}





\begin{definition}\label{VIII:4-1-2}
On dira que $\Phi$ admet un \emph{functeur d\'eriv\'e total} \`a droite (resp. 
\`a gauche) si le foncteur $\%$ (resp. $\%'$) est repr\'esentable. Un objet 
repr\'esentant le foncteur $\%$ (resp. $\%'$) ser appel\'e foncteur d\'eriv\'e 
total \`a droite (resp. \`a gauche) de $\Phi$ et sera not\'e 
$\eR \Phi$ (resp. $\eL \Phi$).\footnote{Une d\'efinition plus maniable (et en 
practique \'equivalente) est donn\'ee dans \cite[XVII 1.2]{sga4}.}
\end{definition}





\subsubsection{Notations}\label{VIII:4-1-3}

Soient $\cA$ et $\cB$ deux cat\'egories ab\'eliennes et $f:\cA\to \cB$ un 
foncteur additif. Le \emph{foncteur d\'eriv\'e total \`a droite} du foncteur 
$\eK^+(f):\eK^+(\cA) \to \eK^+(\cB)$ sera not\'e $\eR^+ f$. On d\'efinit de 
m\^eme $\eR^- f$, $\eR f$, $\eL^+ f$, $\eL^- f$, $\eL f$. 

Soit de m\^eme $F:\eK^\ast(\cA) \to \cB$, un foncteur cohomologique. Le 
foncteur canonique de passage au quotient $Q:\eK^\ast(\cA) \to \eD^\ast(\cA)$ 
d\'efinit par composition un foncteur: 
\[\xymatrix{
  \mathsf{Foco}(\eD^\ast(\cA),\cB) \ar[r] 
    & \mathsf{Foco}(\eK^\ast(\cA),\cB) 
}\]
($\mathsf{Foco}(-,-)$ d\'esigne la cat\'egorie des foncteurs cohomologiques) 
d'o\`u un foncteur $\%$ (resp. $\%'$) 
$\mathsf{Foco}(\eD^\ast(\cA),\cB) \to \mathsf{Ab}$, $G\mapsto \hom(F,G\circ Q)$ 
(resp. $G\mapsto \hom(G\circ Q,F)$). 





\begin{definition}\label{VIII:4-1-4}
On dira que le foncteur $F$ admet un \emph{foncteur d\'eriv\'e cohomologique} 
\`a droite (resp. \`a gauche) si le foncteur $\%$ (resp. $\%'$) est 
repr\'esentable. Un objet repr\'esentant le foncteur $\%$ (resp. $\%'$) sera 
appel\'e foncteur d\'eriv\'e cohomologique \`a droite (resp. \`a gauche) et 
sera not\'e $\eR F$ (resp. $\eL F$). 
\end{definition}





\subsubsection{Notations}\label{VIII:4-1-5}

Soient $\cA$ et $\cB$ deux cat\'egories ab\'eliennes et $f:\cA\to \cB$ un 
foncteur additif. Le foncteur d\'eriv\'e cohomologique \`a droite du foncteur: 
\[
  \h^0{}\circ \eK^+ f:\eK^+(\cA) \to \cB 
\]
sera not\'e $\eR^+ f$. Lorsqu'aucune confusion n'en r\'esultera le foncteur 
$\eR^+ f\circ T^i$ ($T$ est le foncteur de translation) sera not\'e: 
$\eR^i f$. Le foncteur d\'eriv\'e cohomologique \`a droite du foncteur 
$\h{}\circ \eK f$ sera not\'e $\eR f$. Lorsqu'aucune confusion n'en r\'esultera 
le foncteur $\eR f\circ T^i$ sera not\'e $\eR^i f$. 

On d\'efinit de m\^eme les foncteurs $\eL^- f$, $\eL f$, $\eL^i f$. 





\subsubsection{Remarques}\label{VIII:4-1-6}

Ces d\'efinitions contiennent, ainsi que nous le verrons dans le num\'ero 
suivant, la d\'efinition classique des foncteurs d\'eriv\'es lorsque les 
cat\'egories ont suffisamment d'injectifs ou de projectifs et que les complexes 
sont convenablement limit\'es. Elles sont cependant plus g\'en\'erales. Mais 
sans autres hypoth\`eses elles sont peu maniables. Par exemple, soit 
$f:\cA\to \cB$ un foncteur additif entre deux cat\'egories ab\'eliennes. 
Supposons que les foncteurs $\eR^+ f$ et $\eR f$ existent. Je ne sais pas 
d\'montrer que $\eR f$ induit sur $\eD^+(\cA)$ le foncteur $\eR^+ f$. Supposons 
de plus que le foncteur $\eR^+ f$ (d\'eriv\'e cohomologique) existe. Je ne sais 
pas d\'emontrer que le foncteur $\h^0{}\circ \eR^+ f$ est isomorphe au foncteur 
$\eR^+ f$. Cependant dans la pratique ces propri\'et\'es seront v\'erif\'ees. 










\subsection{Existence des foncteurs d\'eriv\'es}\label{VIII:4-2}





\begin{proposition}\label{VIII:4-2-1}
Les hypoth\`eses sont celles de la d\'efinition \label{VIII:4-1-4}. Supposons 
en outre que $\cA$ soit un $\sU$-cat\'egorie, o\`u $\sU$ est un universe tel 
que l'ensemble $\dZ$ des entiers rationnel soit un \'el\'ement de $\sU$, telle 
que l'ensemble des objets de $\cA$ soit un \'el\'ement de $\sU$. Supposons 
de plus que la cat\'egorie $\cB$ soit la cat\'egorie des $\sU$-groupes 
ab\'eliens ou plus g\'en\'erelement la cat\'egorie des faisceaux de 
$\sU$-groupes ab\'eliens sur un $\sU$-site quelconque. (Un $\sU$-site est une 
$\sU$-cat\'egorie, dont l'ensemble des onjets est un \'el\'ement de $\sU$, 
munie d'une topologie.) Le foncteur $F$ admet un foncteur d\'eriv\'e 
cohomologique \`a droite. Ce foncteur se calcule de la mani\`ere suivante: 

Soient $X$ un objet de $\eD^\ast(\cA)$, $\underline X$ l'objet de 
$\eK^\ast(\cA)$ au-dessus de $X$. On a alors un isomorphisme fonctoriel: 
\[
  \eR F(X) = \varinjlim_{\underline X/\mathsf{Qis}^\ast} F(-) \text{.}
\]
\end{proposition}

En particulier soit $f:\cA\to \cB$ un foncteur additif. Les foncteurs 
$\eR f$ et $\eR^+ f$ existent et se calculent de la mani\`ere indiqu\'ee 
ci-dessus. Le foncteur $\eR f$ induit sur $\eD^+(\cA)$ le foncteur 
$\eR^+ f$. Comme aucune confusion ne peut alors en r\'esulter, nous 
emploierons la notation $\eR^i f$. 





\subsubsection{Remarques}\label{VIII:4-2-2}

On peut exprimer la proposition \ref{VIII:4-2-1} sous une forme plus 
g\'en\'erale dans le cadre des cat\'egories triangul\'ees. Les hypoth\`eses 
\`a faire sur la cat\'egorie $\cB$, pour assurer la validit\'e de la 
proposition \ref{VIII:4-2-1} sont des hypoth\`eses d'existence et d'exactitude 
des limites inductives suivant les $\sU$-cat\'egories pseudo-filtrantes, dont 
les ensembles d'objet sont \'el\'ements de $\sU$. La proposition 
\ref{VIII:4-2-1} introduit une dissym\'etrie entre les foncteurs d\'eriv\'es 
droits et les foncteurs d\'eriv\'es gauches tout au moins dans la pratique. 





\begin{theorem}\label{VIII:4-2-3} % labelled 4.2.2 in original
Les donn\'ees sont celles de la d\'efinition \ref{VIII:4-1-2}. On se donne de 
plus une sous-cat\'egorie triangul\'ee pleine de $\eK^\ast(\cA)$: $\cC$, et on 
d\'signe par $\cN$ la sous-cat\'egorie triangul\'ee pleine des objets de $\cC$ 
qui sont acycliques. Supposons que 
\begin{enumerate}[(1)]
  \item Tout objet de $\cN$ soit transform\'e par $\Phi$ en objet acyclique 
    de $\eK^{\ast'}(\cB)$. 
  \item Tout objet $X$ de $\eK^\ast(\cA)$ soit source (resp. but) d'un 
    quasi-isomorphisme dont le but (resp. la source) soit un objet de $\cC$. 
\end{enumerate}
\begin{enumerate}[a)]
  \item Alors $\Phi$ admet un foncteur d\'eriv\'e total \`a droite (resp. 
    \`a gauche). 
  \item Le fonteur $\eR \Phi$ (resp. $\eL \Phi$) s'obtient de la mani\`ere 
    suivante: 
    
    La restriction du foncteur 
    $Q'\circ \Phi:\eK^\ast(\cA) \to \eD^{\ast'}(\cB)$ \`a la cat\'egorie $\cC$ 
    s'annule sur les objets de $\cN$, donc se factorise par $\cC/\cN$. On 
    obtient ainsi un foncteur $\Phi':\cC/\cN\to \eD^{\ast'}(\cB)$. Mais le 
    foncteur naturel: $\cC/\cN\to \eD^\ast(\cA)$ est une \'equivalence de 
    cat\'egories (th\'or\`me \ref{VIII:2-4-2}). D'o\`u en composant avec le 
    foncteur quasi-inverse le foncteur $\eR\Phi$ (resp. $\eL \Phi$). 
  \item Soit $X$ un objet de $\eK^\ast(\cA)$. Soient $Y$ un objet de $\cC$ et 
    $X\to Y$ (resp. $Y\to X$) un quasi-isomorphisme. L'objet 
    $\eR\Phi\circ Q(X)$ (resp. $\eL \Phi\circ Q(X)$) est isomorphe \`a 
    $Q\circ \Phi(Y)$. 
  \item Le foncteur $X\mapsto \eR\Phi\circ Q(X)$ (resp. 
    $X\mapsto \eL \Phi\circ Q(X)$) est isomorphe au foncteur: 
    $X\mapsto \varinjlim_{X/\mathsf{Qis}^\ast} Q\circ\Phi(-)$ (resp. 
    $X\mapsto \varinjlim_{\mathsf{Qis}^\ast/X} Q\circ \Phi(-)$). 
  \item Le foncteur $\h^0{}\circ \Phi:\eK^\ast(\cA)\to \cB$, admet un 
    foncteur d\'eriv\'e cohomologique \`a droite (resp. \`a gauche) qui n'est 
    autre que le foncteur $\h^0{}\circ\eR\Phi$ (resp. $\h^0{}\circ \eL\Phi$). 
\end{enumerate}
\end{theorem}





\begin{corollary}\label{VIII:4-2-3-1} % originally corollary 1
Supposons que $\ast=+$ (resp. $\ast=-$) dans les hypoth\`eses du th\'eor\`eme 
pr\'ec\'edent, et que $\cA$ poss\`ede suffisamment d'injectifs (resp. de 
projectifs). Alors le th\'eor\`eme \ref{VIII:4-2-3} s'applique en prenant pour 
$\cC$ la cat\'egorie des complexes injectifs (resp. projectifs). Supposons de 
plus que $\cA$ soit de dimension cohomologique finie, alors le th\'eor\`eme 
s'applique, sans restriction sur $\ast$, en prenant sur $\cC$ la cat\'egorie 
des complexes injectifs (resp. projectifs). 
\end{corollary}





\begin{definition}\label{VIII:4-2-5} % originally 2.4
Soient $\cA$ et $\cB$ deux cat\'egories ab\'eliennes et $f:\cA\to \cB$ un 
foncteur additif. 

On dira que $f$ est de \emph{dimension cohomologique finie} \`a droite (resp. 
\`a gauche) si $\eR^+ f$ (resp. $\eL^- f$) existe et s'il existe un entier 
$m\geqslant 0$ tel que pour tout entier $n>m$, $\eR^n f$ (resp. 
$\eL^{-n} f$) soit nul sur les objets provenant de $\cA$ par le foncteur $D$. 
La \emph{dimension cohomologique} \`a droite (resp. \`a gauche) de $f$ est 
alors le plus petit des entiers $m$ poss\'edant la propri\'et\'e ci-dessus. Si 
$\eR^+ f$ (resp. $\eL^- f$) existe et si $f$ n'est pas de dimension 
cohomologique finie, on dira que $f$ est de dimension cohomologique infinie. 

Un objet $X$ de $\cA$ sera dit \emph{$f$-acyclique} \`a droite (resp. \`a 
gauche) si pour tout entier $n>0$, $\eR^n f(D(X)) = 0$ (resp. 
$\eL^{-n} f(D(X))=0$). 
\end{definition}





\begin{corollary}\label{VIII:4-2-3-2} % originally corollary 2
Les donn\'ees sont celles de la d\'efinition \ref{VIII:4-2-5}. Supposons qu'il 
existe un ensemble $\sigma$ d'objets de $\cA$ stable par somme directe 
poss\'edant les propri\'et\'es suivantes: 
\begin{enumerate}
  \item Tout complexe acyclique $\in \operatorname{Ob}{\eK^+(\cA)}$ (resp. 
    $\in\operatorname{Ob}{\eK^-(\cA)}$) d'objets de $\sigma$, est transform\'e 
    par $f$ en complexe acyclique. 
  \item Tout objet $X$ de $\cA$ s'injecte dans (resp. est quotient de) un objet 
    de $\sigma$. Les hypoth\`eses du th\'eor\`eme \ref{VIII:2-4-3} sont 
    v\'erifi\'ees en prenant $\ast=+$ (resp. $\ast=-$) et en prenant pour 
    cat\'egorie $\cC$, la cat\'egorie des complexes dont les objets et tout 
    degr\'e sont des \'el\'ements de $\sigma$. Les \'el\'ements de $\sigma$ 
    sont des objets $f$-acycliques \`a droite (resp. \`a gauche). 
\end{enumerate}

Si de plus $f$ est de dimension cohomologique finie \`a droite (resp. \`a 
gauche), l'ensemble des objets $f$-acycliques \`a droite (resp. \`a gauche) 
[pss\`ede en plus des propri\'et\'es (1) et (2) ci-dessus, la propri\'et\'e 
(2) de proposition \ref{VIII:3-1-9}. Le Th\'eor\`eme \ref{VIII:4-2-3} 
s'applique alors au foncteur $\eK f$ en prenant pour cat\'egorie $\cC$ la 
cat\'egorie des complexes dont les objets sont $f$-acycliques \`a droite 
(resp. \`a gauche) en tout degr\'e. Le foncteur $\eR f$ (resp. $\eL f$) existe 
et induit sur $\eD^+(\cA)$ le foncteur $\eR^+ f$ (resp. $\eL^+ f$) et sur 
$\eD^-(\cA)$ le foncteur $\eR^- f$ (resp. $\eL^- f$). 
\end{corollary}










\subsection{Produit. Composition}\label{VIII:4-3}

Soient $\cA$ une cat\'egorie ab\'elienne, $F:\eD(\cA) \to \mathsf{Ab}$ un 
foncteur cohomologique. Soient $X^\bullet$ et $Y^\bullet$ deux objets de 
$\eC(\cA)$. Le foncteur $F$ d\'efinit une application: 
\[
  \ext^0(X^\bullet,Y^\bullet) \to \hom_{\mathsf{Ab}}(F(D(X^\bullet)),F(D(Y^\bullet)))
\]
d'o\`u un accouplement: 
$F(D(X^\bullet))\times \ext^0(X^\bullet,Y^\bullet) \to F(D(Y^\bullet))$. 

Cet accouplement donne, en appliquant des translations, des accouplements: 
\[
  F^i(D(X^\bullet))\times \ext^j(X^\bullet,Y^\bullet) \to F^{i+j}(D(Y^\bullet)) \text{.}
\]
Ces accouplements, appliqu\'es aux foncteurs d\'eriv\'es cohomologiques, ne 
sont autres, par d\'efinition, que les produits de Yoneda. Les propri\'et\'es 
de ces produits se d\'eduisent imm\'ediatement de cette d\'efinition. Nous ne 
les d\'evelopperons pas ici. 

Soient $\cA$, $\cA'$, $\cA''$ trois cat\'egories ab\'eliennes et 
$f:\cA\to \cA'$, $g:\cA'\to \cA''$ deux foncteurs additifs. Supposons que les 
foncteurs $\eR^+ f$ et $\eR^+ g$ existent. Je ne sais pas en d\'eduire que le 
foncteur $\eR^+ g\circ f$ existe et m\^eme si ce dernier foncteur existe, il 
n'est probablement pas vraie en g\'en\'eral que l'on ait la formule: 
\[\xymatrix{
  \eR^+ g\circ f\ar[r]^-\sim 
    & \eR^+ g \circ \eR^+ f 
}\]
On a cependant la proposition suivante: 





\begin{proposition}\label{VIII:4-3-1}
Supposons que la cat\'egorie $\cA'$ poss\`ede suffisamment d'objets 
$g$-acycliques \`a droite et que la cat\'egorie $\cA$ poss\`ede suffisamment 
d'objets $f$-acycliques \`a droite transform\'es par $f$ en objets 
$g$-acycliques \`a droite. Alors le foncteur $\eR^+ g\circ f$ existe et on a un 
isomorphisme: 
\[\xymatrix{
  \eR^+ g\circ f \ar[r]^-\sim 
    & \eR^+ g \circ \eR^+ f \text{.}
}\]
\end{proposition}

Enonc\'e analogues pour les foncteurs d\'eriv\'es gauches. 















\section{Exemples}\label{VIII:6} % originally chapter 2, section 3










\subsection{Le foncteur \texorpdfstring{$\mathsf{Rhom}$}{Hom*}}\label{VIII:6-1}

Soit $\cA$ une cat\'egorie ab\'elienne, poss\'edant suffisamment d'injectifs. 

Le foncteur: $Y\mapsto \hom^\bullet(X^\bullet,Y^\bullet)$ (\ref{VIII:1-3}) 
o\`u $X^\bullet$ est un objet de $\eK(\cA)$ et $Y^\bullet$ un objet de 
$\eK^+(\cA)$, admet un foncteur d\'eriv\'e total \`a droite 
(Corollaire \ref{VIII:4-2-3-1}) qui sera not\'e $\rhom^\bullet(X^\bullet,-)$.  
Soit $Y$ un objet de $\eD^+(\cA)$ le foncteur 
\begin{align*}
  \eK(\cA) &\to \eD(\mathsf{Ab}) \\
  X^\bullet &\mapsto \rhom^\bullet(X^\bullet,Y) 
\end{align*}
est exact et s'annule sur les complexes acycliques, d'o\`u un bi-foncteur 
exact 
\[\xymatrix{
  \eD(\cA)^\circ \times \eD^+(\cA) \ar[r] 
    & \eD(\mathsf{Ab}) 
}\]
que nous d\'esignerons par $\mathsf{Rhom}^\bullet$. 

Lorsque $\cA$ est de dimension cohomologique finie, le foncteur 
$\mathsf{Rhom}^\bullet$ se prolonge \`a $\eD(\cA)^\circ\times \eD(\cA)$ 
(loc. cit.). 

Si en outre $\cA$ poss\`ede suffisamment de projectifs on peut d\'eriver le 
foncteur $\hom$ par rapport au premier argument. Les deux d\'efinitions 
co\"incident dans leur domaine commun de validit\'e. 










\subsection{\texorpdfstring{$\Tor$}{Tor} de faisceaux}\label{VIII:6-2}

Soient $E$ un site annel\'e (en particulier un espace topologique annel\'e) 
$\sA$ le faisceau d'anneaux et $\underline E$ la cat\'egorie des 
$\sA$-modules. Un objet $X$ de $\underline E$ est dit 
\emph{$\sA$-plat} si pout toute suite exacte: 
\[\xymatrix{
  0 \ar[r] 
    & Y' \ar[r] 
    & Y \ar[r] 
    & Y'' \ar[r] 
    & 0 
}\]
la suite: $0 \to Y'\otimes_\sA X \to Y\otimes_\sA X \to Y''\otimes_\sA X\to 0$ 
est exacte. Il existe suffisamment d'objets $\sA$-plats: les sommes directes 
d'objets $\sA_{U'}$, o\`u $U$ est un objet de $E$ et $\sA_U$ le faisceau 
$\sA$ ``prolongu\'e par z\'ero en dehors de $U$.'' Le th\'eor\`eme 
\ref{VIII:4-2-3} s'applique donc et on peut d\'efinir le foncteur 
\begin{align*}
  \eD^-(\underline E) \times \eD^-(\underline E) &\to \eD^-(\underline E) \\
  (X,Y) &\mapsto X\lotimes Y 
\end{align*}
produit tensoriel total, d\'eriv\'e gauche du foncteur produit tensoriel. 
On d\'efinit ainsi en passant \`a la cohomologie les $\Tor_i^\sA(X,Y)$: 
hyper-tor locaux. 

La cat\'egorie $\cE$ ayant suffisamment d'injectifs, on peut d\'eriver le 
foncteur: $\hom^\bullet(X,Y)$: complexe des homomorphismes locaux, 
d'o\`u $\mathsf{Rhom}(X,Y)$ ($X\in \eD$, $Y\in \eD^+$). 

Soient $X$ et $Y$ deux objets de $\eD^-(\underline E)$, $Z$ un objet de 
$\eD^+(\underline E)$. Il existe un isomorphisme fonctoriel: 
\[\xymatrix{
  \hom_{\eD(\underline E)}(X\lotimes Y,Z) \ar[r]^-\sim 
    &\hom_{\eD(\underline E)}(X,\mathsf{Rhom}(Y,Z)) \text{.}
}\]










\subsection{Foncteur d\'eriv\'e gauche de l'image r\'eciproque}\label{VIII:6-3}

Soit $f:E\to E'$ une application d'espaces topologiques annel\'es. Le 
foncteur image r\'eciproque: 
\[
  f^\ast:\underline E' \to \underline E 
\]
n'est pas n\'ecessairement exact. Mais on sait d\'efinir des objets 
$f^\ast$-plats et il existe suffisamment de tels objets. On sait donc 
d\'efinir le foncteur $\eL^- f^\ast$. On a de m\^eme une formule de dualit\'e: 

Soient $X$ un objet de $\eD^-(\underline E)$ et $Y$ un objet de 
$\eD^+(\underline E')$; il existe un isomorphisme fonctoriel: 
\[\xymatrix{
  \hom_{\eD(\underline E')}(\eL^- f^\ast(X),Y) \ar[r]^-\sim 
    & \hom_{\eD(\underline E)}(X,\eR^+ f_\ast(Y)) \text{.}
}\]
















