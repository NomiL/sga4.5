% !TEX root = sga4.5.tex

\chapter{Applications de la formule des traces aux sommes trigonométriques}\label{VI}










Dans cet expos\'e, j'explique comment la formule des traces permet de calculer 
ou d'\'etudier diverses sommes trigonom\'etriques et comment, jointe \`a la 
conjecture de Weil, elle peut permettre de les majorer.

Les deux premiers paragraphes donnent un ``mode d'emploi'' de ces outils. Le 
paragraphe 3 est un espos\'e, dans un langage cohomologique, des r\'esultats de 
Weil sur les sommes \`a $1$ variable. Les paragraphes $4$ \`a $6$ forment une 
\'etude d\'etaill\'ee des sommes de Gauss et de Jacobi -- y inclus les 
r\'esultats anciens et r\'ecents de Weil sur les caract\`eres de Hecke 
d\'efinis par des sommes de Jacobi. Au paragraphe $7$, nous \'etudions une 
g\'en\'eralisation \`a plusiers variables des sommes de Kloosterman. Enfin, au 
paragraphe $8$, on trouvera quelques indications sur d'autres usages qui ont 
\'et\'e faits on peuvent \^etre faits de ces m\'ethodes. 










\section*{Notations}\label{VI:0}





\subsection{}\label{VI:0-1}

On utilise les notations de \hyperref[II]{Rapport}, paragraphe \ref{II:1}. On 
aura souvent \`a consid\'erer une extension finie $\dF_{q^n}\subset \dF$ de 
$\dF_q$. On notera par un indice $0$ un objet sur $\dF_q$, et par un indice $1$ 
un objet sur $\dF_{q^n}$. Remplacer un indice $0$ par un indice $1$ (resp. 
supprimer l'incide) signifie qu'on \'etend les scalaires \`a $\dF_{q^n}$ (resp. 
\`a $\dF$).





\subsection{}\label{VI:0-2}

On d\'esigne par $\ell$ un nombre premier $\ne p$. Nous utilisons librement le 
language des $\dQ_\ell$-faisceaux, ainsii que celui des $E_\lambda$-faisceaux, 
pour $E_\lambda$ une extension finie de $\dQ_\ell$ (cf. \hyperref[II]{Rapport}, 
paragraphe \ref{II:2} et sp\'ecialement \ref{II:2-11}). 





\subsection{}\label{VI:0-3}

Soit $H^\bullet$ un espace vectoriel gradu\'e. Si $T$ est un endomorphisme de 
$H^\bullet$, on pose (cf. \hyperref[IV]{Cycle}, \ref{IV:eq:1-3-5}) 
\[
  \tr(T,H^\bullet) = \sum (-1)^i \tr(T,H^i) \text{.}
\]










\section{Principes}\label{VI:1}





\subsection{}\label{VI:1-1}

Soient $X_0$ un sch\'ema s\'epar\'e de type fini sur $\dF_q$, $E_\lambda$ une 
extension finie de $\dQ_\ell$ et $\sF_0$ un $E_\lambda$-faisceau sur $X-0$. La 
formule des traces dit que 
\begin{equation*}\tag{1.1.1}\label{VI:eq:1-1-1}
  \sum_{x\in X^F} \tr(F_x^\ast,\sF) = \sum (-1)^i \tr\left(F^\ast,\h_c^i(X,\sF)\right) \text{.}
\end{equation*}

Avec la notation \ref{VI:0-3}, le membre de droite s'\'ecrit simplement 
$\tr(F^\ast,\h_c^\bullet(X,\sF))$. 

Cette formule des traces pour les $E_\lambda$-faisceaux peut soit \^etre 
d\'eduite de \hyperref[II]{Rapport} \ref{II:4-10} par passage \`a limite (cf. 
\hyperref[II]{Rapport} \ref{II:4-11} \`a \ref{II:4-13}), soit \^etre d\'eduite 
de la formule des traces pour les $\dQ_\ell$-faisceaux (\hyperref[II]{Rapport} 
\ref{II:3-2}) par la m\'ethode de (\hyperref[III]{Functions $L$ mod. $\ell^n$}, 
\ref{III:4-3}). 

Nous allons interpr\'eter diverses sommes trigonom\'etriques comme le membre de 
gauche de \eqref{VI:eq:1-1-1}, pour $X_0$ et $\sF_0$ convenables. 





