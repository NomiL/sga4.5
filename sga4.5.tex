\documentclass{article}


\title{Cohomologie Etale}
\author{Pierre Deligne}
\date{1977}

\usepackage[utf8]{inputenc}
\usepackage[all]{xy}
\usepackage{amsmath,amssymb,amsthm,enumerate,fullpage} %,titlesec} to have inline subsubsections
\usepackage[colorlinks,backref]{hyperref}

\DeclareMathOperator{\ab}{Ab}
\DeclareMathOperator{\aut}{Aut}
\DeclareMathOperator{\br}{Br}
\DeclareMathOperator{\dv}{div}
\DeclareMathOperator{\Div}{Div}
\DeclareMathOperator{\gal}{Gal}
\DeclareMathOperator{\gl}{GL}
\DeclareMathOperator{\nrd}{Nrd}
\DeclareMathOperator{\ob}{Ob}
\DeclareMathOperator{\pgl}{PGL}
\DeclareMathOperator{\pic}{Pic}
\DeclareMathOperator{\sh}{Sh}
\DeclareMathOperator{\spec}{Spec}

\newcommand{\const}[1]{\underline{#1}}
\newcommand{\et}[1]{{#1}_{\textnormal{et}}}
\newcommand{\cC}{\mathcal{C}}
\newcommand{\cF}{\mathcal{F}}
\newcommand{\cG}{\mathcal{G}}
\newcommand{\cH}{\mathcal{H}}
\newcommand{\cL}{\mathcal{L}}
\newcommand{\cO}{\mathcal{O}}
\newcommand{\cS}{\mathcal{S}}
\newcommand{\cU}{\mathcal{U}}
\newcommand{\dC}{\mathbb{C}}
\newcommand{\dG}{\mathbb{G}}
\newcommand{\dmu}{\boldsymbol{\mu}}
\newcommand{\dN}{\mathbb{N}}
\newcommand{\dP}{\mathbb{P}}
\newcommand{\dR}{\mathbb{R}}
\newcommand{\dZ}{\mathbb{Z}}
\newcommand{\fa}{\mathfrak{a}}
\newcommand{\fr}{\mathfrak{r}}
\newcommand{\iso}{\xrightarrow\sim}

\newtheorem{proposition}[subsubsection]{Proposition}
\newtheorem{corollary}[subsubsection]{Corollaire}
\newtheorem{definition}[subsubsection]{Définition}
\newtheorem{theorem}[subsubsection]{Théorème}
\newtheorem{lemma}[subsubsection]{Lemme}
\setcounter{tocdepth}{2}
% \titleformat{\subsubsection}[runin]{\normalfont\large\bfseries}{\thesubsubsection}{1em}{}



\begin{document}
\maketitle
\tableofcontents

% French characters
% \`a -> à
% \'e -> é
% \`e -> è
% \^e -> ê
% \^o -> ô
% \`u -> ù
% \^u -> û

% other notes: there are two I.6.5's in Deligne's original - one defining sheaf 
% cohomology, the other defining torsors    




















\section*{Introduction}

Ce volume a pour but de faciliter au non-expert l'usage de la 
cohomologie $\ell$-adique. J'espère qu'il lui permettra souvent d'éviter le 
recours aux exposés touffus de SGA 4 et SGA 5. Il contient aussi quelques 
résultats nouveaux. 

Le premier exposé, édigé par J.F. Boutot, survole SGA 4. Il donne les 
principaux résultats -- avec une généralité minimale, souvent 
insuffisante pour les applications -- et une idée de leur démonstration. Pour 
des résultats complets, ou des démonstrations détaillées, SGA 4 reste 
indispensable. 

Le ``Rapport sur la formule des traces'' contient une démonstration complété de 
la formule des traces pour l'endomorphisme de Frobenius. La démonstration est 
celle donnée par Grothendieck dans SGA 5, élaguée de tout détail inutile. Ce 
Rapport devrait permettre à utilisateur d'oublier SGA 5, qu'on pourra 
considérer comme une série de digression, certaines très intéressantes. Son 
existence permettra de publier prochainement SGA 5 tel quel. Il est complété 
par l'exposé ``Applications de la formule des traces aux sommes 
trigonométriques'' qui explique comment la formule des traces permet l'étude de 
sommes trigonométriques, et donne des exemples. 

Le public visé par les autres exposés est plus limité, et leur style s'en 
ressent. L'exposé ``Fonctions $L$ modulo $\ell^n$ et modulo $p$'' est une 
généralisation ``modulaire'' du Rappoport, basée sur l'étude SGA 4 XVII 5.5 des 
puissances symétriques. L'exposé ``La classe de cohomologie associée à un 
cycle'' définit cette classes dans divers contextes, et donne la compatibilité 
entre intersections et cup-produits. Dans ``Dualité'' sont rassemblés quelques 
résultats connus, pour lesquels manquait une référence, et quelques 
compatibilités. L'exposé ``Théorèmes de finitude en cohomologie $\ell$-adique'' 
est nouveau. Il donne notamment, en cohomologie sans supports, des théorèmes de 
finitude analogues à ceux connus en cohomologie à supports compacts. 

Pour plus de détails sur les exposés, je renvoie à leur introduction 
respective. 

Je remercie enfin J.L. Verdier de m'avoir permis de reproduire ici ses notes 
``Catégories dérivées (Etat $0$).'' Elles restent je crois très utiles, et 
étaien: devenues introuvables. 

Dans les références internes à ce volume, les exposés sont cités par 
un titre abrégé, indiqué entre [ ] dans la table des matières. 

Bures-sur-Yvette, le 20 Septembre 1976

Pierre Deligne




















\section{Topologies de Grothendieck}\label{1}

A l'origine, les topologies de Grothendieck sont apparues comme sous-jacentes 
à sa théorie de la descente (cf. SGA 1 VI, VIII); l'usage des théorie de 
cohomologie correspondantes est plus tardif. La même démarche est suivi 
ici: en formalisant les notions classiques de localisation, de propriété 
locale et de recollement (\S\ref{1-1}, \ref{1-2}, \ref{1-3}), en dégage le 
concept général de topologie de Grothendieck (\S \ref{1-6}); pour en justifier 
l'introduction en géométrie algébrique, on démontre un théorème 
de descente fidèlement plat (\S\ref{1-4}), généralisation du classique théorème 
de Hilbert (\S\ref{1-5}). 

Le lecteur trouvera une exposition plus compète, mais concise, du 
formalisme dans Giraud \cite{Gi}. Les notes de M. Artin: ``Grothendieck 
topologies'' \cite{Ar} (chapitres I à III) restent également utiles. Les 
866 pages des exposés I à V de SGA 4 sont précieuses lorsqu'on 
considère des topologies exotiques, telle celle qui donne naissance à la 
cohomologie cristalline; pour utiliser la topologie étale si proche de 
l'intuition classique, il n'est pas indispensable de les lire. 










\subsection{Cribles}\label{1-1}

Soient $X$ un espace topologique et $f:X\to \dR$ une fonction à valeur 
réelles sur $X$. La continuité de $f$ est une propriété de nature 
locale; autrement dit, si $f$ est continue sur tout ouvert suffisamment petit 
de $X$, $f$ est continue sur $X$ tout entier. Pour formaliser la notion de 
``propriété de nature locale,'' nous introduirons quelques définitions.

On dit qu'un ensemble $\cU$ d'ouverts de $X$ est un \emph{crible} si pour tout 
$U\in\cU$ et $V\subset U$, on a $V\in\cU$. On dit qu'un crible est 
\emph{couvrant} si la réunion de tout les ouverts appartenant à ce crible est 
égale à $X$.

Etant donnée une famille $\{U_i\}$ d'ouverts de $X$, le crible engendré par 
$\{U_i\}$ est par définition l'ensemble des ouverts $U$ de $X$ tels que $U$ 
soit contenu dans l'un des $U_i$. 

On dit qu'une propriété $P(U)$, définie pour tout ouvert $U$ de $X$, est 
\emph{locale} si, pour tout crible couvrant $\cU$ de tout ouvert $U$ de $X$, 
$P(U)$ est vraie si et seulement si $P(V)$ est vraie pour tout $V\in \cU$. Par 
exemple, étant donné $f:X\to \dR$, la propriété ``$f$ est continue sur $U$'' 
est locale. 





\subsection{Faisceaux}\label{1-2}

Précisons la notion de fonction donnée localement sur $X$.





\subsubsection{Point de vue des cribles}\label{1-2-1}

Soit $\cU$ un crible d'ouverts de $X$. On appelle fonction donnée 
$\cU$-localement sur $X$ la donnée pour tout $U\in \cU$ d'une fonction $f_U$ 
sur $U$ telle que, si $V\subset U$, on ait $f_V=f_U|V$. 





\subsubsection{Pointe de vue de Čech}\label{1-2-2}

Si le crible $\cU$ est engendré par une famille d'ouverts $U_i$ de $X$, se 
donner une fonction $\cU$-localement revient à se donner une fonction $f_i$ 
sur chaque $U_i$, telle que $f_i|{U_i\cap U_j} = f_j|{U_i\cap U_j}$. 

Autrement dit, si $Z=\coprod U_i$, se donner une fonction $\cU$-localement 
revient à se donner une fonction sur $Z$ qui soit constante sur les fibres de 
la projection naturelle $Z\to X$. 





\subsubsection{}\label{1-2-3}

Les fonctions continues forment un faisceau; cela signifie que pour tout crible 
couvrant $\cU$ d'un ouvert $V$ de $X$ et toute fonction donnée 
$\cU$-localement $\{f_U\}$ telle que chaque $f_U$ soit continue sur $U$, il 
existe une unique fonction continue $f$ sur $V$ telle que $f|U=f_U$ pour tout 
$U\in \cU$.










\subsection{Champs}\label{1-3}

Précisons maintenant la notion de fibré vectoriel donné localement sur $X$. 





\subsubsection{Point de vue des cribles}\label{1-3-1}

Soit $\cU$ un crible d'ouverts de $X$. On appelle fibré vectoriel donné 
$\cU$-localement sur $X$ les données de 
\begin{enumerate}[\indent a)]
  \item un fibré vectoriel $E_U$ sur chaque $U\in \cU$, 
  \item si $V\subset U$, un isomorphisme $\rho_{U,V} : E_V\iso E_U|V$, vérifiant 
  \item si $W\subset V\subset U$, le diagramme 
    \[\xymatrix{
      E_W \ar[r]^-{\rho_{U,W}} \ar[dr]_-{\rho_{V,W}}
        & E_U|W \\
      & E_V|W \ar[u]_-{\rho_{U,V}|W}
    }\]
    commute, c'est-à-dire 
    $\rho_{U,V} = (\mbox{$\rho_{U,V}$ restreint à $W$}) \circ \rho_{V,W}$. 
\end{enumerate}





\subsubsection{Point de vue de Čech}\label{1-3-2}

Si le crible $\cU$ est engendré par une famille d'ouverts $U_i$ de $X$, se 
donner un fibré vectoriel $\cU$-localement revient à se donner:
\begin{enumerate}[\indent a)]
  \item un fibré vectoriel $E_i$ sur chaque $U_i$, 
  \item si $U_{ij} = U_i\cap U_j = U_i\times_X U_j$, un isomorphisme 
    $\rho_{ji} : E_i|U_{ij} \iso E_j|U_{ij}$, de sorte que 
  \item si $U_{ijk} = U_i\times_X U_j\times_X U_k$, le diagramme 
    \[\xymatrix{
      E_i |U_{ijk} \ar[r]^-{\rho_{ki}|U_{ijk}} \ar[dr]_-{\rho_{ji}|U_{ijk}} 
        & E_k|U_{ijk} \\
      & E_j|U_{ijk} \ar[u]_-{\rho_{kj}|U_{ijk}}
    }\]
    commute, c'est-à-dire $\rho_{ki} = \rho_{kj}\circ \rho_{ji}$ sur 
    $U_{ijk}$. 
\end{enumerate}

Autrement dit, si $Z=\coprod U_i$ et si $\pi:Z\to X$ est la projection 
naturelle, se donner un fibré vectoriel $\cU$-localement revient à se donner: 
\begin{enumerate}[\indent a)]
  \item un fibré vectoriel $E$ sur $Z$, 
  \item si $x$ et $y$ sont deux points de $Z$ tels que $\pi(x) = \pi(y)$, un 
    isomorphisme $\rho_{y x}:E_x\iso E_y$ entre les fibres de $E$ en $x$ et en 
    $y$, dépendant continûment de $(x,y)$ et tel que, 
  \item si $x$, $y$ et $z$ sont trois points de $Z$ tels que 
  $\pi(x)=\pi(y)=\pi(z)$, on ait $\rho_{zx} = \rho_{zy}\circ \rho_{yx}$. 
\end{enumerate}





\subsubsection{}\label{1-3-3}

Une fibré vectoriel $E$ sur $X$ définit un fibré vectoriel donné 
$\cU$-localement $E_\cU$; le système des restrictions $E_U$ de $E$ aux objets 
de $\cU$. Le fait que la notion de fibré vectoriel est de nature locale peut 
s'exprimer ainsi: pour tout crible couvrant $\cU$ de $X$, le foncteur 
$E\mapsto E_\cU$, des fibrés vectoriels sur $X$ dans les fibrés vectoriels 
donnés $\cU$-localement, est une équivalence de catégories. 





\subsubsection{}\label{1-3-4}

Si dans \ref{1} on remplace ``ouvert de $X$'' par ``partie de 
$X$,'' on obtient la notion de crible de sous-espaces de $X$. Dans ce cadre 
aussi on dispose de théorèmes de recollement. Par exemple: soient $X$ un 
espace et $\cC$ un crible de sous-espaces de $X$ engendré par un recouvrement 
fermé localement fini de $X$, alors le foncteur $E\mapsto E_\cC$, des fibrés 
vectoriels sur $X$ dans les fibrés vectoriels donnés $\cC$-localement est une 
équivalence de catégories. 

En géométrie algébrique, il est utile de considérer aussi des ``cribles 
d'espaces au-dessus de $X$''; c'est ce que nous verrons au paragraphe suivant. 










\subsection{Descente fidèlement plat}\label{1-4}





\subsubsection{}\label{1-4-1}

Dans le cadre des schémas, la topologie de Zariski n'est pas assez fine pour 
l'étude des problémes non linéaires et on est amené à remplacer dans les 
définitions précédentes les immersions ouvertes par des morphismes plus 
généraux. De ce point de vue, les techniques de descente apparaissent come des 
techniques de localisation. Ainsi l'énoncé de descente suivant peut sexprimer 
en disant que les propriétés considérées sont de nature locale pour la 
topologie fidélement plate (on dit qu'un morphisme de schémas est fidélement 
plat s'il est plat est surjectif). 


\begin{proposition}\label{1-4-2}
Soient $A$ un anneau et $B$ une $A$-algébre fidélement plate. Alors:
\begin{enumerate}[(i)]
  \item Une suite $\Sigma=(M'\to M\to M'')$ de $A$-modules est exacte dés 
    que la suite $\Sigma_{(B)}$ qui s'en déduit par extension des scalaires 
    à $B$ est exacte.
  \item Un $A$-module $M$ est de type fini (resp. de présentation finie, 
    plat, localement libre de rang fini, inversible (i.e. localement libre 
    de rang un)) dés que le $B$-module $M_{(B)}$ l'est.
\end{enumerate}
\end{proposition}
\begin{proof}
(i) Le foncteur $M\mapsto M_{(B)}$ étant exact (platitude de $B$), il suffit 
de montrer que, si un $A$-module $N$ est non nul, $N_{(B)}$ est non nul. Si 
$N$ est non nul, $N$ contient un sous-module monogéne non nul $A/\fa$; alors 
$N_{(B)}$ contient un sous-module monogéne $(A/\fa)_{(B)} = B/\fa B$, non nul 
par surjectivité du morphisme structural $\varphi:\spec(B)\to\spec(A)$ (si 
$V(\fa)$ est non vide, $\varphi^{-1}(V(\fa))=V(\fa B)$ est non vide). 

(ii) Pour toute famille $(x_i)$ d'éléments de $M_{(B)}$, il existe un 
sous-module de type fini $M'$ de $M$ tel que $M'_{(B)}$ contienne les $x_i$. Si 
$M_{(B)}$ est de type fini et si les $x_i$ engendrent $M_{(B)}$, on a 
$M'_{(B)}=M_{(B)}$, donc $M'=M$ et $M$ est de type fini. 
\end{proof}

Si $M_{(B)}$ est de présentation finie, on peut, d'aprés ce qui précède, 
trouver une surjection $A^n\to M$. Si $N$ est le noyau de cette surjection, le 
$B$-module $N_{(B)}$ est de type fini, donc $N$ l'est, et $M$ est de 
présentation finie. L'assertion pour ``flat'' résulte aussitôt de (i); 
``localement libre de rang fini'' signifie ``plat et de présentation finie'' 
et le rang se teste par extension des scalaires à des corps.





\subsubsection{}\label{1-4-3}

Soient $X$ un schéma et $\cS$ une classe de $X$-schémas stable par produit 
fibré sur $X$. Une classe $\cU\subset \cS$ est un \emph{crible} sur $X$ 
(relativement à $\cS$) si, pour tout morphisme $\varphi:V\to U$ de 
$X$-schémas, avec $U,V\in \cS$ et $U\in\cU$, on a $V\in \cU$. Le crible 
\emph{engendré} par une famille $\{U_i\}$ de $X$-schémas dans $\cS$ est la 
classe des $V\in\cS$ tels qu'il existe un morphisme de $X$-schémas de $V$ dans 
l'un des $U_i$. 






\subsubsection{}\label{1-4-4}

Soit $\cU$ un crible sur $X$. On appelle module quasi-cohérent donné 
$\cU$-localement sur $X$ la donnée de 
\begin{enumerate}[\indent a)]
  \item un module quasi-cohérent $E_U$ sur chaque $U\in\cU$, 
  \item pour tout $U\in\cU$ et pour tout morphisme $\varphi:V\to U$ de 
    $X$-schémas dans $\cS$, un isomorphisme 
    $\rho_\varphi:E_V\iso \varphi^* E_U$, ceux-ci étant tels que 
  \item si $\psi:W\to V$ est un morphisme de $X$-schémas dans $\cS$, le 
    diagramme 
    \[\xymatrix{
      E_W \ar[r]^-{\rho_{\varphi\circ\psi}} \ar[dr]_-{\rho_\psi} 
        & \psi^*\varphi^* E_U \\
      & \psi^* E_V \ar[u]_-{\psi^*\rho_\varphi}
    }\]
    commute, c'est-à-dire 
    $\rho_{\varphi\circ\psi}=(\psi^*\rho_\varphi)\circ \rho_\psi$. 
\end{enumerate}

Si $E$ est un module quasi-cohérent sur $X$, on note $E_\cU$ le module 
donné $\cU$-localement valant $\varphi_U^* E$ sur $\varphi:U\to X$ et tel que, 
pour tout morphisme $\psi:V\to U$ l'isomorphisme de restriction $\rho_\psi$ 
soit l'isomorphisme canonique 
$E_V=(\varphi_U\circ \psi)^* E\iso \psi^* \varphi_U^* E = \psi^* E_U$. 


\begin{theorem}\label{1-4-5}
Soit $\{U_i\}\in\cS$ une famille finie de $X$-schémas plats sur $X$ telle que 
$X$ soit le réunion des images des $U_i$, et soit $\cU$ le crible engendré par 
$\{U_i\}$. Alors le foncteur $E\mapsto E_\cU$ est une équivalence de la 
catégorie des modules quasi-cohérents sur $X$ avec la catégorie des modules 
quasi-cohérents donnés $\cU$-localement.
\end{theorem}
\begin{proof}
Nous ne traiterons que le cas où $x$ est affine et où $\cU$ est engendré par 
un $X$-schéma affine $U$ fidèlement plat sur $X$. La réduction à ce cas est 
formelle. On pose $X=\spec(A)$ et $U=\spec(B)$. 

Si le morphisme $U\to X$ admet une section, $X$ appartient au crible $\cU$ et 
l'assertion est évidente. Nous nous réduirons à ce cas. 

Un module quasi-cohérent donné $\cU$-localement définit des modules $M'$, $M''$ 
et $M'''$ sur $U$, $U\times_X U$ et $U\times_X U\times_X U$, et des 
isomorphismes $\rho:p^* M^\bullet\simeq M^\bullet$ pour tout morphisme de 
projection $p$ entre ces espaces; c'est là un \emph{diagramme cartésien} 
\[\xymatrix{
  M^* : M' \ar@<2pt>[r] \ar@<-2pt>[r] 
    & M'' \ar[r] \ar@<4pt>[r] \ar@<-4pt>[r] 
    & M'''
}\]
au-dessus de 
\[\xymatrix{
  U_* : U 
    & U\times_X U \ar@<2pt>[l] \ar@<-2pt>[l] 
    & U\times_X U\times_X U \ar[l] \ar@<4pt>[l] \ar@<-4pt>[l] \mbox{.}
}\]
Réciproquement $M^*$ détermine le module donné $\cU$-localement: pour 
$V\in\cU$, il existe $\varphi:V\to U$ et on pose $M_U=\varphi^* M'$; 
pour $\varphi_1,\varphi_2:V\to U$, on a une identification naturelle 
$\varphi_1^* M'\simeq (\varphi_1\times \varphi_2)^* M'' \simeq \varphi_2^* M'$, 
et on voit en utilisant $M'''$ ue ces identifications sont compatibles, de 
sorte que la définition est légitime. Bref, il revient au même de se donner un 
module $\cU$-localement ou un diagramme $M^*$ cartésien sur $U_*$. 

Traduisons en termes algébriques: se donner $M^*$ revient à se donner un 
diagramme cartésian de modules 
\[\xymatrix{
  M' \ar@<3pt>[r]|-{\partial_0} \ar@<-3pt>[r]|-{\partial_1} 
    & M'' \ar@<6pt>[r]|-{\partial_0} \ar[r]|-{\partial_0} \ar@<-6pt>[r]|-{\partial_2}
    & M'''
}\]
au-dessus du diagramme d'anneaux 
\[\xymatrix{
  B \ar@<3pt>[r]|-{\partial_0} \ar@<-3pt>[r]|-{\partial_1} 
    & B\otimes_A B \ar@<6pt>[r]|-{\partial_0} \ar[r]|-{\partial_0} \ar@<-6pt>[r]|-{\partial_2} 
    & B\otimes_A B\otimes_A B
}\]
(précisons: on a $\partial_i(bm)=\partial_i(b)\cdot\partial_i(m)$, les 
identités usuelles telles que $\partial_0\partial_1=\partial_0\partial_0$ 
sont vraies, et ``cartésien'' signifie que les morphismes 
$\partial_i:M'\otimes_{B,\partial_i}(B\otimes_A B)\to M''$ et 
$M'''\otimes_{B\otimes_A B,\partial_i}(B\otimes_A B\otimes_A B)\to M'''$ sont 
des isomorphismes). 

Le foncteur $E\mapsto E_\cU$ devient le foncteur qui, à un $A$-module $M$, 
associe 
\[\xymatrix{
  M^* = (M\otimes_A B \ar@<2pt>[r] \ar@<-2pt>[r] 
    & M\otimes_A B\otimes_A B \ar@<-4pt>[r] \ar[r] \ar@<4pt>[r] 
    & M\otimes_A B\otimes_A B\otimes_A B)\text{.}
}\]
Il admet pour adjoint à droite le foncteur
\[\xymatrix{
  (M' \ar@<2pt>[r] \ar@<-2pt>[r] 
    & M'' \ar@<-4pt>[r] \ar[r] \ar@<4pt>[r]
    & M''') \ar@{|->}[r] 
    & \ker(M' \ar@<2pt>[r] \ar@<-2pt>[r] 
    & M''')\text{.}
}\]
Il nous faut prouver que les flèches d'adjonction 
\[
  M \to\ker(M\otimes_A B\rightrightarrows M\otimes_A B\times_A B)
\]
et
\[
  \ker(M''\twoheadrightarrow M'')\otimes_A B \to M'
\]
sont des isomorphismes. D'après (\ref{1-4-2}.i), il suffit de le prouver 
après un changement de base fidèlement plat $A\to A'$ ($B$ devenant 
$B'=B\otimes_A A'$). Prenant $A'=B$, ceci nous ramène au cas où $U\to X$ 
admet une section. 
\end{proof}










\subsection{Un cas particulier: le théorème 90 de Hilbert}\label{1-5}





\subsubsection{}\label{1-5-1}

Soient $k$ un corps, $k'$ une extension galoisienne de $k$ et $G=\gal(k'/k)$. 
Alors l'homomorphisme 
\begin{align*}
  k'\otimes_k k' &\to \oplus_{\sigma\in G} k' \\
  x\otimes y     &\mapsto \{x\cdot \sigma(y)\}_{\sigma\in G}
\end{align*}
est bijectif. 

On en déduit qu'il revient au même de se donner un module localement pour 
le crible engendré par $\spec(k')$ sur $\spec(k)$ ou de se donner un 
$k'$-espace vectoriel muni d'une action semi-linéaire de $G$, c'est-à-dire: 
\begin{enumerate}[\indent a)]
  \item un $k'$-espace vectoriel $V'$,
  \item pour tout $\sigma\in G$, un endomorphisme $\varphi_\sigma$ de la 
    structure de groupe de $V'$ tel que 
    $\varphi_\sigma(\lambda v) = \sigma(\lambda) \varphi_\sigma(v)$, pour 
    tout $v\in V'$, vérifiant la condition 
  \item pour tout $\sigma,\tau\in G$, on a 
    $\varphi_{\tau\sigma} = \varphi_\tau\circ\varphi_\sigma$. 
\end{enumerate}

Soit $V={V'}^G$ le groupe des invariants par cette action de $G$; c'est un 
$k$-espace vectoriel et, d'après le théorème (\ref{1-4-5}), on a:

\begin{proposition}\label{1-5-2}
L'inclusion de $V$ dans $V'$ définit un isomorphisme 
$V\otimes_k k' \iso V'$.
\end{proposition}

En particulier, si $V'$ est de dimension $1$ et si $v'\in V$ est non nul, 
$\varphi_\sigma$ est déterminé par la constante 
$c(\sigma)\in {k'}^\times$ telle que $\varphi_\sigma(v')=c(\sigma)v'$ et la 
condition c) s'écrit 
\[
  c(\tau\sigma) = c(\tau)\cdot \tau(c(\sigma))\text{.}
\]
D'après la proposition il existe un vecteur invariant non nui 
$v=\mu v'$, $\mu\in {k'}^\times$. On a donc pour tout $\sigma\in G$, 
\[
  c(\sigma) = \mu\cdot \sigma(\mu^{-1})\text{.}
\]
Autrement dit tout $1$-cocycle de $G$ à valeurs dans ${k'}^\times$ est un 
cobord: 

\begin{corollary}\label{1-5-3}
On a $H^1(G,{k'}^\times)=0$.
\end{corollary}










\subsection{Topologies de Grothendieck}\label{1-6}

Nous transcrivons maintenant les définitions des paragraphes précédents 
dans un cadre abstrait englobant à la fois le cas des espaces topologiques 
et celui des schémas. 





\subsubsection{}\label{1-6-1}

Soient $\cS$ une catégorie et $U$ un objet de $\cS$. On appelle \emph{crible} 
sur $U$ un sous-ensemble $\cU$ de $\ob(\cS/U)$ tel que si $\varphi:V\to U$ 
appartient à $\cU$ et si $\psi:W\to V$ est un morphisme dans $\cS$, alors 
$\varphi\circ\psi:W\to U$ appartient à $\cU$. 

Si $\{\varphi_i:U_i\to U\}$ est une famille de morphismes, le crible 
engendré par les $U_i$ est par définition l'ensemble des morphismes 
$\varphi:V\to U$ qui se factorisent à travers l'un des $\varphi_i$. 

Si $\cU$ est un crible sur $U$ et si $\varphi:V\to U$ est un morphisme, la 
restriction $\cU_V$ de $\cU$ à $V$ est par définition le crible sur $V$ 
constitué oar les morphismes $\psi:w\to V$ tels que 
$\varphi\circ\psi:W\to U$ appartienne à $\cU$.





\subsubsection{}\label{1-6-2}

La donnée d'une \emph{topologie de Grothendieck} sur $\cS$ consiste en la 
donnée pour tout objet $U$ de $\cS$ d'un ensemble $C(U)$ de cribles sur $U$, 
dits cribles couvrants, de telle sorte que les axiomes suivants soient 
satisfaits:
\begin{enumerate}[\indent a)]
  \item Le crible engendré par l'identité de $U$ est couvrant.
  \item Si $\cU$ est un crible couvrant sur $U$ et si $V\to U$ est un 
    morphisme, le crible $\cU_V$ est couvrant.
  \item Un crible localement couvrant est couvrant. Autrement dit, si $\cU$ est 
    un crible couvrant sur $U$ et si $\cU'$ est un crible sur $U$ tel que, pour 
    tout $V\to U$ appartenant à $\cU$, le crible $\cU_V'$ est couvrant, alors 
    $\cU'$ est couvrant.
\end{enumerate}

On appelle \emph{site} la donnée d'une catégorie munie d'une topologie de 
Grothendieck.





\subsubsection{}\label{1-6-3}

Etant donnée un site $\cS$, on appelle préfaisceau sur $\cS$ un foncteur 
contravariant $\cF$ de $\cS$ dans la catégorie des ensembles. Pour tout 
objet $U$ de $\cS$, on appelle section de $\cF$ au-dessus de $U$ les 
éléments de $\cF(U)$. Pour tout morphisme $V\to U$ et pour tout 
$s\in \cF(U)$, on note $s|V$ ($s$ restreint à $V$) l'image de $s$ dans 
$\cF(V)$.

Si $\cU$ est un crible sur $U$, on appelle section donnée $\cU$-localement la 
donnée, pour tout $V\to U$ appartenant à $\cU$, d'une section 
$s_V\in\cF(V)$ telle que, pour tout morphisme $W\to V$, on sit $s_V|W=s_W$. On 
dit que $\cF$ est un \emph{faisceau} si, pour tout objet $U$ de $\cS$, pour 
tout crible couvrant $\cU$ sur $U$ et pour tout section donnée 
$\cU$-localement $\{s_V\}$, il existe une unique section $s\in\cF(U)$ telle que 
$s|V=s_V$, pour tout $V\to U$ appartenant à $\cU$. 

On définit de manière analogue les \emph{faisceaux abéliens} en 
remplaçant la catégorie des ensembles par celle des groupes abéliens. On 
montre que la catégorie des faisceaux abéliens sur $\cS$ est une 
catégorie abélienne possédant suffisamment d'injectifs. Une suite 
$\cF\xrightarrow f \cG\xrightarrow g \cH$ de faisceaux est exacte si, pour 
tout objet $U$ de $\cS$, et pour tout $s\in \cG(U)$ telle que $g(s)=0$, 
il existe localement $t$ tel que $f(t)=s$; i.e. s'il existe un crible couvrant 
$\cU$ sur $U$ et pour tout $V\in\cU$, une section $t_V$ de $\cF$ sur $V$ telle 
que $f(t_V)=s|V$. 





\subsubsection{Exemples}\label{1-6-4}

Nous en avons vu deux plus haut. 

\begin{enumerate}[\indent a)]
  \item Soient $X$ un espace topologique et $\cS$ la catégorie dont les objets 
    sont les ouverts de $X$ et les morphismes les inclusions naturelles. La 
    topologique de Grothendieck sur $\cS$ correspondant à la topologie usuelle 
    de $X$ est celle pour laquelle un crible $\cU$ sur un ouvert $U$ de $X$ est 
    couvrant si la réunion des ouverts appartenant à ce crible est égale à 
    $U$. Il est clair que la catégorie des faisceaux sur $\cS$ est équivalente 
    à la catégorie des faisceaux sur $X$ au sens usuel. 
  \item Soient $X$ un schéma et $\cS$ la catégorie des schémas sur $X$. On 
    appelle topologie fpqc (fidèlement plate quasi-compacte) sur $\cS$ la 
    topologie de Grothendieck pour laquelle un crible sur un $X$-schéma $U$ est 
    couvrant s'il est engendré par une famille finie de morphismes plats dont  
    les images recouvrent $U$. 
\end{enumerate}





\subsubsection{Cohomologie}\label{1-6-5}

On supposera toujours que la catégorie $\cS$ a un objet final $X$. Alors on 
appelle sections globales d'un faisceaux abélien $\cF$, et on note 
$\Gamma\cF$ ou $H^0(X,\cF)$, le groupe $\cF(X)$. Le foncteur 
$\cF\mapsto\Gamma\cF$ est un foncteur exact \'a gauche de la catégorie des 
faisceaux abéliens sur $\cS$ dans la catégorie des groupes abéliens, ou 
note $H^i(X,\bullet)$ ses dérivés (ou satellites). Ces groupes de 
cohomologie représentent les obstructions à passer du local au global. Par 
définition, si $0\to\cF\to\cG\to\cH\to 0$ est une suite exacte de faisceaux 
abéliens, on a une suite exacte longue de cohomologie:
\[\xymatrix{
  0 \ar[r] 
    & H^0(X,\cF) \ar[r] 
    & H^0(X,\cG) \ar[r] 
    & H^0(X,\cH) \ar[r] 
    & H^1(X,\cF) \ar[r] 
    & \cdots \\
  \ldots \ar[r] 
    & H^n(X,\cF) \ar[r] 
    & H^n(X,\cG) \ar[r]
    & H^n(X,\cH) \ar[r] 
    & H^{n+1}(X,\cF) \ar[r] 
    & \cdots
}\]





\subsubsection{}\label{1-6-6}

Etant donné un faisceau abélien $\cF$ sur $\cS$, on appelle 
\emph{$\cF$-torseur} un faisceau $\cG$ muni d'une action $\cF\times\cG\to\cG$ 
de $\cF$ telle que localement (aprés restriction à tous les objets d'un 
crible couvrant l'objet final $X$) $\cG$ muni de l'action de $\cF$ soit 
isomorphe à $\cF$ muni de l'action canonique $\cF\times \cF\to \cF$ par 
translations. 

On peut montrer que $H^1(X,\cF)$ s'interprète comme l'ensemble des classes 
à isomorphisme près de $\cF$-torseurs. 




















\section{Topologie étale}\label{2}

On spécialise les définitions du chapitre précédent au cas de la 
topologie étale d'un schéma $X$ (\S \ref{2-1}, \ref{2-2}, \ref{2-3}). La 
cohomologie correspondante coincide dans le cas où $X$ est le spectre d'un 
corps $K$ avec la cohomologie galoisienne de $K$ (\S\ref{2-4}). 










\subsection{Topologie étale}\label{2-1}

Nous commencerons par quelques rappels sur la notion de morphisme étale. 

\begin{definition}\label{2-1-1}
Soit $A$ un anneau (commutatif). On dit qu'une $A$-algèbre $B$ est étale si 
$B$ est une $A$-algèbre de présentation finie et si les conditions 
équivalentes suivantes sont vérifiées:
\begin{enumerate}[\indent a)]
  \item Pour toute $A$-algèbre $C$ et pour tout idéal de carré nul $J$ de 
    $C$, l'application canonique 
    \[
      \hom_{A\textnormal{-alg}}(B,C) \to \hom_{A\textnormal{-alg}}(B,C/J)
    \]
    est une bijection.
  \item $B$ est un $A$-module plat et $\Omega_{B/A}=0$ (on note $\Omega_{B/A}$ 
    le module des différentielles relatives).
  \item Soit $B=A[X_1,\dotsc,X_n]/I$ une présentation de $B$. Alors pour tout 
    idéal premier $\fr$ de $A[X_1,\dotsc,X_n]$ contenant $I$, il existe des 
    polynômes $P_1,\dotsc,P_n\in I$ tels que $I_\fr$ soit engendré par les 
    images de $P_1,\dotsc,P_n$ et $\det(\partial P_i/\partial X_j)\notin \fr$.  
\end{enumerate}
\end{definition}
(cf. \cite[I]{7} ou \cite[V]{11})
%[cf. SGA I, exposé I ou M. {\sc Raynoud}, \emph{Anneaux Locaux Henséliens}, chapitre V]. 

On dit qu'un morphisme de schémas $f:X\to S$ est \emph{étale} si pour tout 
$x\in X$ il existe un voisinage ouvert affine $U=\spec(A)$ de $f(x)$ et un 
voisinage ouvert affine $V=\spec(B)$ de $x$ dans $X\times_S U$ tel que $B$ soit 
une $A$-algèbre étale. 





\subsubsection{Exemples}\label{2-1-2}
\begin{enumerate}[\indent a)]
  \item Si $A$ est un corps, une $A$-algèbre $B$ est étale si et seulement 
    si c'est un produit fini d'extensions séparables de $A$. 
  \item Si $X$ et $S$ sont des schémas de type fini sur $\dC$, un morphisme 
    $f:X\to S$ est étale si et seulement si son analyticité 
    $f^{\text{an}}:X^{\text{an}}\to Y^{\text{an}}$ est un isomorphisme local.
\end{enumerate}





\subsubsection{Sorite}\label{2-1-3}
\begin{enumerate}[\indent a)]
  \item (changement de base) Si $f:X\to S$ est un morphisme étale, il en est 
    de même de $f_{S'}:X\times_S S'\to S'$ pour tout morphisme $S'\to S$. 
  \item Si $f:X\to S$ et $g:Y\to S$ sont deux morphismes étales, tout 
    $S$-morphisme de $X$ dans $Y$ est étale.
  \item (descente) Soit $f:X\to S$ un morphisme. S'il existe un morphisme 
    fidèlement plat $S'\to S$, tel que $f_{S'}:X\times_S S'\to S'$ soit 
    étale, alors $f$ est étale.
\end{enumerate}





\subsubsection{}\label{2-1-4}

Soit $X$ un schéma. Soit $\cS$ la catégorie des $X$-schémas étales; 
d'après (\ref{2-1-3}.c) tout morphisme de $\cS$ est un morphisme étale. 
On appelle \emph{topologie étale} sur $\cS$ la topologie pour laquelle un 
crible sur $U$ est couvrant s'il est engendré par une famille finie de 
morphismes $\varphi_i:U_i\to U$ tels que la réunion des images des 
$\varphi_i$ recouvre $U$. On appelle \emph{site étale} de $X$, et note 
$\et X$, le site défini par $\cS$ de la topologie étale. 










\subsection{Exemples de faisceaux}\label{2-2}





\subsubsection{Faisceau constant}\label{2-2-1}

Soit $C$ un groupe abélien et supposons pour simplifier $X$ noethérien. On 
notera $\const C_X$ (ou même $C$ s'il n'y a pas d'ambiguïté) le faisceau 
défini par $U\mapsto C^{\pi_0(U)}$, où $\pi_0(U)$ est l'ensemble (fini) des 
composantes connexes de $U$. Le cas le plus important sera $C=\dZ/n$. On a donc 
par définition 
\[
  H^0(X,\dZ/n)=\left(\dZ/n\right)^{\pi_0(X)}\text{.}
\]
De plus $H^1(X,\dZ/n)$ est l'ensemble des classes d'isomorphisme de 
$\dZ/n$-torseurs (\ref{1-6-6}), autrement dit de revêtements étales 
galoisiens de $X$ de groupe $\dZ/n$. En particulier, si $X$ est connexe et si 
$\pi_1(X)$ est son groupe fondamental pour un pointe bas choisi, on a 
\[
  H^1(X,\dZ/n) = \hom(\pi_1(X),\dZ/n)\text{.}
\]





\subsubsection{Groupe multiplicatif}\label{2-2-2}

On notera $\dG_{m,X}$ (ou $\dG_m$ s'il n'y a pas d'ambiguïté) le faisceau 
défini par $U\mapsto \Gamma(U,\cO_U^\times)$; il s'agit bien d'un faisceau 
grâce au théorème de descente fidèlement plate (\ref{1-4-5}). On a 
par définition 
\[
  H^0(X,\dG_m) = H^0(X,\cO_X)^\times\text{;}
\]
en particulier si $X$ est réduit, connexe et propre sur un corps 
algébriquement clos $k$, on a:
\[
  H^0(X,\dG_m) = k^\times\text{.}
\]

\begin{proposition}\label{2-2-3}
On a un isomorphisme:
\[
  H^1(X,\dG_m) = \pic(X)\text{,}
\]
où $\pic(X)$ est le groupe des classes de faisceaux inversibles sur $X$.
\end{proposition}
\begin{proof}
Soit $*$ le foncteur qui, à un faisceau inversible $\cL$ sur $X$, associe le 
préfaisceau $\cL^*$ suivant sur $\et X$: pour $\varphi:U\to X$ étale, 
\[
  \cL^*(U)=\operatorname{Isom}_U(\cO_U,\varphi^*\cL)\text{.}
\]
% the original references here are just to (4.2) and (4.5), but the context 
% makes it seem that they refer to the results in chapter 1. Likewise a little 
% later
D'après (\ref{1-4-2}.i) et et (\ref{1-4-5}) (pleine fidélité), ce 
préfaisceau est un faisceau; c'est même un $\dG_m$-torseur. On vérifie 
aussitôt que 
\begin{enumerate}[\indent a)]
  \item le foncteur $*$ est compatible à la localisation (étale);
  \item il induit une équivalence de la catégorie des faisceaux 
    inversibles triviaux (i.e. à $\cO_X$) avec la catégorie des 
    $\dG_m$-torseurs triviaux: $\cL$ est trivial si et seulement si $\cL^*$ 
    l'est.
\end{enumerate}

De plus, d'aprés (\ref{1-4-2}.ii) et (\ref{1-4-5}), 
\begin{enumerate}[\indent a)]
\setcounter{enumi}{2}
  \item la notion de faisceau inversible est locale pour la topologie étale. 
\end{enumerate}

Il résulte formellement de a), b), c) que $*$ est une équivalence entre la 
catégorie des faisceaux inversibles sur $X$ est celle des $\dG_m$-torseurs 
sur $\et X$; elle induit l'isomorphisme cherché. On construit comme suit 
l'équivalence inverse: si $T$ est un $\dG_m$-torseur, il existe un 
recouvrement étale fini $\{U_i\}$ de $X$ tel que les torseurs $T/U_i$ soient 
triviaux; $T$ est alors trivial sur chaque $V$ étale sur $X$ appartenant au 
crible $\cU\subset \et X$ engendré par $\{U_i\}$. Sur chaque 
$V\in\cU$, $T|V$ correspond à un faisceau inversible $\cL_V$ (par b)) et les 
$\cL_V$ constituent un faisceau inversible donné $\cU$-localement $\cL_\cU$ 
(par a)). Par c), ce dernier provient d'un faisceau inversible $\cL(T)$ sur 
$X$, et $T\mapsto \cL(T)$ est l'inverse cherché de $*$. 
\end{proof}





\subsubsection{Racines de l'unité}\label{2-2-4}

Pour tout entier $n>0$, on appelle faisceau des racines $n$-ièmes de 
l'unité, et on note $\dmu_n$, le noyau d l'élévation à puissance 
$n$-iéme dans $\dG_m$. Si $X$ est un schéma sur un corps séparablement 
clos $k$ et si $n$ est inversible dans $k$, le choix d'une racine primitive 
$n$-ième de l'unité $\zeta\in k$ définit un isomorphisme 
$i\mapsto \zeta^i$ de $\dZ/n$ avec $\dmu_n$. 

La relation entre cohomologie à coefficients dans $\dmu_m$ et cohomologie à 
coefficients dans $\dG_m$ est donnée par la suite exacte de cohomologie déduite 
de la 






\begin{theorem}[Théorie de Kummer]\label{2-2-5}
Si $n$ est inversible sur $X$, l'élévation à la puissance $n$-ième dans 
$\dG_m$ est un épimorphisme de faisceaux. On a donc une suite exacte 
\[
  0 \to \dmu_n \to \dG_m \to \dG_m \to 0\text{.}
\]
\end{theorem}
\begin{proof}
Soient $U\to X$ un morphisme étale et $a\in \dG_m(U) =\Gamma(U,\cO_U^\times)$. 
Puisque $n$ est inversible sur $U$, l'équation $T^n-a = 0$ est séparable; 
autrement dit $U'=\spec\left(\cO_U[T]/(T^n-a)\right)$ est étale su-dessus de 
$U$. Par ailleurs $U'\to U$ est surjectif et a admet une racine $n$-ième sur 
$U'$, d'où résultat. 
\end{proof}










\subsection{Fibres, images directes}\label{2-3}





\subsubsection{}\label{2-3-1}

On appelle \emph{pointe géométrique} de $X$ un morphisme $\bar x\to X$, où 
$\bar x$ est le spectre d'un corps séparablement clos $k(\bar x)$. On le notera 
abusivement $\bar x$, sous-entendent le morphisme $\bar x\to X$. Si $x$ est 
l'image de $\bar x$ dans $X$, on dit que $\bar x$ est centré en $x$. Si le 
corps $k(\bar x)$ est une extension algébrique du corps résiduel $k(x)$, on dit 
que $\bar x$ est un point géométrique \emph{algébrique} de $X$. 

On appelle \emph{voisinage étale} de $\bar x$ un diagramme commutatif 
\[\xymatrix{
  & U \ar[d] \\
  \bar x \ar[ur] \ar[r] 
  & X\text{,}
}\]
où $U\to X$ est un morphisme étale. 

Le \emph{localisé stricte} de $X$ en $\bar x$ est l'anneau 
$\cO_{X,\bar x} = \varinjlim \Gamma(U,\cO_U)$, la limite inductive étant sur les 
voisinages étales de $\bar x$. C'est un anneau local strictement hensélien dont 
le corps résiduel est la clôture séparable du corps résiduel $k(x)$ de $X$ en 
$x$ dans $k(\bar x)$. Il joue le rôle d'anneau local pour la topologie étale. 





\subsubsection{}\label{2-3-2}

Étant donné un faisceau $F$ sur $\et X$, on appelle \emph{fibre} de $F$ en 
$\bar x$ l'ensemble (resp. le groupe,\dots) $F_{\bar x}=\varprojlim F(U)$, 
la limite inductive étant toujours prise sur les voisinages étales de  $X$. 

Pour qu'un homomorphisme de faisceaux $F\to G$ soit un mono-/epi-/isomorphisme 
il faut et il suffit qu'il en soit ainsi des morphismes 
$F_{\bar x}\to G_{\bar x}$ induit sur les fibres et tout point géométrique de 
$X$. Si $X$ est de type fini sur un corps algébriquement clos, il suffit qu'il 
en soit ainsi en les points rationnelles de $X$. 





\subsubsection{}\label{2-3-3}

Si $f:X\to Y$ est un morphisme de schémas et $F$ un faisceau sur $\et X$, 
l'\emph{image directe} $f_* F$ de $F$ par $f$ est le faisceau sur $\et Y$ 
défini par $f_* F(V) = F(X\times_Y V)$ pour tout $V$ étale sur $Y$. Le foncteur 
$f_*:\ab\sh(\et X)\to \ab\sh(\et Y)$ est exact à gauche. Ses foncteurs 
dérivés à droite $R^q f_*$ s'appellent images directes supérieures. Si 
$\bar y$ est un point géométrique de $Y$, on a 
\[
  (R^q f_* F)_{\bar y} = \varinjlim H^q(V\times_Y X,F)\text{,}
\]
limite inductive prise sur les voisinages étales $V$ de $\bar y$. 

Soient $\cO_{Y,\bar y}$ le localisé stricte de $Y$ en $\bar y$, 
$\widetilde Y=\spec(\cO_{Y,\bar y})$ et $\widetilde X=X\times_Y \widetilde Y$. 
On peut étendre $F$ à $\et{\widetilde X}$ (c'est un cas particulier de la 
notion générale d'image réciproque) de la manière suivante: soit 
$\widetilde U$ un schéma étale sur $\widetilde X$, alors il existe un 
voisinage étale $V$ de $\bar y$ et un schéma étale $U$ sur $X\times_Y V$ tel 
que $\widetilde U=U\times_V \widetilde Y$; on posera 
\[
  F(\widetilde U)=\varinjlim F\left(U\times_V V'\right)\text{,}
\]
le limite inductive étant prise sur les voisinages étales $V'$ de $\bar y$ qui 
dominent $V$. Avec cette définition, on a 
\[
  \left(R^q f_* F\right)_{\bar y} = H^q(\widetilde X,F)\text{.}
\]

Le foncteur $f_*$ a un adjoint à gauche $f^*$, le foncteur ``image 
réciproque.'' Si $\bar x$ est un point géométrique de $X$ et $f(\bar x)$ son 
image dans $Y$, on a $(f^* F)_{\bar x} = F_{f(\bar x)}$. Cette formule montre 
que $f^*$ est un foncteur exact. Le foncteur $f_*$ transforme donc faisceau 
injectif en faisceau injectif, et la suite spectrale du foncteur compose 
$\Gamma\circ f_*$ (resp. $g_* f_*$) fournit la 





\begin{theorem}[Suite spectrale de Leray]\label{2-3-4}
Soient $F$ un faisceau abélien sur $\et X$ et $f:X\to Y$ un morphisme de 
schémas (resp. des morphismes de schémas $X\xrightarrow f Y \xrightarrow g Z$). 
On a une suite spectrale 
\begin{align*}
  E_2^{pq} &= H^p(Y,R^q f_* F) \Rightarrow H^{p+q}(X,F) \\
  \text{(resp.}\qquad E_2^{pq} &= R^p g_* R^q f_* F \Rightarrow R^{p+q}(gf)_* F\text{).}
\end{align*}
\end{theorem}





\begin{corollary}\label{2-3-5}
Si $R^q f_* F=0$ pour tout $q>0$, on a $H^p(Y,f_* F)=H^p(X,F)$ (resp. 
$R^p g_*(F_* F) = R^p(g f)_* F$) pour tout $p\geqslant 0$. 
\end{corollary}

Cela s'applique en particulier dans le cas suivant:





\begin{proposition}\label{2-3-6}
Soit $f:X\to Y$ un morphisme fini (voire, par passage à la limite, un 
morphisme entier) et $F$ un faisceau abélien sur $X$. Alors $R^q f_* F=0$, 
pour tout $q>0$.
\end{proposition}

En effet soient $\bar y$ un pointe géométrique de $Y$, $\widetilde Y$ le 
spectre du localisé strict de $Y$ en $y$ et 
$\widetilde X=X\times_Y \widetilde Y$; d'après ce qui précède, il suffit de 
montrer que $H^q(\widetilde X,F) = 0$ pour tout $q>0$. Or $\widetilde X$ est le 
spectre d'un produit d'anneaux locaux strictement henséliens (cf. \cite[I]{11}), le foncteur $\Gamma(\widetilde X,\bullet)$ est 
exact car tout $\widetilde X$-schéma étale et surjectif admet une section, 
d'où l'assertion. 










\subsection{Cohomologie galoisienne}\label{2-4}

Pour $X=\spec(K)$ le spectre d'un corps, nous allons voir que la cohomologie 
étale s'identifie à la cohomologie galoisienne. 





\subsubsection{}\label{2-4-1}

Commençons par une analogie topologique. Si $K$ est le corps des fonctions 
d'une variété algébrique affine intègre $Y=\spec(A)$ sur $\dC$, on a 
$K=\varprojlim_{f\in A} A[1/f]$. 

Autrement dit $X=\varprojlim U$, $U$ parcourant l'ensemble des ouverts de $Y$. 
On sait qu'il existe des ouverts de Zariski arbitrairement petits qui pour la 
topologie classique sont des $K(\pi,1)$. On ne sera donc pas surpris si l'on 
considère $\spec(K)$ lui-même comme un $K(\pi,1)$, $\pi$ étant le groupe 
fondamental (au sens algébrique) de $X$, autrement dit le groupe de Galois de 
$\bar K/K$, où $\bar K$ est clôture séparable de $K$. 





\subsubsection{}\label{2-4-2}

Plus précisément soient $K$ un corps, $\bar K$ un clôture séparable de $K$ et 
$G = \gal(\bar K/K)$ le groupe de Galois topologique. A toute $K$-algèbre finie 
étale $A$ (produit fini d'extensions séparables de $K$), associons l'ensemble 
fini $\hom_K(A,\bar K)$. Le groupe de Galois $G$ opère sur cet ensemble à 
travers un quotient discret (donc fini). Si $A=K[T]/(F)$, il s'identifie à 
l'ensemble des racines dans $\bar K$ du polynôme $F$. La théorie de Galois, 
sous la forme que lui a donnée Grothendieck, dit que:





\begin{proposition}\label{2-4-3}
Le foncteur 
\[
  \left(\begin{array}{c}
          \textnormal{$K$-algèbres} \\ 
          \textnormal{finies étales}
        \end{array}\right)
  \to 
  \left(\begin{array}{c}
          \textnormal{ensembles finis sur lesquels} \\
          \textnormal{$G$ opère continûment}
        \end{array}\right)
\]
qui à une algèbre étale $A$ associe $\hom_K(A,\bar K)$ est une 
anti-équivalence de catégories.
\end{proposition}

On en déduit une description analogue des faisceaux pour la topologie étale sur 
$\spec(K)$:





\begin{proposition}\label{2-4-4}
Le foncteur 
\[
  \left(\begin{array}{c}
          \textnormal{Faisceaux étales} \\ 
          \textnormal{sur $\spec(K)$}
        \end{array}\right)
  \to 
  \left(\begin{array}{c}
          \textnormal{ensembles sur lesquels} \\
          \textnormal{$G$ opère continûment}
        \end{array}\right)
\]
qui à un 
faisceau $F$ associe sa fibre $F_{\bar K}$ au point géométrique 
$\spec(\bar K)$ est une équivalence de catégories.
\end{proposition}

On dit que $G$ opère continûment sur un ensemble $E$ si le fixateur de tout 
élément de $E$ est un sous-groupe ouvert de $G$. Le foncteur en sens 
inverse est décrit de la manière évidente: soient $A$ une 
$K$-algèbre finie étale, $U=\spec(A)$ et $U(\bar K)=\hom_K(A,\bar K)$ le 
$G$-ensemble correspondant à; alors on a 
$F(U)= \hom_{G\text{-ens}}(U(\bar K),F_{\bar K})$. 

En particulier, si $X=\spec(K)$, on a $F(X) = F_{\bar K}^G$. Si l'on se 
restreint aux faisceaux abéliens, on obtient en passant aux foncteurs 
dérivés des isomorphismes canoniques 
\[
  H^q(\et X,F) = H^q(G,F_{\bar K})
\]





\subsubsection{Exemples}\label{2-4-5}

\begin{enumerate}[\indent a)]
  \item Au faisceau constant $\dZ/n$ correspond $\dZ/n$ avec action triviale de 
    $G$. 
  \item Au faisceau des racines $n$-ièmes de l'unité $\dmu_n$ correspond 
    le groupe $\dmu_n(\bar K)$ des racines $n$-ièmes de l'unité dans 
    $\bar K$, avec l'action naturelle de $G$.
  \item Au faisceau $\dG_m$ correspond le groupe $\bar K^\times$ avec l'action 
    naturelle de $G$.
\end{enumerate}




















\section{Cohomologie des courbes}\label{3}

Dans le cas des espaces topologiques, des dévissages utilisant la formule de 
K\"unneth et des décompositions simpliciales permettent de se ramener pour 
calculer la cohomologie à l'intervalle $I=[0,1]$ pour lequel on a 
$H^0(I,\dZ)=\dZ$ et $H^q(I,\dZ)=0$ pour $q>0$. 

Dans notre cas, les dévissages aboutiront à des objets plus compliqués, 
à savoir les courbes sur un corps algébriquement clos; nous allons calculer 
leur cohomologie dans ce chapitre. La situation est plus complexe que dans le 
cas topologique car les groupes de cohomologie sont nuis pour $q>2$ seulement. 
L'ingrédient essentiel des calculs est la nullité du groupe de Brauer du 
corps des fonctions d'une telle courbe (théorème de Tsen, \S\ref{3-2}). 










\subsection{Le groupe de Brauer}\label{3-1}

Rappelons-en tout d'abord la définition classique:





\begin{definition}\label{3-1-1}
Soit $K$ un corps et $A$ une $K$-algèbre de dimension finie. On dit que $A$ 
est une algèbre simple centrale sur $K$ si les conditions équivalentes 
suivantes sont vérifiées:
\begin{enumerate}[\indent a)]
  \item $A$ n'a pas d'idéal bilatère non trivial et son centre est $K$. 
  \item Il existe une extension galoisienne finie $K'/K$ telle que 
    $A_{K'} = A\otimes_K K'$ soit isomorphe à une algèbre de matrices 
    carrées sur $K'$.
  \item $A$ est $K$-isomorphe à algèbre de matrices carrées sur un corps 
    gauche de centre $K$.
\end{enumerate}
\end{definition}

Deux telles algèbres sont dites équivalentes si les corps gauches qui 
leur sont associés par c) sont $K$-isomorphes. Si ces algèbres out 
même dimension, cela revient à dire qu'elles sont $K$-isomorphes. Le 
produit tensoriel définit par passage au quotient une structure de groupe 
abélien sur l'ensemble des classes d'équivalence. C'est ce groupe que l'on appelle classiquement \emph{le groupe de Brauer} de $K$ et que l'on note 
$\br(K)$. 





\subsubsection{}\label{3-1-2}

On notera $\br(n,K)$ l'ensemble des classes de $K$-isomorphisme de 
$K$-algèbres $A$ telles qu'il existe une extension galoisienne finie $K'$ de 
$K$ pour laquelle $A_{K'}$ est isomorphe à l'algèbre $M_n(K')$ des matrices 
carrées $n\times n$ sur $K'$. Par définition $\br(K)$ est réunion des 
sous-ensembles $\br(n,K)$ pour $n\in\dN$. Soient $\bar K$ une clôture 
algébrique de $K$ et $G=\gal(\bar K/K)$. L'ensemble $\br(n,K)$ est 
l'ensemble des ``formes'' de $M_n(\bar K)$, il est donc canoniquement 
isomorphe à $H^1\left(G,\aut(M_n(\bar K))\right)$. 

On sait que tout automorphisme de $M_n(\bar K)$ est intérieur. Par 
conséquent le groupe $\aut(M_n(\bar K))$ s'identifie au groupe linéare 
projectif $\pgl(n,\bar K)$ et on a une bijection canonique:
\[
  \theta_n : \br(n,K) \iso H^1\left(G,\pgl(n,\bar K)\right)\text{.}
\]

D'autre part la suite exacte:
\begin{equation}\label{eq:1}
  1 \to \bar K^\times \to \gl(n,\bar K) \to \pgl(n,\bar K) \to 1\text{,}
\end{equation}
permet de définir un opérateur cobord:
\[
  \Delta_n : H^1\left(G,\pgl(n,\bar K)\right) \to H^2(G,\bar K^\times)\text{.}
\]

En composant $\theta_n$ et $\Delta_n$, on obtient une application:
\[
  \delta_n : \br(n,K) \to H^2(G,\bar K^\times)\text{.}
\]
On vérifie facilement que les applications $\delta_n$ sont compatibles entre 
elles et définissent un homomorphisme de groupes:
\[
  \delta : \br(K) \to H^2(G,\bar K^\times)\text{.}
\]





\begin{proposition}\label{3-1-3}
L'homomorphisme $\delta:\br(K)\to H^2(G,\bar K^\times)$ est bijectif. 
\end{proposition}

Cela résulte des deux lemmes suivants:





\begin{lemma}\label{3-1-4}
L'application 
$\Delta_n:H^1\left(G,\pgl(n,\bar K)\right)\to H^2(G,\bar K^\times)$ est 
injective.
\end{lemma}

D'après \cite{14}, cor. à la prop. I-44, il suffit de vérifier que chaque 
fois qu'on tord la suite exacts (\ref{eq:1}) par un élément de 
$H^1\left(G,\pgl(n,\bar K)\right)$, le $H^1$ du groupe médian est trivial. 
Ce groupe médian est le groupe des $\bar K$-points du groupe multiplicatif 
d'une algèbre centrale simple $A$ de rang $n^2$ sur $K$. Pour prouver que 
$H^1(G,A_{\bar K}^\times) = 0$, n interprète $A^\times$ comme le groupe 
des automorphismes du $A$-module libre $L$ de rang $1$, et $H^1$ comme 
l'ensemble des ``formes'' de $L$ -- des $A$-modules de rang $n^2$ sur $K$, 
automatiquement libres. 





\begin{lemma}\label{3-1-5}
Soient $\alpha\in H^2(G,\bar K^\times)$, $K'$ un extension finie de $K$ 
contenue dans $\bar K$, $n=[K':K]$, et $G'=\gal(\bar K/K')$. Si l'image de 
$\alpha$ dans $H^2(G',\bar K^\times)$ est nulle, alors, $\alpha$ appartient 
à l'image de $\Delta_n$. 
\end{lemma}

Remarquons tout d'abord qu'on a: 
\[
  H^2(G',\bar K^\times) \simeq H^2\left(G,(\bar K\otimes_K K')^\times\right)\text{.}
\]
(D'un point de vue géométrique si l'on note $x=\spec(K)$, $x'=\spec(K')$ et 
$\pi:x'\to x$ le morphisme canonique, on a $R^q \pi_*(\dG_{m,x'}) = 0$ pour 
$q>0$ et par suite $H^q(x',\dG_{m,X'})\simeq H^q(x,\pi_*\dG_{m,X'})$ pour 
$q\geqslant 0$). 

Par ailleurs la choix d'une base de $K'$ en tant qu'espace vectoriel sur $K$ 
permet de définir un homomorphisme 
\[
  (\bar K\otimes_K K')^\times \to \gl(n,\bar K)
\]
qui, à un élément $x$, fait correspondre l'endomorphisme de 
multiplication par $x$ de $\bar K\otimes_K K'$. On a alors un diagramme 
commutatif à lignes exactes: 
\[\xymatrix{
  1 \ar[r] 
    & \bar K^\times \ar[r] \ar@{=}[d]
    & (\bar K\otimes_K K')^\times \ar[r] \ar[d] 
    & (\bar K\otimes_K K')^\times / \bar K^\times \ar[r] \ar[d] 
    & 1 \\
  1 \ar[r] 
    & \bar K^\times \ar[r] 
    & \gl(n,\bar K) \ar[r] 
    & \pgl(n,\bar K) \ar[r] 
    & 1
}\]
Le lemme résulte du diagramme commutatif que l'on en déduit en passant 
à la cohomologie:
\[\xymatrix{
  H^1\left(G,(\bar K\otimes_K K')^\times/\bar K^\times\right) \ar[r] \ar[d] 
    & H^2(G,\bar K^\times) \ar[r] \ar@{=}[d]
    & H^2\left(G,(\bar K\otimes_K K')^\times\right) \\
  H^1\left(G,\pgl(n,\bar K)\right) \ar[r]^-{\Delta_n} 
    & H^2(G,\bar K^\times) \text{.}
}\]





\begin{proposition}\label{3-1-6}
Soient $K$ un corps, $\bar K$ une clôture algébrique de $K$ et 
$G=\gal(\bar K/K)$. Supposons que, pour tout extension finie $K'$ de $K$, on 
ait $\br(K')=0$. Alors on a:
\begin{enumerate}[\indent i)]
  \item $H^q(G,\bar K^\times) = 0$ pour tout $q>0$.
  \item $H^q(G,F) = 0$ pour tout $G$-module de torsion $F$ et pour tout 
    $q\geqslant 2$.
\end{enumerate}
\end{proposition}

(Pour la démonstration, cf. \cite{14}).










\subsection{Le théorème de Tsen}\label{3-2}





\begin{definition}\label{3-2-1}
On dit qu'un corps $K$ est $C_1$ si tout polynôme homogène non constant 
$f(x_1,\dotsc,x_n)$ de degré $d<n$ a un zéro non trivial.
\end{definition}





\begin{proposition}\label{3-2-2}
Si un corps $K$ est $C_1$, on a $\br(K)=0$.
\end{proposition}

Il s'agit de montrer que tout corps gauche $D$ de centre $K$ et fini sur $K$ 
est égale à $K$. Soient $r^2$ le degré de $D$ sur $K$ et $\nrd:D\to K$ la 
norme réduite. 

(Localement pour la topologie étale sur $K$, $D$ est isomorphe -- non 
canoniquement -- à une algèbre de matrices $M_r$ et la norme réduite 
coïncide avec l'application déterminant. Celle-ci est bien définie, 
indépendamment de l'isomorphisme choisi entre $D$ et $M_r$ car tout 
automorphisme de $M_r$ est intérieur et deux matrices semblables ont 
m\^eme déterminant. Cette application définie localement pour la 
topologie étale se descende, à cause de son unicité locale, en une 
application $\nrd:D\to K$). 

Le seul zéro de $\nrd$ est l'élément nul de $D$, car, si $x\ne 0$, on a 
$\nrd(x)\cdot \nrd(x^{-1}) = 1$. D'autre part, si $\{e_1,\dotsc,e_{r^2}\}$ est 
une base de $D$ sur $K$ et si $x=\sum x_i e_i$, la fonction $\nrd(x)$ s'écrit 
comme un polynôme homogène $\nrd(x_1,\dotsc,x_{r^2})$ de degré $r$ (c'est 
clair localement pour la topologie étale). Puisque $K$ est $C_1$, on a 
$r^2\leqslant r$, c'est-à-dire $r=1$ et $D=K$. 





\begin{theorem}[Tsen]\label{3-2-3}
Soient $k$ un corps algébriquement clos et $K$ une extension de degré de 
transcendance $1$ de $k$. Alors $K$ est $C_1$.
\end{theorem}

Supposons tout d'abord que $K=k(X)$. Soit 
\[
  f(\underline T) = \sum a_{i_1,\dotsc i_n} T_1^{i_1} \dotsm T_n^{i_n}
\]
un polynôme homogène de degré $d<n$ à coefficients dans $k(X)$. Quitte 
à multiplier les coefficients par un dénominateur commun on peut supposer 
qu'ils sont dans $k[X]$. Soit alors $\delta=\sup\deg(a_{i_1\dotsc i_n})$. On 
cherche un zéro non trivial dans $k[X]$ par la méthode des coefficients 
indéterminés en écrivant chaque $T_i$ ($i=1,\dotsc,n$) comme un 
polynôme de degré $N$ en $X$. Alors l'équation 
$f(\underline T)=0$ devient un système d'équations homogènes en les 
$n\times (N+1)$ coefficients des polynômes $T_i(X)$ exprimant la nullité 
des coefficients du polynôme en $X$ obtenu en remplaçant $T_i$ par 
$T_i(X)$. Ce polynôme est de degré $\delta+N D$ au plus, il y a donc 
$\delta+N d+1$ équations en $n\times (N+1)$ variables. Comme $k$ est 
algébriquement clos ce système a une solution non triviale si 
$n(N+1)>N d+\delta+1$, ce qui sera le cas pour $N$ assez grand si $d<n$. 

Il est clair que, pour démontrer le théorème dans la cas général, il 
suffit de le démontrer lorsque $K$ est une extension finie d'une extension 
transcendent pure $k(X)$ de $k$. Soit $F(\underline T)=f(T_1,\dotsc,T_n)$ un 
polynôme homogène de degré $d<n$ à coefficients dans $K$. Soient 
$a=[K:k(X)]$ et $e_1,\dotsc,e_s$ une bas de $K$ sur $k(X)$. Introduisons de 
nouvelles variables $U_{i j}$, en nombre $s n$, telles que 
$T_i=\sum U_{i j}e_j$. Pour que le polynôme $f(\underline T)$ ait un zéro 
non trivial dans $K$, il suffit que le polynôme 
$g(X_{i j}) = N_{K/k}(f(\underline T))$ ait un zéro non trivial dans $k(X)$. 
Or $g$ est un polynôme de degré $s d$ en $s n$ variables, d'où le 
résultat. 





\begin{corollary}\label{3-2-4}
  Soient $k$ un corps algébriquement clos et $K$ une extension de degré 
de transcendance $1$ de $k$. Alors les groupes de cohomologie étale 
$H^q(\spec(K),\dG_m)$ sont nuls pour tout $q>0$. 
\end{corollary}










\subsection{Cohomologie des courbes lisses}

Dorénavant, et sauf mention expresse du contraire, les groupes de 
cohomologie considérés sont les groupes des cohomologie étale. 





\begin{proposition}\label{3-3-1}
Soient $k$ un corps algébriquement clos et $X$ une courbe projective non 
singulière connexe sur $k$. Alors on a:
\begin{align*}
  H^0(X,\dG_m) &= k^\times \text{,} \\
  H^1(X,\dG_m) &= \pic(X) \text{,} \\
  H^q(X,\dG_m) &= 0 \text{ pour $q\geqslant 2$.}
\end{align*}
\end{proposition}

Soient $\eta$ le point générique de $X$, $j:\eta\to X$ le morphisme 
canonique et $\dG_{m,X}$ le groupe multiplicatif du corps des fractions 
$K(X)$. Pour tout pointe fermé $x$ de $X$, soient $i_x:x\to X$ l'immersion 
canonique et $\dZ_x$ le faisceau constant de valeur $\dZ$ sur $x$. Ainsi 
$j_*\dG_{m,\eta}$ est le faisceau des fonctions méromorphes non nulles sur 
$X$ et $\oplus_{x\in X} i_{x_I}\dZ_x$ le faisceau des diviseurs, on a donc une 
suite exacte de faisceaux:
\begin{equation}\label{eq:2}
\xymatrix{
  0 \ar[r] 
    & \dG_m \ar[r]
    & j_* \dG_{m,\eta} \ar[r]^-{\dv} \ar[r] 
    & \bigoplus_{x\in X} i_{x*} \dZ_x \ar[r] 
    & 0\text{.}
}
\end{equation}





\begin{lemma}\label{3-3-2}
On a $R^q j_* \dG_{m,\eta} = 0$ pour tout $q>0$.
\end{lemma}

Il suffit de montrer que la fibre de ce faisceau en tout pointe fermé $x$ 
de $X$ est nulle. Si $\tilde\cO_{X,x}$ est l'hensélisé de $X$ en $x$ et $K$ 
le corps des fractions de $\tilde\cO_{X,x}$, on a 
\[
  \spec(K) = \eta\times_X \spec(\tilde\cO_{X,x})\text{,}
\]
donc $(R^qj_*\dG_{m,\eta})_x = H^q(\spec(K),\dG_m)$. 

Or $K$ est une extension algébrique de $k(X)$, donc une extension de degré 
de transcendante $1$ de $k$: le lemme résulte de (\ref{3-2-4}). 





\begin{lemma}\label{3-3-3}
On a $H^q(X, j_*\dG_{m,\eta}) = 0$ pour tout $q>0$.
\end{lemma}

En effet de (\ref{3-3-2}) et de la suite spectrale de Leray pour $j$, on 
déduit:
\[
  H^q(X, j_* \dG_{m,\eta}) = H^q(\eta,\dG_{m,\eta})
\]
pour tout $q\geqslant 0$ et le deuxième membre est nul pour $q>0$ d'après 
(\ref{3-2-4}). 





\begin{lemma}\label{3-3-4}
On a $H^q\left(X,\bigoplus_{x\in X} i_{x*} \dZ_x\right) = 0$ pour tout $q>0$. 
\end{lemma}

En effet pour tout point fermé de $X$, on a $R^q i_{x*}\dZ_x 0$ pour $q>0$, 
car $i_x$ est un morphisme fini (\ref{2-3-6}), et 
\[
  H^q(X,i_{x*}\dZ_x) = H^q(x,\dZ_x)\text{.}
\]
Le deuxième membre est nul pour tout $q>0$, car $x$ est la spectre d'un 
corps algébriquement clos (On voit que le lemme est vrai plus 
généralement pour tout faisceau ``gratte-ciel'' sur $X$). 

On déduit des lemmes précédents et de la suite exacte (\ref{eq:2}) 
les égalités:
\[
  H^q(X,\dG_m) = 0 \text{ pour $q\geqslant 2$,}
\]
et une suite exacte de cohomologie en bas degré:
\[
  1 \to H^0(X,\dG_m) \to H^0(X, j_*\dG_{m,\eta}) \to H^0\left(X,\bigoplus_{x\in X} i_{x*}\dZ_x\right) \to H^1(X,\dG_m) \to 1
\]
qui n'est autre que la suite exacte:
\[
  1 \to k^\times \to k(X)^\times\to \Div(X) \to \pic(X) \to 1\text{.}
\]
De la proposition (\ref{3-3-1}) on déduit que les groupes de cohomologie de 
$X$ à valeur dans $\dZ/n$, $n$ premier à la caractéristique de $k$, ont 
une valeur raisonnable:





\begin{corollary}\label{3-3-5}
Si $X$ est de genre $g$ et si $n$ est inversible dans $k$, les 
$H^q(X,\dZ/n)$ sont nuls pour $q>2$, et libres sur $\dZ/n$ de rang $1,2 g,1$ 
pour $q=0,1,2$. Remplaçant $\dZ/n$ par le groupe isomorphe $\dmu_n$, on a des 
isomorphismes canoniques 
\begin{align*}
  H^0(X,\dmu_n) &= \dmu_n \\
  H^1(X,\dmu_n) &= \pic^0(X)_n \\
  H^2(X,\dmu_n) &= \dZ/n \text{.}
\end{align*}
\end{corollary}

Comme le corps $k$ est algébriquement clos, $\dZ/n$ est isomorphe (non 
canoniquement) à $\mu_n$. De la suite exacte de Kummer:
\[
  0 \to \mu_n \to \dG_m \to \dG_m \to 0\text{,}
\]
et de la proposition (\ref{3-3-1}), on déduit les égalité:
\[
  H^q(X,\dZ/n) = 0 \text{ pour $q>2$,}
\]
et, en bas degré, des suites exactes:
\[\xymatrix{
  0 \ar[r] 
    & H^0(X,\dmu_n) \ar[r] 
    & k^\times \ar[r]^-n 
    & k^\times \ar[r] 
    & 0
}\]
\[\xymatrix{
  0 \ar[r] 
    & H^1(X,\dmu_n) \ar[r] 
    & \pic(X) \ar[r]^-n 
    & H^2(X,\dmu_n) \ar[r] 
    & 0\text{.}
}\]
De plus on a une suite exacte:
\[\xymatrix{
  0 \ar[r] 
    & \pic^0(X) \ar[r] 
    & \pic(X) \ar[r]^-{\deg}
    & \dZ \ar[r] 
    & 0 \text{,}
}\]
et $\pic^0(X)$ s'identifie au groupe des points rationnels sur $k$ d'une 
variété abélienne de dimension $g$, la jacobienne de $X$. Dans un tel 
groupe, la multiplication par $n$ est surjective et son noyau est un 
$\dZ/n\dZ$-module libre de rang $2 g$ (car $n$ est inversible dans $k$); d'où 
le corollaire. 

De dévissage astucieux, utilisant la ``méthode de la trace,'' permet 
d'obtenir en corollaire la 





\begin{proposition}[{\cite[IX 5.7]{4}}]\label{3-3-6}
Soient $k$ un corps algébriquement clos, $X$ une courbe algébrique sur $k$ 
et $F$ un faisceau de torsion sur $X$. Alors:
\begin{enumerate}[\indent i)]
  \item On a $H^q(X,F) = 0$ pour $q>2$.
  \item Si $X$ est affine, on a même $H^q(X,F) = 0$ pour $q>1$.
\end{enumerate}
\end{proposition}

Pour la démonstration, ainsi que pour l'exposé de la ``méthode de la 
trace,'' nous renvoyons à \cite[IX 5]{4}.










\subsection{Dévissages}\label{3-4}

Pour calculer la cohomologie des variétés de dimension $>1$ on emploie des 
fibrations par des courbes, ce qui permet de se ramener à étudier les 
morphismes dont les fibres sont de dimension $\leqslant 1$. Ce principe 
possède plusieurs variantes, indiquons-en quelques-unes.





\subsubsection{}\label{3-4-1}

Soient $A$ une $k$-alg\`ebre de type fini et $a_1,\dotsc,a_n$ des générateurs de 
$A$. Si l'on pose $X_0=\spec(k)$, $X_i=\spec(k[a_1,\dotsc,a_i])$, $X_n=\spec(A)$, les inclusions canoniques 
$k[a_1,\dotsc,a_i]\to k[a_1,\dotsc,a_i,a_{i+1}]$ d\'efinissent des morphismes 
$X_n \to X_{n-1} \to \cdots \to X_1 \to X_0$ dont les fibres sont de dimension 
$\leqslant 1$. 





\subsubsection{}\label{3-4-2}

Dans le cas d'un morphisme lisse, on peut être plus pr\'ecis. On appelle 
\emph{fibration \'el\'ementaire} un morphisme de sch\'emas $f:X\to S$ qui peut 
\^etre plong\'e dans un diagramme commutatif
\[\xymatrix{
  X \ar@{^{(}->}[r]^-j \ar[dr]_-f
    & \bar X \ar[d]^-{\bar f}
    & \ar@{_{(}->}[l]_-i \ar[dl]^-g Y \\
  & S
}\]
satisfaisant aux conditions suivantes:
\begin{enumerate}[\indent i)]
  \item $j$ est une immersion ouverte dense dans chaque fibre et 
    $X=\bar X \setminus Y$. 
  \item $\bar f$ est lisse et projectif, \`a fibres g\'eom\'etriques 
    irr\'eductibles et de dimension $1$.
  \item $g$ est un rev\^etement \'etale et aucune fibre de $g$ n'est vide.
\end{enumerate}

On appelle \emph{bon voisinage} relatif \`a $S$ un $S$-sch\'ema $X$ tel qu'il 
existe $S$-sch\'emas $X=X_n,\dotsc,X_0=S$ et des fibrations \'el\'ementaires 
$f_i:X_i\to X_{i-1}$, $i=1,\dotsc,n$. On peut montrer \cite[XI, 3.3]{4} que si 
$X$ est un sch\'ema lisse sur un corps alg\'ebriquement clos $k$ tout point 
rationnel de $X$ poss\`ede un voisinage ouvert qui est un bon voisinage 
(relatif \`a $\spec(K)$). 





\subsubsection{}\label{3-4-3}

On peut d\'evisser un morphisme propre $f:X\to S$ de la façon suivante. 
D'apr\`es le lemme de Chow, il existe un diagramme commutatif
\[\xymatrix{
  X \ar[dr]_-f 
    & \ar[l]_-\pi \ar[d]^-{\bar f} \bar X \\
  & S
}\]
o\`u $\pi$ et $\bar f$ sont des morphismes projectifs, $\pi$ \'etant de plus un 
isomorphisme au-dessus d'un ouvert dense de $X$. Localement sur $S$, $\bar X$ 
est un sous-sch\'ema ferm\'e d'un espace projectif type $\dP_S^n$. 

On d\'evisse ce dernier en consid\'erant la projection 
$\varphi:\dP_S^n\to\dP_S^1$ qui envoie le point de coordonnées homog\`enes 
$(x_0,x_1,\dotsc,x_n)$ sur $(x_0,x_1)$. C'est une application rationnelle 
d\'efinie en dehors du ferm\'e $Y\simeq \dP_S^{n-2}$ de $\dP_S^n$ d'\'equations 
homog\`enes $x_0=x_1=0$. Soit $u:P\to \dP_S^n$ l'\'eclatement \`a centre 
$Y$; les fibres de $u$ soit de dimension $\leqslant 1$. De plus il existe un 
morphisme naturel $v:P\to \dP_S^1$ qui prolonge l'application rationnelle 
$\varphi$ et $v$ fait de $P$ un $\dP_S^1$-sch\'ema localement isomorphe \`a 
l'espace projectif type $\dP^{n-1}$ que l'on peut \`a son tour projeter sur un 
$\dP^1$, etc. 





\subsubsection{}\label{3-4-4}

On peut balayer une vari\'et\'e projective et lisse $X$ par un \emph{pinceau de 
Lefschetz}. L'\'eclat\'e $\widetilde X$ de l'intersection de l'axe du pinceau 
avec $X$ se projette sur $\dP^1$ et les fibres de cette projection sont les 
sections hyperplans de $X$ par les hyperplans du pinceau.




















\section{Théorème de changement de base pour un morphisme propre}\label{4}










\subsection{Introduction}\label{4-1}

Ce chapitre est consacr\'e \`a la d\'emonstration et aux applications du 





\begin{theorem}\label{4-1-1}
Soient $f:X\to S$ un morphisme propre de sch\'emas et $F$ un faisceau abélien 
de torsion sur $X$. Alors, quel que soit $q\geqslant 0$, la fibre de 
$R^q f_* F$ en un point g\'eom\'etrique $s$ de $S$ est isomorphe \`a la 
cohomologie $H^q(X_s,F)$ de la fibre $X_s=X\otimes_S \spec k(s)$ de $f$ en 
$s$. 
\end{theorem}

Pour $f:X\to S$ une application continue propre et s\'epar\'e (s\'epar\'ee 
signifie que la diagonale de $X\times_S X$ est ferm\'ee) entre espaces 
topologiques, et $F$ un faisceau ab\'elien sur $X$, le r\'esultat analogue est 
bien connu, et \'el\'ementaire: comme $f$ est ferm\'ee, les $f^{-1}(V)$ pour 
$V$ voisinage de $s$ formet un syst\`eme fondamental de voisinages de $X_s$, et 
on v\'erifie que $H^*(X_s,F) = \varinjlim_U H^*(U,F)$, pour $U$ parcourant les 
voisinages de $X_s$. En pratique, $X_s$ a m\^eme un syst\`eme fondamental 
$\cU$ de voisinages $U$ dont il est r\'etracte par d\'eformation et, pour $F$ 
constant, on a donc $H^*(X_s,F)=H^*(U,F)$. En termes imag\'es: la fibre 
sp\'eciale avale la fibre g\'en\'erale. 










\begin{thebibliography}{ab}
  \bibitem{Ar} M. Artin, \emph{Grothendieck topologies}, (Harvard University 1962).

  \bibitem{Gi} J. Giraud, \emph{Analysis situs}, Séminaire Bourbaki 256, mai 1963 -- Benjamin. 

  \bibitem{4} M. Artin, A. Grothendieck, J.L. Verdier. \emph{Théorie des Topos et Cohomologie Étale des Schémas}, SGA 4, Springer nos.269,270 et 305, 1972/73.

  \bibitem{7} A. Grothendieck, \emph{Revêtements Etales et Groupe Fondamental}, SGA 1, Springer no.224, 1971.

  \bibitem{11} M. Raynaud, \emph{Anneaux Locaux Henséliens}, Springer no.169, 1970. 

  \bibitem{14} J.-P. Serre, \emph{Cohomologie galoisienne}, Springer no.5, 1965.

\end{thebibliography}





\end{document}
