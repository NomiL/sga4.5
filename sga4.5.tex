\documentclass[oneside]{book} % add option draft to highlight overful hbox
\usepackage{sga-style}





\title{Séminaire de Géométrie Algébrique du Bois-Marie \\ \vspace{20pt}
Cohomologie Étale \\ \vspace{20pt}
(SGA $4\frac 1 2$)}
\author{par P. Deligne \vspace{10pt}\\ 
avec la collaboration de J.F. Boutot, \\ A. Grothendieck, L. Illusie et J.L. Verdier}
\date{1977}


%% the part I'm currently working on 
%  \includeonly{sga4.5-ch6}
% NOTE: could go from \setminus to \smallsetminus, not clear which looks better


\begin{document}
\pagenumbering{alph} % title page # will be a
\maketitle

% French characters
% à é è ê ô ù û

% other notes: there are two I.6.5's in Deligne's original - one defining sheaf 
% cohomology, the other defining torsors






\pagenumbering{roman} % intro will be i, ii, ...
\section*{Typesetters note}

This is a \LaTeX rendition of Deligne's \emph{Cohomologie \'Etale}, Lecture 
Notes in Mathematics, 569, Springer-Verlag, Berlin-Heidelberg-New York. The 
typesetting was done by Daniel Miller. The source code may be found at 
GitHub (\url{https://www.github.com/dkmiller/sga4.5}). It is an essentially 
unmodified version of the original. The only changes are slight, e.g. using 
script instead of roman letters for sheaves. Any comments or corrections should 
be sent to \url{dkmiller@math.cornell.edu}.
% !TEX root = sga4.5.tex

\section*{Introduction}

Ce volume a pour but de faciliter au non-expert l'usage de la 
cohomologie $\ell$-adique. J'espère qu'il lui permettra souvent d'éviter le 
recours aux exposés touffus de SGA 4 et SGA 5. Il contient aussi quelques 
résultats nouveaux. 

Le premier exposé, rédigé par J.F. Boutot, survole SGA 4. Il donne les 
principaux résultats -- avec une généralité minimale, souvent 
insuffisante pour les applications -- et une idée de leur démonstration. Pour 
des résultats complets, ou des démonstrations détaillées, SGA 4 reste 
indispensable. 

Le ``\nameref{II}'' contient une démonstration complété de 
la formule des traces pour l'endomorphisme de Frobenius. La démonstration est 
celle donnée par Grothendieck dans SGA 5, élaguée de tout détail inutile. Ce 
rapport devrait permettre à utilisateur d'oublier SGA 5, qu'on pourra 
considérer comme une série de digression, certaines très intéressantes. Son 
existence permettra de publier prochainement SGA 5 tel quel. Il est complété 
par l'exposé ``\nameref{VI}'' qui explique comment la formule des traces permet 
l'étude de sommes trigonométriques, et donne des exemples. 

Le public visé par les autres exposés est plus limité, et leur style s'en 
ressent. L'exposé ``\nameref{III}'' est une 
généralisation ``modulaire'' du Rapport, basée sur l'étude SGA 4 XVII 5.5 des 
puissances symétriques. L'exposé ``\nameref{IV}'' 
définit cette classes dans divers contextes, et donne la compatibilité 
entre intersections et cup-produits. Dans ``\nameref{V}'' sont rassemblés 
quelques résultats connus, pour lesquels manquait une référence, et quelques 
compatibilités. L'exposé ``\nameref{VII}'' 
est nouveau. Il donne notamment, en cohomologie sans supports, des théorèmes de 
finitude analogues à ceux connus en cohomologie à supports compacts. 

Pour plus de détails sur les exposés, je renvoie à leur introduction 
respective. 

Je remercie enfin J.L. Verdier de m'avoir permis de reproduire ici ses notes 
``\nameref{VIII}.'' Elles restent je crois très utiles, et 
étaien: devenues introuvables. 

Dans les références internes à ce volume, les exposés sont cités par 
un titre abrégé, indiqué entre [ ] dans la table des matières. 

\vspace{20pt}
Bures-sur-Yvette, le 20 Septembre 1976

\vspace{5pt}
Pierre Deligne
\tableofcontents
\pagenumbering{arabic} % rest of pages are numbered the usual way
\include{sga4.5-ch1}
\include{sga4.5-ch2}
\include{sga4.5-ch3}
\include{sga4.5-ch4}
\include{sga4.5-ch5}
% !TEX root = sga4.5.tex

\chapter{Applications de la formule des traces aux sommes trigonométriques}\label{VI}
\chaptermark{Applications aux sommes trigonom\'etriques}










Dans cet expos\'e, j'explique comment la formule des traces permet de calculer 
ou d'\'etudier diverses sommes trigonom\'etriques et comment, jointe \`a la 
conjecture de Weil, elle peut permettre de les majorer.

Les deux premiers paragraphes donnent un ``mode d'emploi'' de ces outils. Le 
paragraphe 3 est un espos\'e, dans un langage cohomologique, des r\'esultats de 
Weil sur les sommes \`a $1$ variable. Les paragraphes $4$ \`a $6$ forment une 
\'etude d\'etaill\'ee des sommes de Gauss et de Jacobi -- y inclus les 
r\'esultats anciens et r\'ecents de Weil sur les caract\`eres de Hecke 
d\'efinis par des sommes de Jacobi. Au paragraphe $7$, nous \'etudions une 
g\'en\'eralisation \`a plusiers variables des sommes de Kloosterman. Enfin, au 
paragraphe $8$, on trouvera quelques indications sur d'autres usages qui ont 
\'et\'e faits on peuvent \^etre faits de ces m\'ethodes. 










\section*{Notations}\label{VI:0}





\subsection{}\label{VI:0-1}

On utilise les notations de \hyperref[II]{Rapport}, paragraphe \ref{II:1}. On 
aura souvent \`a consid\'erer une extension finie $\dF_{q^n}\subset \dF$ de 
$\dF_q$. On notera par un indice $0$ un objet sur $\dF_q$, et par un indice $1$ 
un objet sur $\dF_{q^n}$. Remplacer un indice $0$ par un indice $1$ (resp. 
supprimer l'incide) signifie qu'on \'etend les scalaires \`a $\dF_{q^n}$ (resp. 
\`a $\dF$).





\subsection{}\label{VI:0-2}

On d\'esigne par $\ell$ un nombre premier $\ne p$. Nous utilisons librement le 
language des $\dQ_\ell$-faisceaux, ainsii que celui des $E_\lambda$-faisceaux, 
pour $E_\lambda$ une extension finie de $\dQ_\ell$ (cf. \hyperref[II]{Rapport}, 
paragraphe \ref{II:2} et sp\'ecialement \ref{II:2-11}). 





\subsection{}\label{VI:0-3}

Soit $H^\bullet$ un espace vectoriel gradu\'e. Si $T$ est un endomorphisme de 
$H^\bullet$, on pose (cf. \hyperref[IV]{Cycle}, \ref{IV:eq:1-3-5}) 
\[
  \tr(T,H^\bullet) = \sum (-1)^i \tr(T,H^i) \text{.}
\]










\section{Principes}\label{VI:1}





\subsection{}\label{VI:1-1}

Soient $X_0$ un sch\'ema s\'epar\'e de type fini sur $\dF_q$, $E_\lambda$ une 
extension finie de $\dQ_\ell$ et $\sF_0$ un $E_\lambda$-faisceau sur $X-0$. La 
formule des traces dit que 
\begin{equation*}\tag{1.1.1}\label{VI:eq:1-1-1}
  \sum_{x\in X^F} \tr(F_x^\ast,\sF) = \sum (-1)^i \tr\left(F^\ast,\h_c^i(X,\sF)\right) \text{.}
\end{equation*}

Avec la notation \ref{VI:0-3}, le membre de droite s'\'ecrit simplement 
$\tr(F^\ast,\h_c^\bullet(X,\sF))$. 

Cette formule des traces pour les $E_\lambda$-faisceaux peut soit \^etre 
d\'eduite de \hyperref[II]{Rapport} \ref{II:4-10} par passage \`a limite (cf. 
\hyperref[II]{Rapport} \ref{II:4-11} \`a \ref{II:4-13}), soit \^etre d\'eduite 
de la formule des traces pour les $\dQ_\ell$-faisceaux (\hyperref[II]{Rapport} 
\ref{II:3-2}) par la m\'ethode de (\hyperref[III]{Functions $L$ mod. $\ell^n$}, 
\ref{III:4-3}). 

Nous allons interpr\'eter diverses sommes trigonom\'etriques comme le membre de 
gauche de \eqref{VI:eq:1-1-1}, pour $X_0$ et $\sF_0$ convenables. 





% NOTE: in this section, following the source (but not convention in the 
% LaTeX version, torsors are denoted in roman (not script) letters
\subsection{}\label{VI:1-2}

Soit $A$ un groupe fini. Un \emph{$A$-torseur} sur un sch\'ema $X$ est un 
faisceau $T$ sur $X$, muni d'une action \`a droite de $A$, qui, localement 
(pour la topologie \'etale) sur $X$, est isomorphe au faisceau constant $A$ sur 
lequel $A$ agit par translation \`a droite. 

Si $\tau:A\to B$ est un homomorphisme, et $T$ un $A$-torseur, il existe \`a 
isomorphisme unique pr\`es un et un seul $B$-torseur $\tau(T)$, muni d'un 
morphisme de faisceaux $\tau:T\to \tau(T)$ tel que 
\begin{equation*}\tag{1.2.1}\label{VI:eq:1-2-1}
  \tau(t a) = \tau(t) \tau(a)\text{.}
\end{equation*}

Si $T$ est un $A$-torseur sur $X$, et $\rho$ une repr\'esentation lin\'eaire 
de $A$: $\rho:A\to \gl(V)$ ($V$ un espace vectoriel de dimension finie sur 
$E_\lambda$), il existe \`a isomorphisme unique pr\`es un seul 
$E_\lambda$-faisceau $\sF$, lisse de rang $\dim(V)$, muni d'un morphisme de 
faisceaux 
\begin{equation*}\tag{1.2.2}\label{VI:eq:1-2-2}
  \rho:T\to \underline{\operatorname{Isom}}(V,\sF) \text{.}
\end{equation*}
tel que $\rho(t a) = \rho(t) \rho(a)$. On le note $\rho(T)$. Pour 
$\tau:A\to B$ et $\rho$ une repr\'esentation de $B$, on a un isomorphisme 
canonique $\rho(\tau(T)) = (\rho\tau)(T)$. 

Dans ce paragraphe (sauf l'appendice), on ne consid\`ere que le cas o\`u $A$ 
est commutatif et o\`u $V=E_\lambda$: $\rho$ est un caract\`ere 
$A\to E_\lambda^\times$, et $\rho(T)$ un $E_\lambda$-faisceau lisse de rang 
un. Le morphisme \eqref{VI:eq:1-2-2} s'interpr\`ete comme un morphisme partout 
non nul 
\begin{equation*}\tag{1.2.3}\label{VI:eq:1-2-3}
  \rho:T\to \rho(T) 
\end{equation*}
tel que $\rho(t a) = \rho(a)\rho(t)$. 





\subsection{}\label{VI:1-3}

Soient $S$ un sch\'ema, $G$ un sch\'ema en groupe commutatif sur $S$, et $G'$ 
une extension de sch\'emas en groupes commutatifs sur $S$
\[\xymatrix{
  0 \ar[r] 
    & A \ar[r] 
    & G' \ar[r]^-\pi 
    & G \ar[r] 
    & 0 \text{.}
}\]
Le faisceau $T$ des sections locales de $\pi$ est alors un $A$-torseur sur 
$G$. 

Si $X$ est un sch\'ema sur $S$, et $f,g$ deux $S$-morphismes de $X$ dans $G$, 
du fait que $T$ est d\'efini par une extension, on a un isomorphisme 
canonique 
\begin{equation*}\tag{1.3.1}\label{VI:eq:1-3-1}
  (f+g)^\ast T = f^\ast T + g^\ast T \text{.}
\end{equation*}
A gauche, $f+g$ est la somme de $f$ et $g$, au sens de la loi de groupe $G$: 
\`a droite, $+$ d\'esigne une somme de torseurs. 

Si $f:X\to G$ se factorise par $G'$, $f^\ast T$ est trivial (et un 
factorisation donne une trivialisation); \`a isomorphisme (non unique) pr\`es, 
le $A$-torseur $f^\ast T$ ne d\'epend que de l'image de $f$ dans 
$\hom_S(X,G)/\pi\hom_S(X,G')$: il est donn\'e par le morphisme $\partial$ dans 
la suite exacte 
\[\xymatrix{
  \hom(X,G') \ar[r] 
    & \hom(X,G) \ar[r]^-\partial 
    & \h^1(X,A) \text{.}
}\]

Nous appliquerons ces constructions aux suites exactes de Kummer 
$0 \to \dmu_n \to \dG_m \to \dG_m \to 0$, et aux torseurs de Lang. 





\subsection{Le torseur de Lang}\label{VI:1-4}

Soit $G_0$ un groupe alg\'ebrique commutatif connexe sur $\dF_q$ (pour le cas 
non commutatif, voir l'appendice). La loi de groupe est not\'ee 
\emph{multiplicativement}. L'isog\'enie de Lang 
\[
\fL:G_0 \to G_0: x\mapsto F x\cdot x^{-1}
\]
est \'etale; son image, un sous-groupe ouvert de $G_0$, ne peut \^etre que 
$G_0$ lui-m\^eme; son noyau est le groupe fini $G_0(\dF_q)$. Le \emph{torseur 
de Lang} $L$ est le $G_0(\dF_q)$-torseur sur $G_0$ d\'efini par la suite 
exacte 
\begin{equation*}\tag{1.4.1}\label{VI:eq:1-4-1}
\xymatrix{
  0 \ar[r] 
    & G_0(\dF_q) \ar[r] 
    & G_0 \ar[r]^-\fL 
    & G_0 \ar[r] 
    & 0 \text{.}
}
\end{equation*}


\paragraph{Notation}
On note encore $L$ et $\fL$ (ou $L_{(q)}$ et $\fL_{(q)}$) les objects d\'eduits 
de $L$ et $\fL$ par extension du corps de base. En cas de besoin, on 
pr\'ecisera par un indice $0$ ou $1$.


\paragraph{Exemples}
Pour $G_0=\dG_a$ ou $\dG_m$, les suites \eqref{VI:eq:1-4-1} s'\'ecrivent 
\begin{align*}\tag{1.4.2}\label{VI:eq:1-4-2}
&\xymatrix{
  0 \ar[r] 
    & \dF_q \ar[r] 
    & \dG_a \ar[r]^-{x^q-x} 
    & \dG_a \ar[r] 
    & 0 
} \\
\tag{1.4.3}\label{VI:eq:1-4-3}
&\xymatrix{
  0 \ar[r] 
    & \dmu_{q-1} \ar[r] 
    & \dG_m \ar[r]^-{x^{q-1}} 
    & \dG_m \ar[r] 
    & 0
}
\end{align*}





\subsection{}\label{VI:1-5}

Calculons l'endomorphisme $F^\ast$ de la fibre du torseur de Lang en 
$\gamma\in G^F=G_0(\dF_q)$. Si $g\in \fL^{-1}(\gamma)\subset G$, on a 
$F g = F g\cdot g^{-1}\cdot g = \gamma g$. D\`es lors (\hyperref[II]{Rapport}, 
\ref{II:1-2}) 
\begin{equation*}\tag{1.5.1}\label{VI:eq:1-5-1}
  \text{sur $L_0(G_0)_\gamma\simeq \fL^{-1}(\gamma)$, $F^\ast$ est $g\mapsto g\gamma^{-1}$.}
\end{equation*}





\subsection{}\label{VI:1-6}

Sur $\dF_{q^n}$, l'identit\'e $F_{(q^n)}=F_{(q)}^n$ entre endomorphismes de 
$G_1$ implique que $\fL_{(q^n)}=\fL_{(q)}\circ \prod_{i=0}^{n-1} F_{(q)}^i$. 
Sur $G_0(\dF_{q^n})$, $F_{(q)}^i$ agit comme l'\'el\'ement 
$x\mapsto x^{q^i}$ de $\gal(\dF_{q^n}/\dF_q)$. Sur $G_0(\dF_{q^n})$, 
$\prod F_{(q)}^i$ est donc le compos\'e de la norme 
$G_0(\dF_{q^n})\to G_0(\dF_q)$ et de l'inclusion 
$G_0(\dF_q)\subset G_0(\dF_{q^n})$: le diagramme 
\begin{equation*}\tag{1.6.1}\label{VI:eq:1-6-1}
\xymatrix{
  0 \ar[r] 
    & G_0(\dF_{q^n}) \ar[r] \ar[d]^-N 
    & G_1 \ar[r]^-{\fL_{(q^n)}} \ar[d]^-{\prod_{i=0}^{n-1} F_{(q)}^i} 
    & G_1 \ar[r] \ar@{=}[d] 
    & 0 \\
  0 \ar[r] 
    & G_0(\dF_q) \ar[r] 
    & G_q \ar[r]^-{\fL_{(q)}} 
    & G_1 \ar[r] 
    & 0
}
\end{equation*}
est commutatif. Dans le langage des torseurs, ceci fournit un isomorphisme 
canonique entre $G_0(\dF_q)$-torseurs sur $G_1$ 
\begin{equation*}\tag{1.6.2}\label{VI:eq:1-6-2}
  N L_{(q^n)} = L_{(q)} \text{.}
\end{equation*}





\begin{definition_}\label{VI:1-7}
Soient $f_0:X_0\to G_0$ un morphisme et 
$\chi:G_0(\dF_q)\to E_\lambda^\times$ un caract\`ere. On pose 
$\sF(\chi,f_0) = \chi^{-1}(f_0^\ast (L_0(G_0))) = f_0^\ast\chi^{-1}(L_0(G_0))$. 
\end{definition_}

On a les propri\'et\'es de fonctorialit\'e suivantes:
\begin{enumerate}[a)]
  \item $\sF(\chi,f_0)$ est bimultiplicatif en $\chi$ et $f_0$: 
    \begin{align*}\tag{1.7.1}\label{VI:eq:1-7-1}
      \sF(\chi,f_0'\cdot f_0'') &= \sF(\chi,f_0')\otimes \sF(\chi,f_0'') \text{,} \\
      \tag{1.7.2}\label{VI:eq:1-7-2}
      \sF(\chi'\chi'',\sF_0) &= \sF(\chi',f_0)\otimes \sF(\chi'',f_0) \text{.}
    \end{align*}
    Cela r\'esulte de la d\'efinition de $\chi(T)$, pour $T$ un torseur, joint 
    \`a \eqref{VI:eq:1-3-1} pour \eqref{VI:eq:1-7-1}. 
  \item Pour $g_0:Y_0\to X_0$ un morphisme, on a 
    \begin{equation*}\tag{1.7.3}\label{VI:eq:1-7-3}
      \sF(\chi,f_0\circ g_0) = g_0^\ast \sF(\chi,f_0) \text{.}
    \end{equation*}
    En particulier, $\sF(\chi,f_0) = f_0^\ast \sF(\chi)$, o\`u $\sF(\chi)$ 
    d\'esigne $\sF(\chi,\operatorname{id}_{G_0})$, 
  \item Pour $u_0:G_0\to H_0$ un morphisme, et 
    $\chi:H_0(\dF_q) \to E_\lambda^\times$, on a 
    \begin{equation*}\tag{1.7.4}\label{VI:eq:1-7-4}
      \sF(\chi,u_0 f_0) = \sF(\chi u_0,f_0) \text{.}
    \end{equation*}
  \item Pour $G_0=\prod_{i\in I} G_0^i$, $\chi$ de coordonn\'ees $\chi_i$ 
    ($i\in I$) et $f_0$ de coordonn\'ees $f_0^i$, il r\'esulte de 
    \eqref{VI:eq:1-7-1} \`a \eqref{VI:eq:1-7-4} que 
    \begin{equation*}\tag{1.7.5}\label{VI:eq:1-7-5}
      \sF(\chi,f_0) = \bigotimes_{i\in I} \sF(\chi_i,f_0^i) \text{.}
    \end{equation*}    
\end{enumerate}

Noter le $\chi^{-1}$ dans la d\'efinition \ref{VI:1-7}. Il assure que, pour 
$x\in X^F$, on ait sur $\sF(\chi,f_0)_x$ 
\begin{equation*}\tag{1.7.6}\label{VI:eq:1-7-6}
  F_x^\ast = \chi f_0(x)
\end{equation*}
(utiliser que la fibre en $x$ du morphisme \eqref{VI:eq:1-2-2}, pour 
$\rho=\chi^{-1}$, commute \`a $F_x^\ast$, et \eqref{VI:eq:1-5-1}). 

Si $\sF(\chi,f_0)_1$ est le faisceau d\'eduit de $\sF(\chi,f_0)$ par extension 
du corps de base \`a $\dF_{q^n}$, on d\'eduit de \eqref{VI:eq:1-6-2} que 
\begin{equation*}\tag{1.7.7}\label{VI:eq:1-7-7}
  \sF(\chi,f_0)_1 = \sF(\chi\circ N,f_1) \text{.}
\end{equation*}





\subsection{Abus de notations}\label{VI:1-8}

(i) Si $\Xi$ est une notation pour l'application compos\'ee 
$\chi f_0:X_0(\dF_q) \to E_\lambda^\times$, on \'ecrira parfois $\sF(\Xi)$ au 
lieu de $\sF(\chi,f_0)$. Grâce \`a \eqref{VI:eq:1-7-1} \`a \eqref{VI:eq:1-7-5}, 
on ne risque gu\`ere d'ambiguit\'e. Par exemple:
\begin{enumerate}[a)]
  \item on \'ecrit $\sF(\chi)$ pour $\sF(\chi,\operatorname{id}_{G_0})$ 
    (notations d\'ej\`a utilis\'ee en \ref{VI:1-7}b)); 
  \item en \'ecrit $\sF(\chi f_0)$ pour $\sF(\chi,f_0)$;
  \item avec les notations de \ref{VI:1-7}d), on \'ecrit 
    $\sF(\prod \chi_i f_0^i)$ pour $\sF(\chi,f))$. 
\end{enumerate}

Avec cette notation, \eqref{VI:eq:1-7-4} exprime que l'\'ecriture 
$\sF(\chi u_0 f_0)$ n'est pas ambiguë, \eqref{VI:eq:1-7-1} \eqref{VI:eq:1-7-2} 
\eqref{VI:eq:1-7-5} expriment une multiplicativit\'e de $\sF(\Xi)$ en $\Xi$, 
et \eqref{VI:eq:1-7-6} se r\'ecrit $F_x^\ast = \Xi(x)$ sur $\sF(\Xi)_x$.

(ii) Si $X$ est un sch\'ema sur une extension $k$ de $\dF_q$, et que $f$ est un 
morphisme de $X$ dans $G_0\otimes_{\dF_q} k$, on notera encore $\sF(\chi,f)$, 
$\sF(\chi f)$, ou $\sF(\chi)$ (pour $f$ une inclusion) l'image r\'eciproque par 
$f$ du $E_\lambda$-faisceau d\'eduit de $\chi^{-1}(L_0(G))$ par extension des 
scalaires de $\dF_q$ \`a $k$. Si $f_0$ est le compos\'e 
$X\to G_0\otimes_{\dF_q} k \to G_0$, c'est encore le $\sF(\chi,f_0)$ de 
\ref{VI:1-7}. Pour $k$ est une extension finie de $\dF_q$, on a 
$\sF(\chi f) = \chi(\chi N_{k/\dF_q} f)$. 

Appliquant \eqref{VI:eq:1-1-1} \`a \eqref{VI:eq:1-7-6} et \eqref{VI:eq:1-7-7}, 
on trouve: 





% NOTE: originally a 'scholium'
\begin{theorem_}\label{VI:1-9}
Soient $S_0$ un sch\'ema s\'epar\'e de type fini sur $\dF_q$, $G_0$ un groupe 
alg\'ebrique commutatif connexe sur $\dF_q$, $f_0:S_0\to G_0$ un morphisme et 
$\chi:G_0(\dF_q) \to E_\lambda^\times$. On a 
\begin{equation*}\tag{1.9.1}\label{VI:eq:1-9-1}
  \sum_{s\in S_0(\dF_q)} \chi f_0(s) = \tr(F^\ast,\h_c^\bullet(S,\sF(\chi,f_0))) 
\end{equation*}
et, pour tout entier $n\geqslant 1$
\begin{equation*}\tag{1.9.2}\label{VI:eq:1-9-2}
  \sum_{s\in S_0(\dF_{q^n})} \chi\circ N_{\dF_{q^n}/\dF_q} f_0(s) = \tr({F^\ast}^n, \h_c^\bullet\left(S,\sF(\chi,f_0))\right) \text{.}
\end{equation*}
\end{theorem_}





\subsubsection{Remarque}\label{VI:1-9-3}

Prenons pour $G_0$ un produit. Avec les notations de \ref{VI:1-7}d) et 
\ref{VI:1-8}c), la formule \eqref{VI:eq:1-9-2} devient 
\[
  \sum_{s\in S_0(\dF_{q^n})} \prod_i \chi_i\circ N_{\dF_{q^n}/\dF_q} (f_0^i(s)) = \tr\Big({F^\ast}^n, \h_c^\bullet\Big(S,\sF\Big(\prod_i \chi_i f_0^i\Big)\Big)\Big) \text{.}
\]





\subsubsection{Remarque}\label{VI:1-9-4}

Prenons $G_0=\{e\}$. La formule \eqref{VI:eq:1-9-1} devient 
\[
  \left|S_0(\dF_q)\right| = \sum_{s\in S_0(\dF_q)} 1 = \tr(F^\ast,\h_c^\bullet(S,\dQ_\ell)) \text{.}
\]

Il est rare qu'on puisse explicitement calculer le membre de droite de 
\eqref{VI:eq:1-9-1}. Voici un exemple amusant, avec $G_0=\{e\}$. 





\subsection{Exemple}\label{VI:1-10}

Une quadrique projective non singuli\`ere de dimension impaire sur \texorpdfstring{$\dF_q$}{Fq} a le m\^eme nombre de points rationnels que l'espace projectif sur \texorpdfstring{$\dF_q$}{Fq} de la m\^eme dimension.

Si, sur $\dC$, $X$ est une hypersurface quadrique non singuli\`ere dans 
l'espace projectif $\dP^{2 N}$, et $Y$ un hyperplan, on sait que pour 
$i\leqslant 2\dim(X)=2\dim(Y)$, les inclusions 
$X\hookrightarrow \dP^{2 N}\hookleftarrow Y$ induisent des isomorphismes (en 
cohomologie ordinaire) 
\[\xymatrix{
  \h^i(X,\dQ) 
    & \ar[l]_-\sim \h^i(\dP^{2 N},\dQ) \ar[r]^-\sim 
    & \h^i(Y,\dQ) \text{.}
}\]

Par sp\'ecialisation, il en r\'esulte que si $X_0'$ est une hypersurface 
quadrique non singuli\`ere dans l'espace projectif $\dP_0^{2 N}$ sur $\dF_q$, 
et $Y_0'$ un hyperplan, les inclusions 
$X_0'\hookrightarrow \dP_0^{2 N}\hookleftarrow Y_0'$ induisent des 
isomorphismes 
\[\xymatrix{
  \h^i(X',\dQ_\ell) 
    & \ar[l]_-\sim \h^i(\dP^{2 N},\dQ_\ell) \ar[r]^-\sim 
    & \h^i(Y',\dQ_\ell) \text{.}
}\]
Ces isomorphismes commutent \`a $F^\ast$, et on applique la formule des traces. 





\subsection{}\label{VI:1-11}

On peut aussi utiliser la formule des traces pour comprendre les formules 
classiques donnant le nombre de points rationnels des groupes lin\'eaires sur 
les corps finis, et ceux de certains espaces homog\`enes (voir le paragraphe 
8). 





\subsection{}\label{VI:1-12}

On dispose d'un dictionnaire permettant de traduire en termes cohomologiques 
ques divers types de manipulations classiques sur les sommes 
trigonom\'etriques. Ce dictionnaire sera donn\'e au paragraphe 2. L'\'enonc\'e 
cohomologique \'etant plus ``g\'eom\'etrique,'' il pourra parfois s'appliquer 
\`a des situations o\`u l'argument classique ne s'applique qu'apr\`es une 
extension du corps fini de base. Voici un exemple (un cas particulier d'un 
th\'eor\`eme de Kazhdan). 





\begin{theorem_}[Kazhdan]\label{VI:1-13}
Soit $X_0$ un sch\'ema sur $\dF_q$. Supposons qu'il existe une action $\rho$ de 
$\dG_a$ sur $X$, et un morphisme $f:X\to Y$ de sch\'emas sur $\dF$, faisant de 
$X$ un $\dG_a$-torseur sur $Y$. Alors, le nombre de points rationnels de $X_0$ 
est divisible par $q$.
\end{theorem_}

Supposons d'abord que $\rho$, $Y$ et $f$ soient d\'efinis sur $\dF_q$. 


\paragraph{Argument classique}
Les fibres de $f:X_0(\dF_q)\to Y_0(\dF_q)$ ont toutes $q$ \'el\'ements: 
$|X_0(\dF_q)| = q\cdot |Y_0(\dF_q)|$, d'o\`u la divisibilit\'e. 


\paragraph{Traduction cohomologique}
Les faisceaux images directes sup\'erieures $\R^i f_!\dQ_\ell$ sont 
\[
  \R^i f_!\dQ_\ell = \begin{cases}
                       \dQ_\ell(-1) & \text{si $i = 2$} \\
                       0            & \text{pour $i\ne 2$}
                     \end{cases}
\]
La suite spectrale de Leray de $f$ d\'eg\'en\`ere donc en un isomorphisme 
\[
  \h_c^i(X,\dQ_\ell) = \h_c^{i-2}(Y,\dQ_\ell)(-1) \text{.}
\]
Pour comprendre l'effet d'un twist \`a la Tate sur les valeurs propres de 
Frobenius, le plus commode est d'utiliser le point de vue galoisien 
(\hyperref[II]{Rapport} \ref{II:1-8}), et de noter que la substitution de 
Frobenius $\varphi$ agit sur $\dQ_\ell(1)$ par multiplication par $q$. Les 
valeurs propres de $F^\ast$ sur $\h_c^p(X,\dQ_\ell)$ sont donc les produits par 
$q$ des valeurs propres de $F^\ast$ agissant sur $\h_c^{i-2}(Y,\dQ_\ell)$. On 
sait que celles-ci sont des entiers alg\'ebriques \cite[XXI 5.2.2]{sga7}; d\`es 
lors 

\begin{equation*}\tag{$*$}\label{VI:eq:*}
  \substack{\displaystyle\text{les valeurs propres de $F^\ast$ agissant sur $\h_c^i(X,\dQ_\ell)$ sont} \\
  \displaystyle\text{des entiers alg\'ebriques divisibles par $q$.}}
\end{equation*}


\paragraph{Descente}
Revenons aux hypoth\`eses du th\'eor\`eme. Pour $n$ convenable, on peut 
supposer que $\rho$, $Y$ et $f$ sont d\'efinis sur $\dF_{q^n}$. Le morphisme de 
Frobenius relatif \`a $\dF_{q^n}$ \'etant la puissance $n$-i\`eme de celui 
relatif \`a $\dF_q$, l'\'enonc\'e \eqref{VI:eq:*} pour le sch\'ema sur 
$\dF_{q^n}$ d\'eduit de $X_0$ par extension des scalaires nous fournit: 
\begin{equation*}\tag{$**$}\label{VI:eq:**}
  \substack{\displaystyle\text{les puissances $n$-\`iemes des valeurs propres de $F^\ast$ agissant sur} \\ \displaystyle\text{$\h_c^i(X,\dQ_\ell)$ sont des entiers alg\'ebriques divisibles par $q^n$.}}
\end{equation*}

Cette assertion implique \eqref{VI:eq:*}. Le nombre de points rationnels de 
$X_0$, soit $\sum (-1)^i \tr(F^\ast,\h^i(X,\dQ_\ell))$, est donc entier 
alg\'ebrique divisible par $q$. Ceci implique qu'il soit divisible par $q$ en 
tant qu'entier rationnel. 





\subsection{}\label{VI:1-14}

La formule \ref{VI:1-9-3} contr\^ole la d\'ependance en $n$ de la somme 
trigonom\'etrique au membre de gauche. Elle implique des identit\'es entre une 
somme trigonom\'etrique et celles qui s'en d\'eduisennt par ``extension des 
scalaires.'' Le plus c\'el\`ebre de ces identit\'es est celle de 
Hasse-Davenport: soient $\chi:\dF_q^\times \to \dC^\times$ et 
$\psi:\dF_q\to \dC^\times$ des caract\`eres, $\psi$ non trivial, et 
d\'efinissons la somme de Gauss $\tau(\chi,\psi)$ par 
\begin{equation*}\tag{1.14.1}\label{VI:eq:1-14-1}
  \tau(\chi,\psi) = - \sum_{x\in \dF_q^\times} \psi(x) \chi^{-1}(x) \text{.}
\end{equation*}





\begin{theorem_}[Hasse-Davenport]\label{VI:1-15}
On a 
\begin{equation*}\tag{1.15.1}\label{VI:eq:1-15-1}
  \tau\left(\chi\circ N_{\dF_{q^n}/\dF_q},\psi\circ \tr_{\dF_{q^n}/\dF_q}\right) = \tau(\chi,\psi)^n \text{.}
\end{equation*}
\end{theorem_}

Soient $E\subset \dC$ un corps de nombres contenant les valeurs de $\chi$ et 
$\psi$, et $E_\lambda$ le compl\'et\'e de $E$ en une place de caract\'eristique 
$\ell\ne p$. L'identit\'e \eqref{VI:eq:1-15-1} est une identit\'e dans $E$. Il 
revient au m\^eme de la prouver dans $\dC$, ou dans $E_\lambda$. Nous la 
prouverons dans $E_\lambda$, en regardant $\chi$ et $\psi$ comme \`a valeurs 
dans $E_\lambda^\times$. 

On applique \ref{VI:1-9-3} pour $X_0=\dG_m$ sur $\dF_q$ et 
$G_0=\dG_m\times \dG_a$. Les $\h_c^i(\dG_m,\sF(\chi^{-1}\psi))$ sont nuls pour 
$i\ne 1$, et le $\h_c^1$ est de dimension $1$ (\ref{VI:4-2}). D'apr\`es 
\ref{VI:1-9-3}, 
$\tau(\chi\circ N_{\dF_{q^n}/\dF_q},\psi\circ \tr_{\dF_{q^n}/\dF_q})$ est 
l'unique valeur propre de $(F^\ast)^n$ sur $\h_c^1$, et \eqref{VI:eq:1-15-1} en 
r\'esulte. 

Des exemples analogues sont donn\'es dans Weil \cite[App V]{we74}. 





\subsection{}\label{VI:1-16}

Un $E_\lambda$-faisceau $\sF_0$ sur $X-0$ est dit \emph{ponctuellement de 
poids $n$} si pour tout point ferm\'e $x\in |X_0|$, les valeurs propres de 
$F_x^\ast$ (\hyperref[II]{Rapport} \ref{II:1-2}) sont des nombres alg\'ebriques 
dont tous les conjugu\'es complexes sont de valeur absolue 
$q_x^{n/2}$, o\`u $q_x$ est le nombre d'\'el\'ements de $k(x)$. Le th\'eor\`eme 
suivant sera d\'emontr\'e dans \cite{de80}. 





\begin{theorem_}\label{VI:1-17}
Si $\sF_0$ est ponctuellement de poids $n$, pour toute valeur propre $\alpha$ 
de l'endomorphisme $F^\ast$ de $\h_c^i(X,\sF)$, il existe un entier 
$m\leqslant n+i$ tel que les conjugu\'es complexes de $\alpha$ soient tous de 
valeurs absolue $q^{m/2}$.
\end{theorem_}

Les faisceaux consid\'er\'es en \ref{VI:1-9} sont de poids $0$. Le th\'eor\`eme 
fournit donc pour la somme \ref{VI:eq:1-9-1} (on plut\^ot, pour tous ses 
conjugu\'es complexes) la majoration 
\begin{equation*}\tag{1.17.1}\label{VI:eq:1-17-1}
  \left|\sum_{s\in S_0(\dF_q)} \prod_i \chi_i(f^i(s)) \right| \leqslant \sum_i \dim \h_c^i\left(S,\sF\left(\prod \chi_i f^i\right)\right)\cdot q^{i/2} \text{.}
\end{equation*}





\subsection{Remarques}\label{VI:1-18}

\begin{enumerate}[a)]
  \item Si $\dim S=n$, pour tout $E_\lambda$-faisceau $\sF$ sur $S$, on a 
    \[
      \h_c^i(S,\sF) = 0 \qquad \text{pour $i\notin [0,2n]$.}
    \]
  \item Le groupe $\h_c^0(X,\sF)$ est le groupe des sections globales de $\sF$ 
    sur $S$ dont le support est propre. Si $S$ est connexe, non complet et 
    $\sF$ lisse, on bien si $S$ est connexe, $\sF$ lisse de rang un, et non 
    constant, ce groupe est nul.
  \item Si $S$ est lisse purement de dimension $n$, et $\sF$ lisse, la 
    dualit\'e de Poincar\'e dit que les espaces vectoriels 
    $\h_c^i(X,\sF)$ et $\h^{2n-i}(X,\sF^\vee)(2n)$ sont duax l'un de l'autre 
    (pour l'effet d'un twist \`a la Tate sur les valeurs propres de Frobenius, 
    cf. la $2$\`eme partie de la preuve de \ref{VI:1-13}).
  \item Si $\dim S=n$, la dualit\'e de Poincar\'e permet de calculer comme 
    suite $\h_c^{2n}(S,\sF)$: on prend un ouvert $U$ de $S_\text{red}$, dont le 
    compl\'ementaire est de dimension $<n$, lisse purement de dimension $n$, et 
    sur lequel $\sF$ est lisse. On a alors 
    $\h_c^{2n}(U,\sF) \iso \h_c^{2n}(S,\sF)$, car dans la suite exacte longue 
    de cohomologie, $\h_c^i(S\setminus U,\sF) = 0$ pour $i=2n,2n-1$. Par 
    Poincar\'e, on a donc 
    \[
      \h_c^{2n}(S,\sF) \iso \left(\h^0(U,\sF^\vee)(2n)\right)^\vee \text{.}
    \]
    
    Supposons $U$ connexe, et soit $u$ un point g\'eom\'etrique de $U$. Le 
    ``syst\`eme local'' $\sF$ correspond \`a une repr\'esentation de 
    $\pi_1(U,u)$ sur $\sF_u$, et $\h^0(U,\sF^\vee)$ est 
    $(\sF_u^\vee)^{\pi_1(U,u)}=((\sF_u)_{\pi_1(U,u)})^\vee$ (dual des 
    coinvariants). On a donc 
    \[
      \h_c^{2n}(S,\sF) \simeq (\sF_u)_{\pi_1(U,u)}(-2n) \text{.}
    \]
  \item Ces remarques permettent souvent le calcul de $b_0$ et $b_{2n}$. 
    Calculer les autres $b_i$ peut \^etre difficile. Si $b_{2n}=0$ et que 
    $b_{2n-1}=0$, la majoration \eqref{VI:eq:1-17-1} est une majoration \`a la 
    Lang-Weil, avec, pour $q$ grand, un gain en $\sqrt q$ par rapport \`a la 
    majoration triviale en $O(q^n)$. On prendra toutefois garde que la 
    majoration \eqref{VI:eq:1-17-1} n'implique pas formellement la majoration 
    triviale 
    \[
      \left|\sum_{s\in S_0(\dF_q)} \prod_i \chi_i(f^i(s))\right| \leqslant \left| S_0(\dF_q)\right| \text{,}
    \]
    donc il peut \^etre utile de tenir compte (cf. la preuve de \ref{VI:3-8}). 
\end{enumerate}

J'explique ci-dessous une m\'ethode pour prouver que $b_i=0$ pour $i\ne n$. 
Quand elle s'applique, elle fournit une estimation en $O(q^{n/2}$. La constante 
implicate dans $0$ est $b_n=(-1)^n \chi(S,\sF(\prod \chi_i f^i))$ en peut 
\^etre difficile \`a calculer. 





\begin{proposition_}\label{VI:1-19}
Soient $\bar X$ un sch\'ema propre sur un corps alg\'briquement clos $k$, 
$j:X\hookrightarrow bar X$ un ouvert de $X$ et $\sF$ un $E_\lambda$-faisceaux 
sur $X$. On suppose que 
\begin{enumerate}[\indent a)]
  \item $(j_\ast\sF)_x = 0$ pour $x\in \bar X\setminus X$, i.e. 
    $j_!\sF\iso j_\ast\sF$; 
  \item $\R^i j_\ast \sF = 0$ pour $i>0$.
\end{enumerate}
Alors, les applications 
\[
  \h_c^i(X,\sF) \to \h^i(X,\sF)
\]
sont des isomorphismes. 
\end{proposition_}

Les hypoth\`eses a) b) signifiet que $j_! \sF \iso \R j_\ast \sF$, d'o\`u 
\[
  \h_c^i(X,\sF) = \h^i(\bar X,j_!\sF) \iso \hh^i(\bar X,\R j_\ast \sF) = \h^i(X,\sF) \text{.}
\]
Autrement dit, la suite spectrale de Leray pour $j$ 
\[
  E_2^{pq} = \h^p(\bar X,\R^q j_\ast \sF) \Rightarrow \h^{p+q}(X,\sF) 
\]
d\'eg\'en\`ere, par b) en des isomorphismes 
\[
  \h^i(\bar X,j_\ast\sF) \iso \h^i(X,\sF) \text{,}
\]
tandis que, par a) 
\[
  \h_c^i(X,\sF) = \h^i(\bar X,j_!\sF) \iso \h^i(\bar X,j_\ast\sF) \text{.}
\]





\subsubsection{Exemple}\label{VI:1-19-1}

Si $X$ est une courbe, ou si $\bar X$ est lisse, $X$ le compl\'ement d'un 
diviseur \`a croisements normaux et $\sF$ mod\'er\'ement ramifi\'e, 
l'hypoth\`ese a) de \ref{VI:1-19} (pas d'invariants sous la monodromie locale) 
implique l'hypoth\`ese b). 





\begin{proposition_}\label{VI:1-20}
Soient $X_0$ un sch\'ema affine lisse purement de dimension $n$ sur $\dF_q$ et 
$\sF_0$ un $E_\lambda$-faisceau lisse ponctuellement de poids $m$ sur $X_0$. On 
suppose que les applications $\h_c^i(X,\sF) \to \h^i(X,\sF)$ sont des 
isomorphismes (cf. \ref{VI:1-17}). Alors, $\h_c^i(X,\sF) = 0$ pour $i\ne n$, et 
les conjugu\'es complexes des valeurs propres de $F^\ast$ sur $\h_c^n(X,\sF)$ 
sont de valeur absolue $q^{(m+n)/2}$. 
\end{proposition_}

La dualit\'e de Poincar\'e met en dualit\'e $\h_c^i(X,\sF)$ (resp. 
$\h_c^i(X,\sF^\vee)$) avec $\h^{2n-i}(X,\sF^\vee(n))$ (resp. 
$\h^{2n-i}(X,\sF(n))$). D'apr\`es \cite[XIV 3.2]{sga4}, les $\h^i$ sans support 
sont nuls pour $i>n$. Par dualit\'e, les $\h_c^i$ sont nuls pour $i<n$; puisque 
$\h_c^i(X,\sF) \iso \h^i(X,\sF)$, ces groupes sont nuls pour $i\ne n$. 
Appliquons \ref{VI:1-15} \`a $\sF_0$ et \`a $\sF_0^\vee(n)$ (de poinds 
$-m-2n$). On trouve que les conjugu\'es complexes $\alpha$ des valeurs propres 
de $F^\ast$ sur $\h_c^n(X,\sF) = \h_c^n(X,\sF^\vee(n))^\vee$ v\'erifient 
\begin{align*}
  |\alpha| &\leqslant q^{(m+n)/2} && \text{et} \\
  |\alpha^{-1}| &\leqslant q^{((-m-2n)+n)/2} = q^{-(m+n)/2} \text{,}
\end{align*}
d'o\`u l'assertion. 





\subsection{Remarque}\label{VI:1-21}

Le dernier argument montre que si $X_0$ est un sch\'ema s\'epar\'e lisse sur 
$\dF_q$, que $\sF_0$ est un $E_\lambda$-faisceau lisse ponctuellement de poids 
$m$ sur $X_0$ et que $\h_c^i(X,\sF) \hookrightarrow \h^i(X,\sF)$, alors les 
conjugu\'es complexes des valeurs propres de $F^\ast$ sur $\h_c^i(X,\sF)$ sont 
de valeur absolue $q^{(m+i)/2}$. 





\subsection*{Appendice. L'isog\'enie de Lang dans le cas non commutatif}




\subsection{}\label{VI:1-22}

Soient $G-0$ un groupe alg\'ebrique connexe sur $\dF_q$ et $\fL$ le morphisme 
$x\mapsto F x\cdot x^{-1}$ de $G-0$ dans lui-m\^eme. C'est un orphisme de 
$G_0$-espaces homog\`enes, si \`a la source on fait agir $G-0$ par translation 
\`a gauche $x\mapsto (y\mapsto x y)$, au but par 
$x\mapsto (y\mapsto F x\cdot y\cdot x^{-1}$). On a $\fL(xy) = \fL(x)$ pour 
$y\in G_0(\dF_q)$, et $\fL$ induit un isomorphisme $G_0/G_0(\dF_q)\iso G_0$. 

Comme dans le cas commutatif, $\fL$ fait donc de $G_0$ un $G_0(\dF_q)$-torseur 
sur $G_0$, le \emph{torseur de Lang} $L$. Si 
$\rho:G_0(\dF_q) \to \gl(n,E_\lambda)$ est une repr\'esentation lin\'eaire de 
$G_0(\dF_q)$, nous noterons $\sF(\rho)$ le $E_\lambda$-faisceau $\rho(L)$ 
(\ref{VI:1-2}). 





\subsection{}\label{VI:1-23}

Soit $\gamma\in G^F=G_0(\dF_q)$, et calculons l'endomorphisme $F^\ast$ de 
$\sF(\rho)_\gamma$. Soit $g\in G$ tel que $\fL(g)=\gamma$, d'o\`u un rep\`ere 
$\rho(g)\in \operatorname{Isom}(E_\lambda^n,\sF(\rho)_\gamma)$. Pour 
$e\in E_\lambda^n$, on a 
\[
  F(\rho(g)(e)) = \rho(F g)(e) = \rho(\gamma g)(e) = \rho(g\cdot g^{-1}\gamma g)(e) \text{.}
\]
On verra que $g^{-1} \gamma g\in G_0(\dF_q)$, d'o\`u 
\[
  F(\rho(g)(e)) = \rho(g) \rho(g^{-1}\gamma g)(e)
\]
et 
\begin{equation*}\tag{1.23.1}\label{VI:eq:1-23-1}
  \rho(g)^{-1} (F_\gamma^\ast)^{-1} \rho(g) = \rho(g^{-1} \gamma g) \text{:}
\end{equation*}
$\rho(g)$ identifie l'inverse de $F^\ast$ en $\gamma$ \`a l'automorphisme 
$\rho(g^{-1} \gamma g)$ de $E_\lambda^n$. Reste \`a comprendre ce qu'est 
$g^{-1} \gamma g$. 





\begin{lemma_}\label{VI:1-24}
\begin{enumerate}[(i)]
  \item Si $F g\cdot g^{-1} \in G_0(\dF_q)$, on a 
    $g^{-1} (F g\cdot g^{-1}) g = g^{-1} F g\in G_0(\dF_q)$.
  \item Soient $\gamma\in G_0(\dF_q)$, et $g$ tel que 
    $F g\cdot g^{-1} = \gamma$. La classe de conjugaison de $\gamma'=g^{-1} Fg$ 
    dans $G_0(\dF_q)$ ne d\'epend que de celle de $\gamma$. 
  \item L'application induite par $\gamma\mapsto \gamma'$, de l'ensemble 
    $G_0(\dF_q)^\natural$ des classes de conjugaison de $G_0(\dF_q)$ dans 
    lui-m\^eme, est bijective. Son inverse est l'application pour le groupe 
    oppos\'e. 
\end{enumerate}
\end{lemma_}

(i) Si $\gamma= F g\cdot g^{-1}\in G_0(\dF_q)$, les identit\'es $F g=\gamma g$ 
et 
$F\gamma=\gamma$ donnent $F(g^{-1} F g) = (F g)^{-1} F F g = g^{-1} \gamma^{-1} F(\gamma g) = g^{-1} \gamma^{-1} F \gamma F g = g^{-1} F g$, 
de sorte que $g^{-1} F g \in G_0(\dF_q)$. 

(ii) Pour $\alpha,\gamma\in G_0(\dF_q)$, les $g$ tels que 
$F g\cdot g^{-1} = \gamma$ forment une classe \`a droite sous $G_0(\dF_q)$, et 
si $F g\cdot g^{-1} = \gamma$, on a 
$F(\alpha g \alpha^{-1})(\alpha g \alpha^{-1})^{-1} = \alpha \gamma\alpha^{-1}$. 
Pour $\gamma$ parcourant une classe de conjugaison dans $G_0(\dF_q)$, les $g$ 
tels que $F g\cdot g^{-1} = \gamma$ parcourent donc une double classe sous 
$G_0(\dF_q)$, et $\gamma'= g^{-1} F g$ parcourt une classe de conjugaison. 

(iii) Enfin, l'inverse de $F g\cdot g^{-1}\mapsto g^{-1} F g$ est 
$g^{-1} F g\mapsto F g\cdot g^{-1}$. 





\subsection{}\label{VI:1-25}

La formule \eqref{VI:eq:1-23-1} peut se lire, bri\`evement: $F^\ast$ en 
$\gamma$ est $\rho(\gamma')^{-1}$. Si $\gamma$ appartient \`a la composante 
neutre de son centralisateur, on peut prendre $g$ dans $Z(\gamma)$. On a alors 
$\gamma=\gamma'$. Si cette condition n'est pas remplie, $\gamma$ et $\gamma'$ 
peuvent ne pas \^etre conjugu\'es. 





\subsection{Un exemple}\label{VI:1-26}

Pour $G_0=\operatorname{SL}(2)$, $q$ impair, 
$\alpha\in F_q^\times\setminus {\dF_q^\times}^2$ et $\bar\gamma$ la classe de 
conjugaison de $-1(1+N)$, avec $N^2=0$, $\bar\gamma'$ est la classe de 
conjugaison de $-1\cdot (1+\alpha N)$, de sorte que $\bar\gamma'\ne \bar\gamma$ 
si $N\ne 0$. 










\section{Dictionnaire}\label{VI:2}

Une manipulation \'el\'ementaire sur les sommes trigonom\'etriques peut souvent 
\^etre vue comme le reflet, via la formule des traces, d'un \'enonc\'e 
cohomologique. Dans ce paragraphe, j\'enum\`ere de tels \'enonc\'es, et leur 
reflet; il portent le m\^eme num\'ero, augment\'e d'une $\ast$ pour 
l'\'enonc\'e cohomologique. 





\subsection{}\label{VI:2-1}

``Une somme d'entiers alg\'ebriques est un entier alg\'ebrique'' admet pour 
analogue cohomologique le th\'eor\`eme suivant, prouv\'e dans 
\cite[XXI 5.2.2]{sga7} (cf. l'usage de \ref{VI:2-1}$^\ast$ en 
\ref{VI:1-13}). 





\begin{theorem*}[2.1*]\label{VI:2-1*}
Soient $X_0$ un sch\'ema s\'epar\'e de type fini sur $\dF_q$ et $\sF_0$ un 
$E_\lambda$-faisceau sur $X_0$. Si, pour tout $x\in |X_0|$, les valeurs 
propres de $F_x^\ast$ sur $\sF_0$ sont des entiers alg\'ebriques, alors les 
valeurs propres de $F^\ast$ sur $\h_c^i(X,\sF)$ sont des entiers alg\'ebriques. 
\end{theorem*}





\subsection{}\label{VI:2-2}

D'autres r\'esultats d'int\'egralit\'es sont prouv\'es dans 
\cite[XXI, 5.2.2 et 5.4]{sga7}: si $\dim(x_0)\leqslant n$, et que $\alpha$ est 
une valeur propre de $F^\ast$ sur $\h_c^i(X,\sF)$, on a: 
\begin{enumerate}[a)]
  \item sous les hypoth\`eses de \ref{VI:2-1}$^\ast$, et si $i>n$, alors 
    $q^{i-n}$ divise $\alpha$;
  \item si les inverses des valeurs propres de $F_x^\ast$ sont des entiers 
    alg\'ebriques, alors $\alpha^{-1}$ est entier, sauf en $p$. Plus 
    pr\'ecisement, $q^{\inf(n,i)}\alpha^{-1}$ est un entier alg\'ebrique. 
\end{enumerate}

Pour les faisceaux consid\'er\'es en \ref{VI:1-9}, les valeurs propres des 
$F_x^\ast$ sont des racines de l'unit\'e, de sorte que les hypoth\`eses du 
th\'eor\`eme et celle de b) ci-dessus sont v\'erif\'ees. 





\subsection{}\label{VI:2-3}

Soient $f:A\to B$ une application d'un ensemble fini dans un autre, et 
$\varepsilon$ un function sur $A$. On a 
\begin{equation*}\tag{2.3.1}\label{VI:eq:2-3-1}
  \sum_{a\in A} \varepsilon(a) = \sum_{b\in B} \sum_{f(a)=b} \varepsilon(a) \text{.}
\end{equation*}





\subsection*{2.3*}\label{VI:2-3*}

Soient $f:X\to Y$ un morphisme de sch\'emas s\'epar\'es de type fini sur $k$ 
alg\'ebriquement clos, et $\sF$ un $E_\lambda$-faisceau sur $X$. La suite 
spectrale de Leray de $f$ en cohomologie \`a support propre s'\'ecrit 
\begin{align*}\tag{2.3.1*}\label{VI:eq:2-3-1*}
  E_2^{p q} = \h_c^p(Y,\R^q f_! \sF) \Rightarrow \h_c^{p+q}(X,\sF) \text{.}
\end{align*}

Soient $f_0:X_0 \to Y_0$ un morphisme de sch\'emas s\'epar\'es de type fini sur 
$\dF_q$ et $\sF_0$ un $E_\lambda$-faisceau sur $X_0$. On a entre 
\eqref{VI:eq:2-3-1} et \eqref{VI:eq:2-3-1*} la compatibilit\'e suivante. 

\begin{enumerate}[a)]
  \item Le faisceau $\R^q f_! \sF$ se d\'eduit par extension des scalaires de 
    $\dF_q$ \`a $\dF$ du faisceau $\R^q f_{0!} \sF_0$ sur $Y_0$. En tant que 
    tel, il est muni d'une correspondance de Frobenius 
    $F^\ast:F^\ast\R^q f_! \sF \to \R^q f_! \sF$. Elle se d\'eduit par 
    fonctorialit\'e de $\R f_!$ de la correspondance de Frobenius de $\sF$: 
    c'est le compos\'e 
    \[\xymatrix{
      F^\ast \R^q f_! \sF \ar[r]^-\sim 
        & \R^q f_! F^\ast \sF \ar[r]^-\sim 
        & \R^q f_! \sF \text{.}
    }\]
  \item La formule des traces pour $\sF_q$ s'\'ecrit 
    \begin{align*}\tag{1}\label{VI:eq:1}
      \sum_{x\in X_0(\dF_q)} \tr(F_x^\ast,\sF_0) = \tr(F^\ast,\h^\bullet(X,\sF)) \text{;}
    \end{align*}
    celle pour $\R^q f_! \sF$ s'\'ecrit 
    \begin{align*}\tag{2$_q$}\label{VI:eq:2_q} 
      \sum_{y\in Y_0(\dF_q)} \tr(F_y^\ast,\R^q f_! \sF) = \tr(F^\ast, \h^\bullet(Y,\R^q f_! \sF)) \text{.}
    \end{align*}
\end{enumerate}

On a, pour $y\in Y(\dF)$, $(\R^q f_! \sF)_y = \h_c^q(f^{-1}(y),\sF)$, et si 
$y\in Y_0(\dF_q)$, 
\[
  \tr(F_y^\ast,\R^q f_! \sF) = \tr(F^\ast,\h_c^q(f^{-1}(y),\sF)) \text{,}
\]
d'o\`u par la formule des traces sur la fibre 
\[
  \sum_{\substack{x\in X_0(\dF_q) \\ f(x) = y}} \tr(F_x^\ast,\sF_0) = \sum (-1)^q \tr(F_y^\ast,\R^q f_! \sF) \text{.}
\]
Prenant la somme altern\'ee des identit\'es \eqref{VI:eq:2_q}, on trouve donc 
\begin{align*}\tag{2}\label{VI:eq:2}
  \sum_{y\in Y_0(\dF_q)} \sum_{\substack{x \in X_0(\dF_q) \\ f(x) = y}} \tr(F_x^\ast,\sF_0) = \tr(F^\ast,\h^\bullet(Y,\R^\bullet f_! \sF)) \text{.}
\end{align*}

L'\'egalit\'e des membres de gauche de \eqref{VI:eq:1} et \eqref{VI:eq:2} 
r\'esulte de \eqref{VI:eq:2-3-1}, celle des membres de droite de 
\eqref{VI:eq:2-3-1*} (compte tenu du fait que $F^\ast$ est un endomorphisme de 
toute la suite spectrale). 





\subsection{}\label{VI:2-4}

Soient $(A_i)_{i\in I}$ une famille finie d'ensembles finis, 
$A=\prod_{i\in I} A_i$, $\varepsilon_i$ une fonction sur $A_i$, et 
$\varepsilon$ la fonction $a\mapsto \prod \varepsilon_i(a_i)$ sur $A$. On a 
\begin{align*}\tag{2.4.1}\label{VI:eq:2-4-1}
  \sum_{a\in A} \varepsilon(a) = \prod_{i\in I} \sum_{a_i\in A_i} \varepsilon_i(a_i) \text{.}
\end{align*}





\subsection*{2.4*}\label{VI:2-4*}

Soient $(X_i)_{i\in I}$ une famille finie de sch\'emas s\'epar\'es de type fini 
sur $k$ alg\'ebriquement clos, $X=\prod_{i\in I} X_i$, $\sF_i$ un 
$E_\lambda$-faisceau sur $X_i$ et $\sF$ le produit tensoriel externe 
$\bigboxtimes_{i\in I}\sF_i = \bigotimes_{i\in I} \operatorname{pr}_i^\ast(\sF_i)$ 
des $\sF_i$. On a la formule de K\"unneth 
\begin{align*}\tag{2.4.1*}\label{VI:eq:2-4-1*}
  \h_c^\bullet(X,\sF) = \bigotimes_{i\in I} \h_c^\bullet(X_i,\sF_i) \text{.}
\end{align*}

Si on veut un isomorphisme \eqref{VI:eq:2-4-1*} canonique, et ind\'ependant du 
choix d'un ordre total sur $I$, il faut prendre au membre droite le produit 
tensoriel gradu\'e au sens de la r\`egle de Koszul 
(\hyperref[IV]{Cycle}, \ref{V:1-3}). 





\subsection{}\label{VI:2-5}

Soient $A$ un ensemble fini, $B$ une partie de $A$ et $\varepsilon$ une 
fonction sur $A$. On a 
\begin{align*}\tag{2.5.1}\label{VI:2-5-1}
  \sum_{a\in A} \varepsilon(a) = \sum_{a\in A\setminus B}\varepsilon(a) + \sum_{a\in B} \varepsilon(a) \text{.}
\end{align*}





\subsection*{2.5*}\label{VI:2-5*}

Soient $x$ un sc\'ema s\'epar\'e de type fini sur $k$ alg\'ebriquement clos, 
$Y$ un sous-sch\'ema ferm\'e et $\sF$ un $E_\lambda$-faisceaux sur $X$. On a 
une suite exacte longue de cohomologie 
\begin{align*}\tag{2.5.1*}\label{VI:eq:2-5-1*}
\xymatrix{
  \cdots \ar[r]^-\partial 
    & \h_c^i(X\setminus Y,\sF) \ar[r] 
    & \h_c^i(X,\sF) \ar[r] 
    & \h_c^i(Y,\sF) \ar[r]^-\partial 
    & \cdots 
}
\end{align*}

Plus g\'en\'eralement, si on part d'une filtration finie de $X$ par des parties 
ferm\'ees $X_p$ ($p\in \dZ$; on suppose que $X_p\supset X_{p+1}$, que $X_p=X$ 
pour $p$ assez petit, et que $X_p=\varnothing$ for $p$ assez grand), on a une 
suite spectrale 
\begin{align*}\tag{2.5.2*}\label{VI:eq:2-5-2*}
  E_1^{p q} = \h_c^{p+1}(X_p\setminus X_{p+1},\sF) \Rightarrow \h_c^{p+q}(X,\sF) \text{.}
\end{align*}





\subsection{}\label{VI:2-6}

Soient $A$ un ensemble fini, $(A_i)_{i\in I}$ un recouvrement fini de $A$ et 
$\varepsilon$ une fonction sur $A$. Pour $J\subset I$, soit $A_J$ 
l'intersection des $A_j$ ($j\in J$). On a 
\begin{align*}\tag{2.6.1}\label{VI:eq:2-6-1}
  \sum_{J\subset I} (-1)^{|J|} \sum_{a\in A_J} \varepsilon(a) = 0 \text{.}
\end{align*}





\subsection*{2.6*}\label{VI:2-6*}

Soient $X$ un sch\'ema s\'epar\'e de type fini sur $k$ alg\'ebriquement clos, 
$(X_i)_{i\in I}$ un recouvrement ferm\'e (resp. ouvert) fini de $X$ par des 
sous-sch\'emas, et $\sF$ un $E_\lambda$-faisceau sur $X$. Pour $J\subset I$, 
soit $X_J$ l'intersection des $X_j$ ($j\in J$). On a des suites spectrales 
respectives (de Leray) 
\begin{align*}\tag{2.6.1*}\label{VI:eq:2-6-1*}
  E_1^{p q} &= \bigoplus_{|J|=p+1>0} \h_c^q(X_J,\sF) \otimes_\dZ \textstyle\bigwedge^{|J|} \dZ^J \Rightarrow \h_c^{p+q}(X,\sF) \\ 
  \tag{2.6.2*}\label{VI:eq:2-6-2*}
  E_1^{p q} &= \bigoplus_{|J|=1-p>0} \h_c^q(X_J,\sF) \otimes_\dZ \textstyle\bigwedge^{|J|} \dZ^J \Rightarrow \h_c^{p+q}(X,\sF)
\end{align*}
Si $J=\varnothing$ n'\'etait pas exclus, on aurait de m\^eme des suites 
spectrales convergeant vers $0$. Les facteurs $\bigwedge^{|J|} \dZ^J$ sont l\`a 
pour nous dispenser de choisir un ordre total sur $I$. 





\subsection{}\label{VI:2-7}

Soient $A$ un groupe commutatif fini, et $\chi:A \to E_\lambda^\times$ un 
caract\`ere non trivial. On a 
\begin{align*}\tag{2.7.1}\label{VI:eq:2-7-1}
  \sum_{a\in A} \chi(a) = 0 \text{.}
\end{align*}
On prouvera l'analogue cohomologique de \eqref{VI:eq:2-7-1} en en transposant 
la preuve suivante: si $x\in A$, $a\mapsto x a$ est une permutation de $A$, et 
\begin{align*}
  \sum_{a\in A}\chi(a) = \sum_{a\in A} \chi(x a) &= \chi(x) \sum_{a\in A} \chi(a) && \text{, d'o\`u} \\
  (\chi(x)-1)\sum_{a\in A} &= 0 \text{,}
\end{align*}
et il existe par hypoth\`ese $x$ tel que $\chi(x)-1\ne 0$. 





\begin{theorem*}[2.7*]\label{VI:2-7*}
Soient $G_0$ un groupe alg\'ebrique commutatif connexe sur $\dF_q$ et 
$\chi:G_0(\dF_q) \to E_\lambda^\times$ un caract\`ere non triviale. On a 
\begin{align*}\tag{2.7.1*}\label{VI:eq:2-7-1*}
  \h_c^\bullet(G,\sF(\chi)) = 0 \text{.}
\end{align*}
\end{theorem*}

Pour $x$ un point rationnel de $G$, notons $t_x$ la translation $t_x(g) = x g$ 
de $G$. La formule $\fL t_x = t_{\fL(x)} \fL$ exprime que $(t_x,t_{\fL(x)})$ 
est un automorphisme du diagrame $G\xrightarrow\fL G$ ($G$ muni du torseur de 
Lang). Soit $\rho(g)$ l'automorphisme de $(G,\sF(\chi))$ qui s'en d\'eduit. 
Pour $g\in G_0(\dF_q)$, i.e. pour $\fL(g) = e$, c'est l'identit\'e sur $G$, et 
la multiplication par $\chi(g)^{-1}$ sur $\sF(\chi)$. 

Notons $\rho_H(g)$ l'automorphisme de $\h_c^\bullet(G,\sF(\chi))$ d\'eduit de 
$\rho(g)$. Un argument d'homotopie (Lemme \ref{VI:2-8} ci-dessous) montre que 
$\rho_H(g) = \rho_H(e)$, donc est l'identit\'e. D'autre part, pour 
$g\in G_0(\dF_q)$, $\rho_H(g)$ est la multiplication par 
$(\chi^{-1}(g):\text{ sur }\h^\bullet(G,\sF(\chi))$, la multiplication par 
$(\chi^{-1}(g)-1)$ est nulle. Prenant $g$ tel que $\chi(g)\ne 1$, on obtient 
\ref{VI:2-7*}. 





\begin{lemma_}\label{VI:2-8}
Soient $X$ et $Y$ deux sch\'emas sur $k$ alg\'ebriquement clos, avec $X$ 
s\'epar\'e de type fini et $Y$ connexe. Soient $\sF$ un faisceau sur $X$ et 
$(\rho,\varepsilon)$ une famille d'endomorphismes de $(X,\sF)$ param\'etr\'ee 
par $Y$: 
\begin{align*} 
  \rho:Y\times_k X &\to Y\times_k X && \text{est un $Y$-morphisme, et} \\
  \varepsilon:\rho^\ast \operatorname{pr}_2^\ast \sF &\to \operatorname{pr}_2^\ast\sF && \text{un morphisme de faisceaux.} 
\end{align*}
On suppose que $\rho$ est propre. Pour $y\in Y(k)$, soit $\rho_H(y)^\ast$ 
l'endomorphisme de $\h_c^\bullet(X,\sF)$ induit par $\rho_y:X\to X$ et 
$\varepsilon_y:\rho_y^\ast\sF\to \sF$. Alors, $\rho_H(y)^\ast$ est 
ind\'ependant de $y$. 
\end{lemma_}

En effet, $\R^p \operatorname{pr}_{1!} \operatorname{pr}_2^\ast\sF$ est le 
faisceau constant sur $Y$ de valeur $\h^p(X,\sF)$, et $\rho_H(y)^\ast$ est la 
fibre en $y$ de l'endomorphisme 
\[\xymatrix{
  \R^p \operatorname{pr}_{1!} \operatorname{pr}_2^\ast \sF \ar[r]^-{\rho^\ast} 
    & \R^p \operatorname{pr}_{1!} \rho^\ast \operatorname{pr}_2^\ast \sF \ar[r]^-\varepsilon 
    & \R^p \operatorname{pr}_{1!} \operatorname{pr}_2^\ast \sF 
}\]
de ce faisceau. 





\subsection{Remarque}\label{VI:2-9}

Soient $G_0$ un groupe alg\'ebrique connexe sur $\dF_q$ et 
$\rho:G_0(\dF_q) \to \operatorname{GL}(V)$ un repr\'esentation lin\'eaire 
telle que $V^{G_0(\dF_q)}=0$. On a encore $\h_c^\bullet(G,\rho(L))=0$. 










\section{Sommes \`a une variable}\label{VI:3}





\subsection{}\label{VI:3-1}

Weil est le premier \`a avoir appliqu\'e des m\'ethodes ``cohomologiques'' \`a 
l'\'etude des sommes trigonom\'etriques \`a une variable; puisqu'il avait 
prouv\'e l'analogue de l'hypoth\`ese de Riemann pour les fonctions $L$ d'Artin 
sur les corps de fonctions, ces m\'ethodes lui fournissaient d'excellentes 
estimations (en $O(\text{racine carr\'ee du nombre de termes})$), et le 
comportement de ces sommes par ``extension du corps de base.'' 

L'essentiel de ce paragraphe est un expos\'e, dans un langage cohomologique, de 
ses r\'esultes. 





\subsection{}\label{VI:3-2}

Les m\'ethodes cohomologiques am\`enent ici \`a calculer des groupes 
$\h_c^i(X,\sF)$, pour $\sF_0$ un $E_\lambda$-faisceau sur un courbe $X-0$ sur 
$\dF_q$. Ces groupes sont nuls pour $i\ne 0,1,2$ et pour 
$i=0,2$, ils ont une interpr\'etation simple (\ref{VI:1-8} a,b,c). De plus, la 
caract\'eristique d'Euler-Poincar\'e 
$\chi_c(\sF) = \sum (-1)^i \dim \h_c^i(X,\sF)$ (par ailleurs \'egale \`a 
$\chi(\sF) = \sum (-1)^i \dim \h^i(X,\sF)$ peut \^etre calcul\'ee en terme de 
$x$, de rang de $\sF$ aux points g\'en\'eriques de $X$, et des propri\'et\'es 
de ramification de $\sF$. Le r\'esultat essentiel est le suivant. 

Soient $\bar X$ une courbe projective lisse et connexe, de genre $g$, sur un 
corps alg\'ebriquement clos $k$, $X$ un ouvert dense de $\bar X$, 
l'compl\'ement d'un ensemble fini $S$ de points et $\sF$ un 
$E_\lambda$-faisceau lisse sur $X$. Soient $\chi(X)=2-2 g-\# S$ la 
caract\'eristique d'Euler-Poincar\'e de $X$, et $\operatorname{rg}(\sF)$ le 
rang de $\sF$. Pour chaque point $s\in S$, on d\'efinit un entier 
$\swan_s(\sF)$, le conducteur de Swan, mesurant la ramification sauvage de 
$\sF$, et 
\begin{align*}\tag{3.2.1}\label{VI:eq:3-2-1}
  \chi_c(\sF) = \operatorname{rg}(\sF) \cdot \chi(X) - \sum_{s\in S} \swan_s(\sF) 
\end{align*}
\cite{ra65}. 





\subsection{}\label{VI:3-3}

Soient $\bar X$, $j:X\hookrightarrow\bar X$ et $\sF$ comme en \ref{VI:3-2}. Le 
th\'eor\`eme de dualit\'e de Poincar\'e admet la forme tr\`es maniable: 
$\h^i(\bar X,j_\ast \sF)$ et $\h^{2-i}(\bar X,j_\ast(\sF^\vee(1)))$ sont 
duaux l'un de l'autre (\hyperlink{V}{Dualit\'e}, \ref{V:1-3}). 





\subsection{}\label{VI:3-4}

Faisons $k=\dF$, et supposons que $X$, $\bar X$, et $\sF$ proviennent de 
$X_0$, $\bar X_0$ et $\sF_0$ sur $\dF_q$. Si $\sF_0$ devient trivial sur un 
rev\^etement fini de $X_0$, Weil a prouv\'e que les conjugu\'es complexes des 
valeurs propres de $F^\ast$ sur $\h^i(\bar X,j_\ast\sF)$ sont de valeur 
absolue $q^{i/2}$. De tels faisceaux $\sF_0$ correspondent aux fonctions $L$ 
d'Artin sur le corps de fonctions de $\bar X_0$. 





\subsection*{3.5}\label{VI:3-5_} % originally one of two 3.5s

Sous les m\^emes hypoth\`eses, ou plus g\'en\'eralement si $E_\lambda$ est le 
compl\'et\'e en une place $\lambda$ d'un corps de nombres $E$ et que $\sF_0$ 
appartient \`a un syst\`eme compatible infini de repr\'esentations $v$-adiques 
($v$ place de $E$), on peut calculer 
\[
  \prod_i \det(-F^\ast,\h^i(X,\sF))^{(-1)^i} 
\]
\`a partir d'informations \emph{locales} sur $\sF_0$. Ceci est explique dans 
\cite{de73}. Pour $\sF_0$ trivial sur un rev\^etement fini de $X_0$, c'est 
l'expression de la constante de l'\'equation fonctionnelle d'une fonction $L$ 
d'Artin comme produit de constantes locales. 





\subsection{Exemple}\label{VI:3-5}

Soit $\psi$ le caract\`ere 
$\exp\left(\frac{2\pi i}{p} \tr_{\dF_q/\dF_p}\right)$ de $\dF_q$; on a 
$\psi(a^p-a)=0$. Soient $X_0$ un courbe projective lisse absolument 
irr\'eductible de genre $g$ sur $\dF_q$, et $f$ une fonction rationnelle sur 
$X_0$, i.e. un morphisme $f:X_0 \to \dP^1$, non identiquement \'egale \`a 
$\infty$. On s'int\'erresse \`a la somme 
\[
  S_f' = \sum_{\substack{x\in X_0(\dF_q) \\ f(x)\ne \infty}} \psi(f(x)) 
  \text{.}
\]
Cette somme est nulle pour $f$ de la forme $g^p-g$. Ceci sugg\`ere de 
modifier la somme $S_f'$ comme suite: 
\begin{enumerate}[\indent a)]
  \item pour tout point ferm\'e $x$ de $X_0$, on pose $v_x(f) = $ ordre du 
    pole de $f$ en $x$ si $f(x)=\infty$, $v_x(f)=0$ sinon, et 
    $v_x^\ast(f) = \inf v_x(f+g^p-g)$ (borne inf\'erieure sur $g$). 
  \item Si $v_x^\ast(f)=0$, que $x\in X_0(\dF_q)$, et que $f+g^p-g$ est 
    r\'egulier en $x$, on pose $\psi(f(x)) = \psi((f+g^p-g)(x))$. On pose enfin 
    \begin{equation*}\tag{3.5.1}\label{VI:eq:3-5-1}
      S_f = \sum'_{x\in X_0(\dF_q)} \psi(f(x)) \text{,}
    \end{equation*}
    o\`u $(-)'$ indique que la somme est \'etendue aux $x$ tels que 
    $v_x^\ast(f)=0$. 
\end{enumerate}

On a $S_f=S_{f+g^p-g}$. On se propose de v\'erifier que si $f$ n'est pas la 
forme $g^p-g+c^{t e}$, alors 
\begin{equation*}\tag{3.5.2}\label{VI:eq:3-5-2}
  |S_f|\leqslant \left(2 g-2+\sum_{v_x^\ast(f)\ne 0} [k(x):\dF_q] (1+v_x^\ast(f))\right) q^{1/2} \text{.}
\end{equation*}

On commence par passer du complexe au $\ell$-adique, en remplaçant $\psi$ par 
un caract\`ere de la forme $\psi_0\circ \tr_{\dF_q/\dF_p}$, avec 
$\psi_0:\dF_p\hookrightarrow E_\lambda^\times$. Pour $j:U_0\hookrightarrow X_0$ 
l'inclusion de l'ouvert o\`u $f\ne\infty$, on preuve alors que 
\begin{equation*}\tag{3.5.3}\label{VI:eq:3-5-3}
  S_f = \sum (-1)^i \tr(F^\ast,\h^i(X,j_\ast \sF(\psi f))) \text{.} 
\end{equation*}

La formule des traes ram\`ene cet \'enonc\'e aux suivants (pour 
$x\in X_0(\dF_q)$): 
\begin{enumerate}[\indent a)]
  \item Si $v_x^\ast(f)=0$, alors $F_x^\ast$ sur $j_\ast \sF(\psi f)$ est 
    $\psi(f(x))$; 
  \item Si $v_x^\ast(f)\ne 0$, alors $j_\ast \sF(\psi f)$ se ramifie en $x$: 
    $(j_\ast \sF(\psi f))_{\bar x}=0$. 
\end{enumerate}

La formule a) r\'esulte de \eqref{VI:eq:1-7-6} si $v_x(f)=0$, et on se ram\`ene 
\`a ce cas en remplaçant $f$ par $f+g^p-g$: \`a isomorphisme pr\`es, ceci ne 
change pas $\sF(\psi f)_0$ sur l'ouvert o\`u $f$ et $g$ sont r\'eguliers 
(\ref{VI:1-3}). 

La formule b) r\'esulte de 
\begin{equation*}\tag{3.5.4}\label{VI:eq:3-5-4}
  \swan_x \sF(f,\psi) = v_x^\ast(f) \text{.} 
\end{equation*}
Apr\`es r\'eduction au cas o\`u $v_x(f)=v_x^\ast(f)$ (d'o\`u $p\nmid v_x(f)$) 
et extension des scalaires \`a $\dF$, cette formule est dans 
\cite[4.4]{se61}. 

Enfin, on v\'erifie que le faisceau $\sF(\psi f)$ sur $U$ est non constant si 
$f$ n'est pas de la forme $g^p-g+c^{t e}$: puisque $\psi_0$ est injectif, la 
trivialit\'e de $\sF(\psi f)=\sF(\psi_0 f)$ (\ref{VI:1-8}.ii) \'equivaut \`a 
celle du $\dF_q$-torseur d'\'equation serait l'image r\'eciproque d'un 
$\dF_q$-torseur sur $\spec(\dF_q)$, d'\'equation $T^p-T-\lambda=0$, le 
$\dF_p$-torseur d'\'equation $T^p-T-(f-\lambda)=0$ sur $U_0$ serait trivial 
et $f$ serait donc de la forme $g^p-g+\lambda$. Les 
$\h^i(\bar X,j_\ast \sF(\psi f))$ sont donc nuls pour $i\ne 1$, la formule 
\eqref{VI:eq:3-5-3} se r\'eduit \`a 
\begin{equation*}\tag{3.5.5}\label{VI:eq:3-5-5}
  S_f = -\tr(F^\ast,\h^1(\bar X,j_\ast \sF(\psi f))) 
\end{equation*}
et, d'apr\`es \eqref{VI:eq:3-2-1} et \eqref{VI:eq:3-5-4}, de $\h^1$ est de 
dimension $2 g-2+\sum_{v_x^\ast(f)>0} [k(x):\dF_q](1+v_x^\ast(f))$ et $-S_f$ 
est somme de ce nombre de valeurs propres de $F^\ast$, chacune de valeurs 
absolues complexes $q^{1/2}$. 





\subsection{}\label{VI:3-6}

Supposons que, pour un automorphisme $\sigma$ de $X_0$, on ait 
\[
  f(\sigma x) = - f(x) \text{.} 
\]
La somme $S_f$ est alors r\'eelle. Si $\sigma$ est involutif, la dualit\'e de 
Poincar\'e permet de dire un peu plus: si on pose 
$\sF_0=j_\ast \sF(\psi f)$, $\sG_0 = j_\ast \sF(\psi(-f))$, on a 
\begin{enumerate}[\indent a)]
  \item $\sF_0$ et $\sG_0$ sont en dualit\'e, 
  \item $\sigma^\ast \sF_0 \iso \sG_0$ et $\sigma^\ast \sG_0 \iso \sF_0$, pour 
    des isomorphismes naturels tel que le compos\'e 
    $\sF_0=(\sigma^2)^\ast \sF_0 \iso \sigma^\ast \sG_0 \iso \sF_0$ soit 
    l'identit\'e. 
  \item L'accouplement $\sF_0\otimes \sG_0 \to E_\lambda$ v\'erifie 
    $\sigma^\ast(f\cdot g) = \sigma^\ast(f) \cdot \sigma^\ast (g)$. 
\end{enumerate} 

Passant \`a cohomologie, on trouve que $\h^1(X,\sF)$ et $\h^1(X,\sF)$ sont en 
dualit\'e parfaite \`a valeur dans $E_\lambda(1)$, et que la forme bilin\'eaire 
$\alpha\cdot \sigma^\ast\beta$ sur $\h^1(X,\sF)$ est \emph{altern\'ee}: 
$\sigma^\ast$ agit trivialement sur $E_\lambda(1)$, et 
\[
  \alpha\cdot \sigma^\ast \beta = \sigma^\ast(\alpha\cdot \sigma^\ast \beta) = \sigma^\ast \alpha\cdot \beta = -\beta\cdot \sigma^\ast \alpha \text{.} 
\]
On en conclut que $\h^1(X,\sF)$ est de dimension paire (on v\'erifie d'ailleurs 
facilement que chaque $v_x^\ast(f)$ non nul est impair) et que les valeurs 
propres de $F^\ast$ sur $\h^1(X,\sF)$ sont group\'ees en pairs $\alpha$ et 
$q/\alpha$. 





\subsection{Exemple}\label{VI:3-7}

Ceci s'applique aux sommes de Kloosterman 
$\sum_{x\in \dF_q^\times} \psi\left(x+\frac a x\right)$ (faire 
$X_0=\dP^1$, $f=x+\frac a x$, $\sigma x=-x$; les p\^oles de $f$ sont $0$ et 
$\infty$, et en chacun d'eux $v_x^\ast(f)=1$; le $\h^1$ est de dimension 
$-2+2+2=2$). On a donc (pour $a\ne 0$) 
\[
  \sum_{\substack{x y=a \\ x,y\in \dF_{q^n}}} \exp\left(\frac{2\pi i}{p} \tr_{\dF_{q^n}/\dF_p}(x+y)\right) = (-\alpha^n+\alpha^{-n}) \qquad \text{, }\alpha\bar \alpha = q \text{.} 
\]
Ce r\'esultat est d\^u \`a L.\ Carlitz \cite{ca69}. 

Voici maintenant une application d'une m\'ethode de Lang-Weil. 





\begin{proposition_}\label{VI:3-8}
Soit $P\in \dF_q[X_1,\dots,X_n]$ un polyn\^ome \`a $n$ variable, de degr\'e 
$d$, dont on suppose qu'il n'est pas de la forme $Q^p-Q+C^{te}$. Posant encore 
$\psi(x) = \exp\left(\frac{2\pi i}{p} \tr_{\dF_q/\dF_p}(x)\right)$ on a 
\[
  \left|\sum_{x_i\in \dF_q} \psi P(x_1,\dots,x_n)\right| \leqslant (d+1) q^{n-\frac 1 2} \text{.}
\]
\end{proposition_}

On a pour le membre de gauche l'estimation triviale $q^n$. Il suffit donc de 
prouver \ref{VI:3-8} lorsque $d-1<q^{1/2}$; supposons seulement que 
$d<q+1$, et soit $P_d$ la partie homog\`ene de degr\'e $d$ de $P$. Rappelons le 





\begin{lemma_}\label{VI:3-9}
Une hypersurface de degr\'e $d$ dans $\dP^r$ ($r\geqslant i$) ne peut passer 
par tout les points rationnels sur $\dF_q$ qui se $d\geqslant q+1$. 
\end{lemma_}

On proc\`ede par r\'ecurrence: s'il existe un hyperplan rationnel non 
enti\`erement contenu dans l'hypersurface (tel n'est pas le cas pour $r=1$), on 
applique l'hypoth\`ese de r\'ecurrence \`a la trace de l'hypersurface sur cet 
hyperplan. Sinon, le degr\'e $d$ est $\geqslant$ le nombre d'hyperplans 
rationnel, $\geqslant q+1$. 

Distinguons maintenant deux cas. 


\paragraph{Cas 1: $p\nmid d$.}
Appliquant le lemma \`a $P_d$, on voit que, quitte \`a faire un changement 
lin\'eaire de variables, on peut supposer que $P_d(1,0,\ldots,0)\ne 0$, i.e. 
que le coefficient de $X_1^d$ dans $P$ est non nul. Un polyn\^ome \`a $1$ 
variable 
\[
  S(X) = \sum_{i=0}^d a_i x^i \text{,} 
\]
avec $a_d\ne 0$ ($p\nmid d$) n'est jamais de la forme $Q^p-Q+C^{t e}$, et 
l'estimation \eqref{VI:eq:3-5-2} se r\'eduit \`a 
\[
  \left| \sum \psi(S(x)) \right| \leqslant (d-1) q^{1/2} \text{.} 
\]
Appliquant cette estimation aux sommes partielles obtenues en ne faisant varier 
que $x_1$, on trouve 
\[
  \left| \sum \psi P(x_1,\dots,x_n) \right| = \left| \sum_{x_2,\dots,x_n} \sum_{x_1} \psi P(x_1,\dots,x_n)\right| \leqslant q^{n-1} (d-1) q^{1/2} 
\]
comme promis. 


\paragraph{Cas 2: $p\mid d$ ($d>0$).}
Si $P_d$ est une puissance $p$-i\`eme, remplacer $P$ par $P-(P_d-P_d^{1/p})$ ne 
change pas la somme consid\'er\'ee, et abaisse le degr\'e: on se d\'ebarasse de 
ce cas en proc\'edant par r\'ecurrence sur $d$. Sinon, la diff\'erentielle de 
$P_d$ n'est pas identiquement nulle; appliquant \ref{VI:3-9}, on peut supposer, 
quitte \`a faire un changement lin\'eaire de variables, qu'elle n'est pas nulle 
au point $(1,0,\dots,0)$. Si on \'ecrit 
\[
  P_d = \sum_{i=0}^d X_1^{d-i} S_i(X_2,\dots,X_n) \text{,} 
\]
cela signifie que la forme lin\'eaire $S_1$ n'est pas identiquement nulle. 

La forme $S_0$ est une constante, et remplaçant $P$ par 
$P-(S_0 X_1^d -S_0^{1/p} X_1^{d/p})$, on peut supposer qu'elle est nulle. Soit 
enfin $-\lambda$ le coefficient de $X_1^{d-1}$ dans $P$. Pour $x_2,\dots,x_n$ 
fixes, on a 
\[
  P(X_1,x_2,\dots,x_n) = (S_1(x_2,\dots,x_n)-\lambda) X_1^{d-1} + \text{termes de plus bas degr\'e en $X_1$} 
\]
et si $S_1(x_2,\dots,x_n)\ne \lambda$, on a donc 
\[
  \left| \sum_{x_1} \psi P(x_1,\dots,x_n)\right| \leqslant (d-2) q^{1/2} \text{.} 
\]
Au total, 
\begin{align*}
  \left|\sum \psi P(x_1,\dots,x_n)\right| 
    &\leqslant \left|\sum_{S=\lambda} \psi P(x_1,\dots,x_n)\right| + \sum_{S(X_2,\dots,X_n)\ne\lambda} \left| \sum_{x_1} \psi P(x_1,\dots,x_n)\right| \\
    &\leqslant q^{n-1} + (q^{n-1}-q^{n-2})(d-2) q^{1/2} \\
    &< (d-2) q^{n-1/2} + q^{n-1} \\
    &< (d-1) q^{n-1/2} \text{.}
\end{align*}

Un autre r\'esultat de cette nature est donn\'e par R.A.\ Smith 
\cite{sm70}. 










\section{Sommes de Gauss et sommes de Jacob}\label{VI:4}





\subsection{}\label{VI:4-1}

Soient $k$ un corps fini de caract\'eristique $p$, $\chi$ un caract\`ere de 
$k^\times$, et $\psi$ un caract\`ere non trivial du groupe additif de $k$. Nous 
prendrons pour d\'efinition des sommes de Gauss: 
\begin{equation*}\tag{4.1.1}\label{VI:eq:4-1-1}
  \tau(\chi,\psi) = -\sum_{x\in k^\times} \psi(x) \chi^{-1}(x) 
\end{equation*}
(noter le signe). Classiquement, $\chi$ et $\psi$ sont \`a valeurs complexes. 
Nous les prendrons \`a valeurs dans $E_\lambda^\times$ (cf. la preuve de 
\ref{VI:1-15}). Regardons $-\tau(\chi,\psi)$ comme une somme sur les points 
rationnels du sch\'ema $\dG_m$ sur $k$. D'apr\`es le paragraphe \ref{VI:1}, 
on a $-\tau(\chi,\psi) = \tr(F^\ast,\h_c^\ast(\dG_m,\sF(\psi \chi^{-1}))$. De 
plus 





\begin{proposition_}\label{VI:4-2}
La cohomologie de $\dG_m$ \`a coefficient dans $\sF(\psi\chi^{-1})$ v\'erifie 
\begin{enumerate}[(i)]
  \item $\h_c^i=0$ pour $i\ne 1$, et $\dim \h_c^1=1$. 
  \item $F^\ast$, agissant sur $\h_c^1$, est la multiplication par 
    $\tau(\chi,\psi)$. 
  \item Si $\chi$ est non trivial, on a $\h_c^\bullet \iso \h^\bullet$. 
\end{enumerate}
\end{proposition_}
\begin{proof}[Preuve]
Pour tout $n$ premier \`a $p$, notons $K_n$ le $\dmu_n$-torseur sur $\dG_m$ 
d\'efini par la suite exacte de Kummer 
$0 \to \dmu_n \to \dG_m \xrightarrow{x^n} \dG_m \to 0$. Si $k$ a $q$ 
\'el\'ements, on a sur $k$: $\dmu_{q-1}=k^\times$, et le torseur de Lang sur 
$\dG_m/k$ est $K_{q-1}$. D\`es lors, $\sF(\chi^{-1})=\chi(K_{q-1})$. 
L'assertion \ref{VI:4-2}(ii) r\'esulte de (i,iii), eux-m\^emes contenus dans 
l'\'enonc\'e g\'eom\'etrique suivant, o\`u $\sF(\psi)$ d\'esigne le faisceau 
sur $\dG_m/\dF$ d\'eduit par extension des scalaires de $k$ \`a $\dF$ de 
$\sF(\psi)$ sur $\dG_m/k$ (\ref{VI:1-8}.b). 
\end{proof}





\begin{proposition_}\label{VI:4-3}
Soit $\chi:\dmu_n \to E_\lambda^\times$. La cohomologie de $\dG_m$ \`a 
coefficient dans $\sF(\psi)\otimes \chi(K_n)$ v\'erifie 
\begin{enumerate}[\indent (i)]
  \item $\h_c^i=0$ pour $i\ne 1$, et $\dim \h_c^1{} = 1$. 
  \item Si $\chi$ est non triviale, on a $\h_c^\bullet{}\iso \h^\bullet{}$. 
\end{enumerate}
\end{proposition_}
\begin{proof}[Preuve]
Le faisceau $\sF(\psi)$ est la restriction \`a $\dG_m$ d'un faisceau localement 
constant sur $\dG_a$, sauvagement ramifi\'e \`a l'infini, de conducteur de 
Swan $1$. Le faisceau $\chi(K_n)$ est constant si $\chi=1$; si $\chi\ne 1$, il 
est ramifi\'e en $0$ et $\infty$, mod\'er\'ement. 

Le faisceau $\sF(\psi)\otimes\chi(K_n)$ est donc ramifi\'e \`a l'$\infty$; 
appliquant \ref{VI:1-18}(b,c), on trouve que 
$\h_c^i(\dG_m,\sF(\psi)\otimes \chi(K_n))=0$ pour $i\ne 1$. Si $\chi\ne 1$, il 
est ramifi\'e en $0$ et $\infty$ et (ii) r\'esulte de \ref{VI:1-19}.a. 

Les conducteurs de Swan sont $0$ en $0$ et $1$ en $\infty$. D'apr\`es 
\eqref{VI:eq:3-2-1}, la caract\'eristique d'Euler-Poincar\'e est donc $-1$ et 
ceci ach\`eve la d\'emonstration. 
\end{proof}





\subsection{Remarque}\label{VI:4-4}

Si $\chi$ est non trivial, \ref{VI:4-2}(iii) et la dualit\'e de Poincar\'e 
montrent que $\h_c^1(\dG_m,\sF(\psi\chi^{-1}))$ et 
$\h_c^1(\dG_m,\sF(\psi^{-1}\chi))$ sont en dualit\'e (dualit\'e \`a valeurs 
dans $E_\lambda(-1)$). On a donc 
$\tau(\chi,\psi)\cdot \tau(\chi^{-1},\psi^{-1}) = q$, i.e. 
$|\tau(\chi,\psi)|=1$. 





\subsection{}\label{VI:4-5}

Si $k$ est une extension de degr\'e $N$ de $\dF_q$, on peut aussi regarder 
\eqref{VI:eq:4-1-1} comme une somme \`a $N$ variables sur $\dF_q$. Soit plus 
g\'en\'eralement $k$ une alg\`ebre \'etale sur $\dF_q$, de degr\'e $N$ sur 
$\dF_q$. C'est un produits de corps $k_i$, de degr\'e $N_i$, et on pose 
\begin{equation*}\tag{4.5.1}\label{VI:eq:4-5-1}
  \varepsilon(k) = (-1)^{\sum (N_i+1)} \text{.} 
\end{equation*}
C'est la signature de la permutation de $S=\hom_{\dF_q}(k,\dF)$ induite par la 
substitution de Frobenius $\varphi\in \gal(\dF/\dF_q)$. 

Soient $\psi:\dF_q\to E_\lambda^\times$ un caract\`ere non trivial, et 
$\chi:k^\times \to E_\lambda^\times$. On pose 
\begin{equation*}\tag{4.5.2}\label{VI:eq:4-5-2}
  \tau_{\dF_q}(\chi,\psi) = (-1)^N \sum_{x\in k^\times} \psi \tr_{k/\dF_q}(x) \cdot \chi^{-1}(x) \text{.} 
\end{equation*}

Si $\chi$ a pour coordonn\'ees les $\chi_i:k_i^\times \to E_\lambda^\times$, on 
a l'identit\'e triviale 
\begin{equation*}\tag{4.5.3}\label{VI:eq:4-5-3}
  \tau_{\dF_q}(\chi,\psi) = \varepsilon(k) \prod \tau(\chi_i,\psi\circ \tr_{k_i/\dF_q}) \text{.}
\end{equation*}

Apr\`es quelques pr\'eliminaires, nous donnerons en \ref{VI:4-10} une 
interpr\'etation cohomologique des sommes \eqref{VI:eq:4-5-2}. En 
\ref{VI:4-12}, nous interpreterons l'identit\'e de Hasse-Davenport comme une 
forme tordue (cf. \ref{VI:1-12}) du cas particulier suivant de 
\eqref{VI:eq:4-5-3}: pour $k=\dF_q^N$, et $\chi$ un caract\`erre de 
$\dF_q^\times$, on a 
$\tau_{\dF_q}(\chi\circ N_{k/\dF_q},\psi)=\tau(\chi,\psi)^N$. 





\subsection{}\label{VI:4-6}

Rappelons que pour $M$ un module projectif de type fini sur un anneau $A$, le 
foncteur $\spec(B)\mapsto M\otimes_A B$, des sch\'emas affines sur $\spec(A)$ 
dans $\mathsf{Ens}$ est repr\'esent\'e par le sch\'ema affine $\dV(M^\vee)$ 
(notations des EGA), le spectre de $\operatorname{Sym}_A^\bullet(M^\vee)$. Si 
$M=A^n$, c'est l'espace affine type de dimension $n$. 

Supposons que $M$ soit une $A$-alg\`ebre \`a unit\'e, et posons 
$V=\dV(M^\vee)$. Par d\'efinition, pour toute extension $B$ de $A$, l'ensemble 
$V(B)$ de points de $V$ \`a coordonn\'ees dans $B$ est $M\otimes_A B$. Le 
foncteur $B\mapsto M\otimes_A B$ \'etant \`a valeurs dans les anneaux \`a 
unit\'e, $V$ est un sch\'ema en anneaux \`a unit\'es sur $\spec(A)$. Les 
morphismes norme et trace: $M\otimes_A B\to B$ sont fonctoriels en $B$; ils 
correspondent donc \`a des morphismes de sch\'ema $N$ et $T$ de $V$ dans 
$\dG_a$. Nous aurons \`a consid\'erer les sch\'emas d\'eduits de $V$ suivant: 
\begin{enumerate}[a)]
  \item $V^\ast$ est l'ouvert des \'el\'ements inversibles de $V$ (un sch\'ema 
    en groupes pour $\cdot$);
  \item $W$ est l'hyperplan d'\'equation $T=0$ et $W^\ast=W\cap V^\ast$; 
  \item $P$ est l'hyperplan \`a l'infini $V\setminus \{0\}/\dG_m$ de l'espace 
    affine $V$ sur $\spec(A)$. Si $Q$ est l'hyperplan \`a l'infini de $W$, 
    $W^\ast/\dG_m$ et $V^\ast/\dG_m$ sont des ouverts des espaces projectifs 
    $Q$ et $P$ sur $\spec(A)$. 
    \begin{equation*}\tag{4.6.1}\label{VI:eq:4-6-1}
    \xymatrix{
      W^\ast \ar@{^{(}->}[r] \ar[d]^-\pi 
        & V^\ast \ar[d]^-\pi \\
      W^\ast/\dG_m \ar@{^{(}->}[r] 
        & V^\ast/\dG_m 
    }\qquad\subset\qquad
    \xymatrix{
      W\setminus \{0\} \ar@{^{(}->}[r] \ar[d]^-\pi 
        & V\setminus \{0\} \ar[d]^-\pi \\
      Q \ar@{^{(}->}[r] 
        & P \text{.}
    }
    \end{equation*}
\end{enumerate}

Si $M=A^I$, $I$ un ensemble fini, alors $V\simeq \dG_a^I$, $V^\ast=\dG_m^I$, 
$V^\ast/\dG_m$ est le tore $\dG_m^I/(\dG_m\text{ diagonal})$, $N$ et $T$ 
s'\'ecrivant $\prod x_i$ et $\sum x_i$, et $Q$ est donc l'hyperplan 
projectif d'\'equation $\sum x_i=0$ de $P$. 

Le cas qui nous int\'eresse est celui o\`u $M$ est fini \'etale sur $A$. La 
description pr\'ec\'edente vaut alors localement sur $\spec(A)$ (pour la 
topologie \'etale) et on peut \'ecrire $V^\ast=\dG_m^I$, pour $I$ un faisceau 
localement constant d'ensembles finis sur $\spec(A)$. Si par exemple $A$ est un 
corps, de cl\^oture alg\'ebrique $\bar A$, et que $M$ est un $A$-alg\`ebre 
s\'eparable, on pose $I=\hom_A(M,\bar A)$, et, sur $\bar A$, on a 
$V\sim \dG_a^I$. Via cet isomorphisme, $N$ et $T$ s'\'ecrivent $\prod x_i$ et 
$\sum x_i$. 





\subsection{Torseur de Kummer}\label{VI:4-7}

Soient $S$ un sch\'ema, $n$ un entier inversible sur $S$ et $G$ un sch\'ema en 
groupes commutatifs \`a fibres connexes (not\'e multiplicativement) sur $S$. 
La suite 
\[\xymatrix{
  0 \ar[r] 
    & G_n \ar[r] 
    & G \ar[r]^-{x^n} 
    & G \ar[r] 
    & 0 
}\]
est exacte. Elle d\'efinit un $G_n$-torseur $K_n(G)$ (ou simplement $K_n$) sur 
$G$. Pour $\chi:G_n \to E_\lambda^\times$ un homomorphisme du $S$-faisceau 
\'etale $G_n$ dans le faisceau constant $E_\lambda^\times$, on note 
$\sK_n(\chi)$ le $E_\lambda$-faisceau $\chi^{-1}(K_n)$. 

Les torseurs $K_n$ forment un syst\`eme projectif de torseurs sous le 
syst\`eme projectif de groupes $G_n$ (morphismes de transition 
$x\mapsto x^d:G_{n d}\to G_n$). Ceci exprime la commutativit\'e des diagrammes 
\[\xymatrix{
  0 \ar[r] 
    & G_{n d} \ar[r] \ar[d]^-{x^d} 
    & G \ar[r]^-{x^{n d}} \ar[d]^-{x^d} 
    & G \ar[r] \ar@{=}[d] 
    & 0 \\
  0 \ar[r] 
    & G_n \ar[r] 
    & G \ar[r]^-{x^n} 
    & G \ar[r] 
    & 0 \text{.} 
}\]
On a donc $\sK_n(\chi) = \sK_{n d}(\chi\circ x^d)$. 





\subsection{}\label{VI:4-8}

Appliquons la construction \ref{VI:4-6} pour $A=\dF_q$, et $M=k$ un alg\`ebre 
\'etale sur $\dF_q$. Conform\'ement aux conventions g\'en\'erales, on notera 
avec un indice $0$ les $\dF_q$-sch\'emas not\'es $V,V^\ast,\ldots$ en 
\ref{VI:4-6}. Les m\^emes lettres sans indice d\'esignent les sch\'emas sur 
$\dF$ qui s'en d\'eduisent par extension des scalaires. 

On pose $I=\hom(k,\dF)$, d'o\`u $V\sim \dG_a^I$ et $T$ s'\'ecrit 
$(x_i)\mapsto \sum x_i$. 
\begin{equation*}\tag{4.8.1}\label{VI:eq:4-8-1}
  \sF(\psi\circ \tr_{k/\dF_q}) = T^\ast \sF(\psi) = \bigotimes_i \operatorname{pr}_i^\ast \sF(\psi) \text{.} 
\end{equation*}
Puisque $V^\ast\sim \dG+m^I$, un caract\`ere $\chi$ de $V_n^\ast\sim \dmu_n^I$, 
\`a valeurs dans $E_\lambda^\times$, s'\'ecrit comme une famille 
$(\chi_i)_{i\in I}$ de caract\`eres de $\dmu_n$ ind\'en\'ee par $I$. Elle est 
d\'efinie sur $\dF_q$ si, pour tout $\sigma\in \gal(\dF/\dF_q)$, on a 
$\chi_{\sigma_i} = \chi_i\circ \sigma^{-1}$. Elle d\'efinit alors un 
$E_\lambda$-faisceau $\sK_n((\chi_i)_{i\in I})$ sur $V_0^\ast$. Si le produit 
des $\chi_i$ est trivial, $\chi$ se factorise par un caract\`ere de 
$(V_0^\ast/\dG_m)_n$ et $\sK_n((\chi_i)_{i\in I})$ est l'image r\'eciproque 
d'un faisceau, not\'e de m\^eme, sur $V-0^\ast/\dG_m$. Cette construction se 
compare comme suit au torseur de Lang. 





\begin{lemma_}\label{VI:4-9}
\begin{enumerate}[(i)]
  \item Pour $n$ assez divisible, on a un diagramme commutatif 
    \[\xymatrix{
      0 \ar[r] 
        & (V_0^\ast)_n \ar[r] \ar[d]^-\tau 
        & V_0^\ast \ar[r]^-{x^n} \ar[d]^-\tau 
        & V_0^\ast \ar[r] \ar@{=}[d] 
        & 0 \\
      0 \ar[r] 
        & k^\times \ar[r] 
        & V_0^\ast \ar[r]^-\fL 
        & V_0^\ast \ar[r] 
        & 0 \text{;} 
    }\]
    si $\chi$ est un caract\`ere $\chi:k^\times \to E_\lambda^\times$, on a 
    $\sF(\chi) = \sK_n(\chi\circ\tau)$. 
  \item Pour $n$ assez divisible, $\tau$ identifie $k^\times$ aux coinvariants 
    de $\gal(\bar\dF/F)$ agissant sur $V_n^\ast\sim \dmu_n^I$. 
  \item Pour $k$ un corps, $N=[k:\dF_q]$, $\omega\in I$ un plongement de $k$ 
    dans $\dF$, et $n=q^N-1$, la composante d'indice $\omega$ de 
    $\chi\circ\tau$ est $\chi\circ \omega^{-1}$. 
  \item Si $\chi$ est non trivial sur chaque facteur de $k$, les 
    $(\chi\circ \tau)_i$ sont tous non triviaux. 
\end{enumerate}
\end{lemma_}

Il suffit de prouver le lemme lorsque $k$ est un corps, de degr\'e $N$ sur 
$\dF_q$. Dans ce cas, $n$ est ``assez divisible'' si $q^N-1\mid n$. 
Choisissons un plongement $\omega$ de $k$ dans $\dF_q$; $I$ s'identifie alors 
\`a $\dZ/N$: \`a $i\in \dZ/N$ correspond $\omega_i=\omega^{q^i}$. Via 
l'isomorphisme $V_0^\ast(\dF) = {F^\times}^I$, on a 
\begin{enumerate}[\indent a)]
  \item $x\in k^\times$ correspond \`a $(\omega_i(x))\in {\dF^\times}^I$; 
  \item $F((x_i)_{i\in \dZ/N})=(x_{i-1}^q)_{i\in \dZ/N}$. 
\end{enumerate}

Un caract\`ere $\chi=(\chi_i)$ de $V_n^\ast\sim \dmu_n^I$ sera d\'efini sur 
$\dF_q$ si $\chi_{-i} =\chi_0(x^{q^i})$ ($i\in \dZ$). Si $q^N-1\mid n$, il y a 
$q^N-1$ tels caract\`eres: $\chi_0$ se factorise par $\dmu_{q^N-1}$, et 
d\'etermine les $\chi_i$. Pour prouver (ii), il suffit donc de v\'erifier (i) 
et (iii) (ou son corollaire (iv)) qui assure que $\chi\mapsto \chi\circ \tau$ 
est injectif. 

Prouvons (i) et (iii), pour $n=q^N-1$. Notons additivement le groupe des 
endomorphismes de $V^\ast\sim \dG_m^I$; en particulier, notons $n$ 
l'op\'erateur $x\mapsto x^n$. Si $\alpha$ est l'op\'erateur de permutations 
circulaire $(x_i)\mapsto (x_{i-1})$, on a 
$q^N-1=(q\alpha)^N-1 = (q\alpha-1)((q\alpha)^{N-1}+\cdots + 1)$. Ceci 
d\'etermine $\tau$: on a $\tau((x_i)) = \prod_{0\leqslant j<N} x_{i-j}^{q^j}$, 
et l'application induite de $(V_0^\ast)_n$ dans $k^\times$ s'\'ecrit [a) 
ci-dessus] $(x_i)\mapsto \prod x_{-i}^{q^i}$; de l\`a r\'esulte (iii). 





\begin{proposition_}\label{VI:4-10}
La cohomologie de $V^\ast$ \`a coefficients dans 
$\sF(\chi^{-1}\cdot \psi\tr_{k/\dF_q})$ v\'erifie 
\begin{enumerate}[\indent (i)]
  \item $\h_c^i = 0$ pour $i\ne N$, et $\dim \h_c^N=1$. 
  \item Sur $\h_c^N$, $F^\ast$ est la multiplication par 
    $\tau_{\dF_q}(\chi,\psi)$. 
  \item Si $\chi$ est non trivial sur chaque facteur de $k$, on a 
    $\h_c^\bullet \iso \h^\bullet$. 
\end{enumerate}
\end{proposition_}

Appliquons \ref{VI:4-9}(i): sur $\dF$, si $\chi\circ \tau=(\chi_i)_{i\in I}$, 
on a $\sF(\chi) = \sK_n((\chi_i)_{i\in I})$. Les points (i) et (iii) 
r\'esultant donc de l'\'enonc\'e plus g\'eom\'etrique suivant (pour (iii), 
appliquer \ref{VI:4-9}(iv)) et (ii) en r\'esulte par la formule des traces: 





\begin{proposition_}\label{VI:4-11}
Soit $(\chi_i)_{i\in I}$ une famille de caract\`ere de $\dmu_n(\dF)$. La 
cohomologie de $V^\ast\simeq \dG_m^I$ \`a coefficient dans 
$\sK_n((\chi_i)_{i\in I})\otimes \sF(\psi\tr_{k/\dF_q})$ v\'erifie 
\begin{enumerate}[\indent (i)]
  \item $\h_c^i{}=0$ pour $i\ne N$, et $\dim \h_c^N{} = 1$. 
  \item Si les $\chi_i$ sont tous non triviaux, on a 
    $\h_c^\bullet{}\iso \h^\bullet$. 
\end{enumerate}
\end{proposition_}

On a $V^\ast\sim \dG_m^I$, 
$\sK((\chi_i)) = \bigotimes_i \operatorname{pr}_i^\ast \chi_i(K_n(\dG_m))$ et 
$\sF(\psi\circ \tr_{\dF_q/k})=T^\ast \sF(\psi) = \bigotimes_i \operatorname{pr}_i^\ast \sF(\psi)$. 
Ceci permet d'appliquer la formule de K\"unneth, et \ref{VI:4-11} r\'esulte de 
\ref{VI:4-3}. 





\subsection{}\label{VI:4-12}

De \ref{VI:2-4*}* et (\hyperlink{VI}{Cycle}, \ref{VI:1-3} exemple 2), on tire 
aussi que le groupe des permutations $\sigma$ de $I$ telles que 
$\chi_i = \chi_{\sigma i}$ ($i\in I$) agit sur $\h_c^N$ par multiplication par 
la signature $\varepsilon(\sigma)$. Nous allons en d\'eduire une seconde preuve 
de l'identit\'e de Hasse-Davenport. 

Si $\chi$ est un caract\`ere de $\dF_q^\times$, le fait que, sur $\dF$, $N$ 
s'\'ecrit $(x_i)\mapsto \prod x_i$ fournit: 
$\sF(\chi\circ N)=N^\ast\sF(\chi) = \bigotimes_i \operatorname{pr}_i^\ast\sF(\psi)$ 
et la formule de K\"unneth fournit: 
\[
  \h_c^\bullet(V^\ast,\sF(\chi^{-1}\circ N,\psi\circ \tr)) \sim \h_c^\bullet(\dG_m,\sF(\chi^{-1}\psi))^{\otimes I} \text{,} 
\]
o\`u, au membre de droite, le produit tensoriel est pris au sens 
\ref{VI:2-4*}*. Puirsque $\h_c^1(\dG_m,\sF(\psi\chi^{-1}))$ est de dimension 
$1$, sa puissance tensorielle $\otimes I$, au sens ordinaire, ne d\'epend que 
du cardinal $n$ de $I$. Appliquant (\hyperlink{IV}{Cycle}, \ref{IV:1-3} exemple 
2), on obtient un isomorphisme canonique 
\begin{equation*}\tag{4.12.1}\label{VI:eq:4-12-1}
  \h_c^N\left(V,\sF(\chi^{-1}\circ N)\otimes T^\ast \sF(\psi)\right) \sim \h_c^1(\dG_m,\sF(\psi\chi^{-1})) \otimes_\dZ \textstyle \bigwedge^N \dZ^I \text{.}
\end{equation*}
Pour calculer l'action de $F^\ast$, le plus commode est d'adopter le point de 
vue galoisien et de dire que l'isomorphisme \eqref{VI:eq:4-12-1} \'etant 
canonique, il est compatible \`a l'action par transport de structure de 
$\gal(\dF/\dF_q)$. Si $\varepsilon(k)$ est la signature de la permutation 
$\varphi$ de $I$, on trouve que 
\begin{align*} 
  \tau_{\dF_q}(\chi\circ N_{k/\dF_q},\psi) 
    &= \tr\left(\varphi^{-1},\h_c^N\left(V^\ast,N^\ast\sF(\chi^{-1})\otimes T^\ast \sF(\psi)\right)\right) \\ 
    &= \varepsilon(k) \tr\left(\varphi^{-1},\h_c^1(\dG_m,\sF(\chi^{-1}\psi))\right)^N \\
    &= \varepsilon(k) \tau(\chi,\psi)^N \text{,} 
\end{align*}
soit 
\begin{equation*}\tag{4.12.2}\label{VI:eq:4-12-2}
  \tau_{\dF_q}\left(\chi\circ N_{k/\dF_q},\psi\right) = \varepsilon(k) \tau(\chi,\psi)^N \text{.} 
\end{equation*}

Si $k$ est un corps, $\varphi$ est une permutation circulaire de $I$, 
$\varepsilon(k) = (-1)^{N+1}$, et on retrouve l'identit\'e de Hasse-Davenport: 
\[
  \tau\left(\chi\circ N_{k/\dF_q},\psi\circ \tr_{k/\dF_q}\right) = (-1)^{N+1} \tau_{\dF_q}\left(\chi\circ N_{k/\dF_q},\psi\right) = \tau(\chi,\psi)^N \text{.} 
\]





\begin{lemma_}\label{VI:4-13}
Pour $\chi\circ \tau=(\chi_i)_{i\in I}$ comme en \ref{VI:4-9}(i), les
conditions suivantes sont \'equivalentes 
\begin{enumerate}[\indent (i)]
  \item $\chi| \dF_q^\times$ est trivial 
  \item Le produit des $\chi_i$ est trivial 
  \item $\sF(\chi) = \sK_n((\chi_i))$ est image r\'eciproque d'un (unique) 
    faisceau sur $V^\ast/\dG_m$. 
  \item L'image r\'eciproque de $\sF(\chi)$ sur $\dG_m$ (envoy\'e dans 
    $V^\ast$ \`a partir du morphisme structural $\dF_q \to k$) est triviale. 
\end{enumerate}
\end{lemma_}

(i)$\Rightarrow$(ii). On regarde $\chi$ comme une caract\`ere de 
$V^\ast/\dG_m(\dF_q)=k^\times/\dF_q^\times$, et on prend $\sF(\chi)$ sur 
$V^\ast/\dG_m$. 

(ii)$\Rightarrow$(iii). De m\^eme, on regarde $(\chi_i)$ comme un caract\`ere 
de $(V^\ast/\dG_m)_n$. L'unicit\'e dans (iii) r\'esulte de ce que $V^\ast$ est 
un fibr\'e \`a fibres connexes sur $V^\ast/\dG_m$, et (iii)$\Rightarrow$(iv)  
est trivial. 

(iv)$\Rightarrow$(i), (ii). Cette image r\'eciproque est 
$\sF(\chi|\dF_q^\times)$ et $\sK_n(\prod \chi_i)$. 





\subsection{}\label{VI:4-14}

Soient $k$ une alg\`ebre \'etale sur $\dF_q$, de 
dimension $N+1$ et $\chi$ un caract\`ere non trivial de $k^\times$, trivial sur 
$\dF_q^\times$. La somme de Jacobi $J(\chi)$ est d\'efinie par 
\begin{equation*}\tag{4.14.1}\label{VI:eq:4-14-1}
  J(\chi) = (-1)^{N-1} \sum_{\substack{x\in k^\times/\dF_q^\times \\ \tr(x)=0}} \chi^{-1}(x) \text{.} 
\end{equation*}

On a entre sommes de Gauss et sommes de Jacobi l'identit\'e suivante. 





\begin{proposition_}\label{VI:4-15}
Pour $\chi$ comme ci-dessus et $\psi$ un caract\`ere additif non trivial de 
$\dF_q$, on a 
\[
  q J(\chi) = \tau_{\dF_q}(\chi,\psi) \text{.} 
\]
\end{proposition_}

Dans le cas particulier o\`u $k=\dF_q^n$, cette formule se r\'ecrit par 
\eqref{VI:eq:4-5-3} 
\begin{equation*}\tag{4.15.1}\label{VI:eq:4-15-1}
  q J(\chi) = \prod_i \tau(\chi_i,\psi) \text{,} 
\end{equation*}
plus \'ecrit sous la forme 
\begin{align*}
  \chi_0(-1) J(\chi) &= (-1)^{N-1} \sum_{\substack{x_1,\dots,x_N\in \dF_q^\times \\ \sum x_i = 1}} \prod_{i=1}^N \chi_i^{-1}(x_i) = \tau(\chi_0^{-1},\psi)^{-1} \prod_{i=1}^N \tau(\chi_i,\psi) \\
  \chi_0^{-1} &= \prod_{i=1}^N \chi_i 
\end{align*}





\begin{proof}[Preuve de \ref{VI:4-15}]
\[
  \tau_{\dF_q}(\chi,\psi) = (-1)^{N+1} \sum_{x\in k^\times} \chi(x)^{-1} \psi \tr(x) = (-1)^{N+1} \sum_{x\in k^\times/\dF_q^\times} \chi(x)^{-1} \sum_{\lambda\in \dF_q^\times} \psi(\lambda \tr x) \text{.}
\]
La somme $\sum_{\lambda\in \dF_q^\times} \psi(\lambda \tr x)$ vaut $q-1$ si 
$\tr(x)=0$, et $-1$ si $\tr(x)\ne 0$. D\`es lors, 
\[
  \tau_{\dF_q}(\chi,\psi) = (-1)^{N+1} \left(q\sum_{x\in k^\times/\dF_q^\times} \chi(x)^{-1} - \sum_{x\in k^\times/\dF_q^\times} \chi(x)^{-1}\right) \text{.}
\]
La second terme au second membre est nul, car somme des valeurs d'un 
caract\`ere, et \ref{VI:4-15} en r\'esulte. 
\end{proof}

La proposition \ref{VI:4-15} se transpose ainsi en cohomologie: 





\begin{proposition_}\label{VI:4-16}
La cohomologie de $W^\ast/\dG_m$ (\ref{VI:4-6}, \ref{VI:4-8}) \`a coefficient 
dans $\sF(\chi^{-1})$ v\'erifie 
\begin{enumerate}[\indent (i)]
  \item $\h_c^i=0$ pour $i\ne N-1$, et $\dim \h_c^{N-1}{}=1$. 
  \item Sur $\h_c^{N-1}$, $F^\ast$ est la multiplication par $J(\chi)$. 
  \item $\h_c^{N-1}$ est canoniquement isomorphe \`a 
    $\h_c^{N+1}\left(V^\ast,\sF(\chi^{-1},\psi\circ \tr_{k/\dF_q})\right)(1)$. 
\end{enumerate}
\end{proposition_}

L'assertion (ii) r\'esulte de (i) et de la formule des traces; (i) et (iii) 
r\'esultent de \ref{VI:4-11} et de l'\'enonc\'e plus g\'eom\'etrique suivant. 





\begin{proposition_}\label{VI:4-17}
Soit $(\chi_i)_{i\in I}$ une famille de caract\`eres non trous triviaux de 
$\dmu_n(\dF)$, de produit $1$. Alors, 
$\h_c^{i-1}\left(W^\ast/\dG_m,\sK_m((\chi_i)_{i\in I})\right)$ est 
canoniquement isomorphe \`a 
\newline
$\h_c^{i+1}\left(V^\ast,\sK_m((\chi_i)_{i\in I})\otimes \sF(\psi\tr_{k/\dF_q})\right)(1)$. 
\end{proposition_}

La premi\`ere ligne de \ref{VI:4-15} devient ($\pi$ comme en 
\eqref{VI:eq:4-6-1}). 
\begin{equation*}\tag{4.17.1}\label{VI:eq:4-17-1}
  \eR^i \pi_!\left(\sK_n((\chi_i)_{i\in I})\otimes \sF(\psi \tr_{k/\dF_q})\right) = \sK_n((\chi_i)_{i\in I}) \otimes \eR^i \pi_! \sF(\psi\tr_{k/\dF_q}) 
\end{equation*}
(sur $V^\ast/\dG_m$). Calculons les faisceaux 
$\eR^i\pi_! \sF(\psi\tr_{k/\dF_q}) = \eR^i \pi_! T^\ast \sF(\psi)$. Soit 
$v_0:\widetilde V_0 \to V_0$ l'\'eclat\'e de $V_0$ en $\{0\}$. Dans le 
diagramme 
\begin{equation*}\tag{4.17.2}\label{VI:eq:4-17-2}
\xymatrix{
  V_0\setminus \{0\} \ar@{^{(}->}[r] \ar[dr]_-\pi 
    & \widetilde V_0 \ar[r]^-{v_0} \ar[d]^-{\bar\pi} 
    & V_0 \\
  & P_0 
}
\end{equation*}
$\pi$ est une fibration de fibre des droites \'epoint\'ees, $\widetilde V_0$ 
est le fibr\'e en droites correspondant, et $Z_0=v_0^{-1}(0)$ sa section $0$. 
Le faisceau $T^\ast\sF(\psi)$ sur $V-0\setminus \{0\}$ se prolonge en 
$(T v_0)^\ast \sF(\psi)$ sur $\widetilde V_0$, et la restriction de ce faisceau 
\`a $Z_0$ est le faisceau constant $E_\lambda$, car $T_0 v_0$ est nul sur 
$Z_0$. La suite exacte longue de cohomologie \`a support propre s'\'ecrit 
donc 
\begin{equation*}\tag{4.17.3}\label{VI:eq:4-17-3}
\xymatrix{
  \ar[r] 
    & \eR^i \pi_! T^\ast \sF(\psi) \ar[r] 
    & \eR^i \bar\pi_! (T v)^\ast \sF(\psi) \ar[r] 
    & (E_\lambda\text{ pour $i=0$, $0$ sinon}) \ar[r] 
    & 
}
\end{equation*}

L'image r\'eciproque $\bar\pi^{-1}(Q_0)$ est l'\'eclat\'e $\widetilde W_0$ de 
$W_0$ en $0$; $T_0 V_0$ s'annule sur $\widetilde W_0$; on a donc 
$(T v)^\ast \sF(\psi)=E_\lambda$ sur $\pi^{-1}(Q_0)$ et, $\bar\pi$ \'etant un 
fibr\'e en droites 
\begin{equation*}\tag{4.17.4}\label{VI:eq:4-17-4}
  \eR^i \bar\pi_! (T v)^\ast \sF(\psi)|Q = 
    \begin{cases}
      0 & \text{pour $i\ne 2$} \\
      E_\lambda(-1) & \text{pour $i=2$.} 
    \end{cases} 
\end{equation*}

Si $x\in P$, $x\notin Q$, la droite $D=\pi^{-1}(x)$ est envoy\'e 
isomorphiquement sur $\dG_a$ par $T v$; on a donc 
\ref{VI:2-7}*) 
\begin{align} \notag 
  \h_c^\bullet(D,(T v)^\ast \sF(\psi)) &= \h^\bullet(\dG_a,\sF(\psi)) = 0 && \text{et} \\
  \tag{4.17.5}\label{VI:eq:4-17-5}
  \eR^i \bar\pi_! (T v)^\ast \sF(\psi) 
    &= \begin{cases} 
         0 & \text{pour $i\ne 2$} \\
         E_\lambda(-1)_Q & \text{pour $i=2$} 
       \end{cases} 
\end{align}

Conjuguant \eqref{VI:eq:4-17-3} et \eqref{VI:eq:4-17-5}, on trouve enfin 
\begin{equation*}\tag{4.17.6}\label{VI:eq:4-17-6}
  \eR^i \pi_! T^\ast\sF(\psi) = 
    \begin{cases}
      0 & \text{pour $i\ne 1,2$} \\
      E_\lambda & \text{pour $i=1$} \\
      E_\lambda(-1)_Q & \text{pour $i=2$.} 
    \end{cases}
\end{equation*}





\subsection{}\label{VI:4-18}

Calculons la cohomologie \`a support propre du faisceau 
$\sK_n((\chi_i))\otimes T^\ast \sF(\psi)$ sur $V^\ast$ \`a l'aide de la suite 
spectrale de Leray de $\pi:V^\ast\to V^\ast/\dG_m$. Appliquant 
\eqref{VI:eq:4-17-1} et \eqref{VI:eq:4-17-6}, on trouve comme termes initiaux 
les $E_2^{p,1} = \h_c^p(V^\ast/\dG_m,\sK_n(\chi_i)) = 0$ (car $(\chi_i)\ne 0$) 
et les $E_2^{p,2} = \h_c^p(W^\ast/\dG_m,\sK(\chi_i))(-1)$. 

La suite spectrale se r\'eduit \`a un isomorphisme, et \ref{VI:4-17} en 
r\'esulte. 





\subsection{}\label{VI:4-19}

Les faisceaux $\sK_n((\chi_i)_{i\in I})$ ont un sens sur n'importe quel corps, 
voire sur n'importe sch\'ema de base. Ceci va nous permettre de g\'en\'eraliser 
\ref{VI:4-16}(i). Avec les notations de \ref{VI:4-6}, supposons $M$ fini 
\'etale partout de rang $N$ sur $A$, et soit $n$ un entier inversible dans $A$. 
On note $a$ la projection de $W^\ast/\dG_m$ sur $\spec(A)$. Soit aussi $\chi$ 
un caract\`ere (morphisme de faisceaux) 
$\chi:(V^\ast/\dG_m)_n \to E_\lambda^\times$, et $\sK_n(\chi)$ le faisceau 
correspondant. Localement pour la topologie \'etale, on peut regarder $\chi$ 
comme une famille de caract\`eres $(\chi_i)_{i\in I}$ de $\dmu_n$, de produit 
trivial. 





\begin{proposition_}\label{VI:4-20}
\begin{enumerate}[(i)]
  \item Si $\chi$ est (en tout point) non trivial, on a 
    $\eR^i a_! (\sK_n(\chi)) = 0$ pour $i\ne N-1$, et 
    $\eR^{N-1} a_!(\sK_n(\chi))$ est lisse, de rang $1$. 
  \item Une automorphisme $\sigma$ de $M$ qui respecte $\chi$ agit sur ce 
    $\eR^{N-1} a_!$ par multiplication par la signature $\varepsilon(\sigma)$ 
    de $\sigma$, vu comme permutation de $I$. 
  \item Si les $\chi_i$ sont tous non triviaux, on a $\eR a_! \iso \eR a_\ast$. 
\end{enumerate}
\end{proposition_}
\begin{proof}[Preuve]
Le sch\'ema $W^\ast/\dG_m$ est le compl\'ement d'un diviseur \`a croisements 
normaux relatif dans le sch\'ema $Q$, propre et lisse sur $\spec(A)$, et 
$\sK_n(\chi)$ est localement constant sur $W^\ast/\dG_m$, \`a ramification 
mod\'er\'ee a l'infini. Il en r\'esulte que les $\eR^i a_!$ et 
$\eR^i a_\ast$ sont lisses, de formation compatible \`a tout changement de 
base. Un argument standard nous ram\`ene alors \`a supposer que $A$ est un 
corps fini, et (i) r\'esulte de \ref{VI:4-17} et \ref{VI:4-11}(i). Pour 
prouver (ii), on utilise que l'action de $\sigma$ est compatible \`a 
l'isomorphisme \ref{VI:4-17}, et \ref{VI:4-12}. Pour (iii), on note que si les 
$\chi_i$ sont tous non triviaux, alors $\sK_n(\chi)$ sur 
$W^\ast/\dG_m\subset Q$ est ramifi\'e le long de chaque diviseur \`a l'infini, 
et \ref{VI:1-19-1}. 
\end{proof}










\section{Caract\`eres de Hecke}\label{VI:5}





\subsection{}\label{VI:5-1}

Soient $F$ un corps de nombres (de degr\'e fini sur $\dQ$) et $k$ un corps de 
caract\'eristique $0$. Voici diverses façon \'equivalentes de dire ce qu'est un 
homomorphisme \emph{alg\'ebrique} $\alpha:F^\times \to k^\times$. 
\begin{enumerate}[(i)]
  \item Soit $\{e_i\}$ une base de $F$ sur $\dQ$. L'homomorphisme $\alpha$ est 
    alg\'ebrique s'il est donn\'e par une formule 
    $\alpha(\sum x^i e_i) = A(x^i)$, $A\in k(X^i)$. 
\end{enumerate}

Ceci signifie que $\alpha$ coïncide sur $F^\times$ avec une application 
rationnelle d\'efinie sur $k$ de $\res_{F/\dQ}(\dG_m)$ dans $\dG_m$. Par 
densit\'e de Zariski de $F^\times$, et le fait qu'un homomorphisme birationnel 
est partout d\'efini, une telle application est un homomorphisme de sch\'emas 
en groupes: 
\begin{enumerate}[(i)]
\setcounter{enumi}{1}
  \item $\alpha$ est induit par un homomorphisme de $k$-sch\'emas en groupes 
    \[
      \res_{F/\dQ}(\dG_m)\otimes_\dQ k \to \dG_m \text{.} 
    \] 
\end{enumerate}

Si $k$ est un corps de nombre, la propri\'et\'e d'adjonction de la restriction 
des scalaires $R$ montre que ceci \'equivaut \`a 
\begin{enumerate}[(i)]
\setcounter{enumi}{2}
  \item ($k$ un corps de nombres) $\alpha$ est induit par un homomorphisme de 
    $\dQ$-sch\'emas en groupes: $\res_{F/\dQ}(\dG_m) \to \res_{k/\dQ}(\dG_m)$. 
\end{enumerate}

Soient $\bar k$ un cl\^oture alg\'ebrique de $k$, et $I=\hom(F,\bar k)$. Sur 
$\bar k$, le groupe des caract\`eres de $\res_{F/\dQ}(\dG_m)$ est $\dZ^I$, de 
base les plongements de $F$ dans $\bar k$. Les caract\`eres d\'efinis sur $k$ 
sont ceux invariants par $\gal(\bar k/k)$; on peut les d\'ecrire soit comme 
ceux envoyant $F^\times$ dans $k^\times\subset \bar k$, soit en terme des 
orbites de $\gal(\bar k/k)$ dans $I$, correspondant elle-m\^eme aux facteurs de 
$F\otimes k$. 
\begin{enumerate}[(i)]
\setcounter{enumi}{3}
  \item $\alpha$ est de la forme 
    $\alpha=\prod_{\omega\in I} \omega^{n_\omega}$. Les familles d'exposants 
    $(n_\omega)$ permises sont celles telles que $n_\omega=n_{\omega'}$ pour 
    $\omega$ et $\omega'$ dans la m\^eme orbite de $\gal(\bar k/k)$. Ce sont 
    encore celles telle que $\prod_\omega (x)^{n_\omega}\in k$ pour tout 
    $x\in F$. 
  \item Posons $F\otimes k=\prod_{j\in J} F_j$, les $F_j$ \'etant des corps. 
    Alors, $\alpha$ s'\'ecrit $\alpha=\prod N_{F_j/k}^{m_j}$. 
\end{enumerate}

Dans le cas particulier o\`u $k$ contient une cl\^oture normale de $F$, ces 
expressions deviennent: $\alpha=\prod \omega^{n_\omega}$, o\`u $\omega$ 
parcourt les $[F:\dQ]$ plongements de $F$ dans $k$. 





\subsection{}\label{VI:5-2}

Supposons que $k$ soit un corps de nombres, et soit $\alpha$ un homomorphisme 
alg\'ebrique de $F^\times$ dans $k^\times$. Il existe alors un et un seul 
homomorphisme, encore not\'e $\alpha$, du groupe $I(F)$ des id\'eaux 
fractionnaires de $F$ dans celui de $k$, tel que $\alpha((x))=(\alpha(x))$. 
L'unicit\'e r\'esulte de ce que tout id\'eal $a$ une puissance qui est un 
id\'eal principal, et de ce que $I(k)$ est sans torsion. Pour prouver 
l'existence, on utilise par exemple \ref{VI:5-1}(v): on a 
$\alpha(a) = \prod N_{F_j/k}^{m_j}((a))$. 





\subsection{}\label{VI:5-3}

Rappelons la d\'efinition des caract\`eres de Hecke alg\'ebriques, appel\'es 
Weil caract\`eres de Hecke (ou: gr\"ossencharaktere) de type $A_0$. Soient $F$ 
un corps de nombres, $\fm$ un id\'eal de $F$ (i.e., de l'anneau des entiers de 
$F$), $I_\fm$ le groupe des id\'eaux fractionnaires de $F$ premiers \`a $\fm$, 
et $k$ un corps de caract\'eristique $0$. Un homomorphisme 
$\chi:I_\fm \to k^\times$ est un \emph{caract\`ere de Hecke alg\'ebrique} (de 
conducteur $\leqslant \fm$) s'il existe un homomorphisme alg\'ebrique 
$\chi_\text{alg}:F^\times \to k^\times$ v\'erifiant 
\begin{equation*}\tag{$*$}\label{VI:eq:5-3-1}
\text{Pour $x\in F^\times$, premier \`a $\fm$, totalement positif et 
$\equiv 1\pmod\fm$, on a $\chi((x))=\chi_\text{alg}(x)$.}
\end{equation*}

Par densit\'e de l'ensemble des $x$ de \eqref{VI:eq:5-3-1}, $\chi_\text{alg}$ 
est enti\`erement d\'etermin\'e par $\chi$. C'est la \emph{partie alg\'ebrique} 
de $\chi$. Si $\chi((x))=\chi_\text{alg}(x)$ pour $x$ totalement positif et 
$\equiv 1\pmod{\fm'}$, $\chi$ se prolonge en un caract\`ere de Hecke de 
conducteur $\leqslant \inf(\fm,\fm')$: $I_{\fm+\fm'}\to k^\times$. On 
identifiera les caract\`eres de Hecke qui coïncident sur leur domaine commun de 
d\'efinition, et on appelle \emph{conducteur} de $\chi$ le plus petit $\fm'$ 
tel que $\chi$ soit de conducteur $\leqslant \fm'$. 





\subsection{Remarque}\label{VI:5-4}

Si un caract\`ere de Hecke alg\'ebrique $\chi$ prend ses valeurs dans un 
sous-corps $k'$ de $k$, c'est d\'ej\`a un caract\`ere de Hecke \`a valeurs dans 
$k'$: il suffit de voire que $\chi_\text{alg}:\res_{F/\dQ}(\dG_m) \to \dG_m$ 
est d\'ej\`a d\'efini sur $k'$, et ceci r\'esulte de la densit\'e de Zariski 
dans $\res_{F/\dQ}(\dG_m)$ de l'ensemble des $x$ totalement positif 
$\equiv 1\pmod\fm$. On pour toujours prendre pour $k'$ un sous-corps de $k$ de 
degr\'e fini sur $\dQ$. 





\subsection{}\label{VI:5-5}

Si $\varepsilon$ est une unit\'e totalement positive $\equiv 1\pmod\fm$, on a 
$\chi_\text{alg}(\varepsilon) = \chi((\varepsilon)) = 1$. L'homomorphisme 
$\chi_\text{alg}$ se factorise donc par le quotient $T_\fm$ de 
$\res_{F/\dQ}(\dG_m)$ par l'adh\'erence de Zariski du groupe 
$E_\fm\subset F^\times$ des unit\'es totalement positifs $\equiv 1\pmod\fm$. 

D'apr\`es Serre \cite[II.3]{se68}, si $k$ est une cl\^oture alg\'ebrique de 
$\dQ$, et que $\fm$ est assez grand, les caract\`eres $\prod \omega^{n_\omega}$ 
de $\res_{F/\dQ}(\dG_m)$ de la forme $\chi_\text{alg}$ sont caract\'eris\'es 
comme suit: il doit exister un entier $N$, le \emph{poids} de $\chi$ (ou de 
$\chi_\text{alg}$), tel que pour tout \'el\'ement $\sigma$ de $\gal(k/\dQ)$ 
conju\'e \`a la conjugaison complexe, on ait $n_\omega+n_{\sigma \omega}=N$. Si 
$F_1$ est le plus grand sous-corps de $F$ qui soit une extension quadratique 
totalement imaginaire d'un corps totalement r\'eel $F_1'$, cela revient \`a 
dire que $\chi_\text{alg}$ est de la forme $\chi_1\circ N_{F/F_1}$, et que 
$\chi_1|F_1=(N_{F_1/\dQ})^N$. 

Si $\chi$ est de poids $N$, pour tout id\'eal $\fa$ de $F$, $\chi(\fa)$ est un 
nombre alg\'ebrique dont tous les conjugu\'es sont de valeur absolue 
$N(\fa)^{N/2}$ \cite[II.3 prop.2]{se68}. 





\subsection{}\label{VI:5-6}

Pour $k$ un corps de nombres fix\'e, des conditions suppl\'ementaires sont 
impos\'ees \`a $\chi_\text{alg}$. Pour tout id\'eal $\fa$ de $F$ premier au 
conducteur, on a en effet 
\begin{equation*}\tag{5.6.1}\label{VI:eq:5-6-1}
  \chi_\text{alg}(\fa) = (\chi(\fa)) \text{,} 
\end{equation*}
de sorte que $\chi_\text{alg}(\fa)$ est principal. Puisque le groupe des 
id\'eaux fractionnaire de $k$ est sous torsion, il suffit de prouver la 
puissance $n$-i\`eme de \eqref{VI:eq:5-6-1}, $n\ne 0$ convenable; ceci permet 
de remplacer $\fa$ par $\fa^n$, donc de supposer $\fa=(x)$, avec $x$ totalement 
positif $\equiv 1\pmod\fm$. Il ne reste qu'\`a utiliser les d\'efinitions. 

Ceci, joint \`a \ref{VI:5-5}, montre que $\chi_\text{alg}$ d\'etermine la norme 
des $\chi(\fa)$ en toutes les places de $k$. 





\subsection{}\label{VI:5-7}

Le groupe $S_\fm$ de Serre pourrait \^etre caract\'eris\'e comme \'etant le 
groupe de type multiplicatif dont le groupe des caract\`eres (sur n'importe 
quel corps) est le groupe des caract\`eres de Hecke alg\'ebriques de conducteur 
$\leqslant \fm$. (cf. \cite[II 2.1 et 2.2]{se68}). Sa relation avec les 
repr\'esentations $\ell$-adiques est expliqu\'ee en \cite[II 2.3]{se68}. 





\begin{theorem_}[{\cite{se68}}]\label{VI:5-8}
Soit $\chi$ un caract\`ere de Hecke alg\'ebrique de $F$ dans $E_\lambda$, pour 
$E_\lambda$ une extension finie de $\dQ_\ell$. Il existe alors un (et un seul) 
homomorphisme $\chi_\lambda:\gal(\bar F/F)^\textnormal{ab} \to E_\lambda$, tel 
que 
\begin{enumerate}[\indent (i)]
  \item $\chi_\lambda$ est non ramifi\'e en dehors du conducteur $\ff$ de 
    $\chi$ et de $\ell$. 
  \item Pour $\fp$ un id\'eal premier de $F$ premier \`a $\ff$ et \`a $\ell$, 
    et $F_\fp\in \gal(\bar F/F)^\textnormal{ab}$ le Frobenius g\'eom\'etrique 
    en $\fp$, on a 
    \[
      \chi_\lambda(F_\fp) = \chi(\fp) \text{.}
    \] 
\end{enumerate}
\end{theorem_}











\chapter{Théorèmes de finitude en cohomologie \texorpdfstring{$\ell$}{l}-adique}\label{VII}

\section{Énoncé des théorèmes}\label{VII:1}

\begin{theorem_}\label{VII:1-1}
foo
\end{theorem_}





\include{sga4.5-ch8}





\appendix
\chapter{Erratum pour SGA 4, tome 3}

\paragraph{XIV p.18 1.14}
(XIX 6) au lieu de (XX 6). 

\paragraph{XVI 2.2}
Il faut supposer $F$ localement constant!

\paragraph{XVI 5.2}
La démonstration donnée est incomplète. Arpès l'énoncé de 
l'hypothèse de récurrence, il faut d'abord se ramener au cas o\`u $F$ est 
constant ($F$ devient constant sur un rev\^etement galoisien étale 
$\pi:X'\to X$ de $X$, de groupe de Galois $G$, et on invoque la suite spectrale 
de Hochschild-Serre $E_2^{p q}=\h^p(G,\h^q(X',\pi^\ast F))\Rightarrow \h^{p+q}(X,F)$). 
Les arguments qui suivent sont alors corrects. 

\paragraph{XVII 1.1.8}
Le signe (1.1.8.1) est erroné lorsque $F$ est contravariant en certaines 
variables. Il faut lire: 
\begin{equation*}\tag{1.1.8.1}
  \rho^{\underline k} = (-1)^{A(\underline k)}: \text{ automorphisme de }
F\circ (G_j)\left(K_i^{k_i\varepsilon(i)\varepsilon(\psi(i))}\right) 
\end{equation*}
avec 
\[
  A(\underline k) = \sum_{\varepsilon(i) = \varepsilon(\psi(i))=-} k_i + \sum_{\varepsilon(j)=-} \sum_{\substack{\psi(a)=\psi(b)=j \\ a<b}} k_a k_b 
\]

\paragraph{XVII 2.1.3}
La démonstration contient des erreurs flagrantes. Il faut supprimer la 
3-ème ligne (p.34 1.9) et remplacer les 6-ème et 7-ème (p.34 1.12, 13) 
par: 

Les flèches du diagramme (2.1.3.2) induisent des applications 
\[
  \hom_y(F,F) \to \hom_{f g'}({g'}^\ast F,f_\ast F) = \hom_{x'}(f'_\ast {g'}^\ast F,g^\ast f_\ast F) \text{,}
\]

\paragraph{XVIII 2.14.4}
Lire XVII 6.2.7.2 au lieu de XVII 6.2.4.

\paragraph{XVIII p.99 1-1}
Lire $u$ au lieu de $U$ et 3.1.16.1 au lieu de 3.1.11.1.






% to use BibTeX
\bibliographystyle{amsplain}
\bibliography{sga-sources}






\end{document}
