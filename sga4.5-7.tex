% !TEX root = sga4.5.tex


\chapter{Théorèmes de finitude en cohomologie \texorpdfstring{$\ell$}{l}-adique}\label{VII}










\section{Énoncé des théorèmes}\label{VII:1}

Dans tout cet expos\'e, $S$ sera un sch\'ema noeth\'erien et $A$ un anneau 
noeth\'erien \`a gauche, de torsion annul\'e par un entier inversible sur $S$. 
Notre r\'esultat principal est le suivant. 





\begin{theorem_}\label{VII:1-1}
On suppose $S$ r\'egulier de dimension $0$ ou $1$. Soient $f:X\to Y$ un 
morphisme de $S$-sch\'emas de type fini et $\sF$ un faisceau constructible de 
$A$-modules \`a gauche sur $X$. Alors, les faisceaux $\eR^i f_\ast \sF$ sont 
constructibles. 
\end{theorem_}

La preuve sera donn\'ee au paragraphe \ref{VII:2} et en \ref{VII:3-10}. 





\subsection{Remarque}\label{VIII:1-2}

En caract\'eristique $0$, ce r\'esultat est moins g\'en\'eral que 
\cite[XIX paragraphe 5]{sga4}, qui prouve la conclusion du th\'eor\`eme pour 
tout morphisme de type fini de sch\'emas excellents de caract\'eristique $0$. 
La d\'emonstration de \cite[XIX]{sga4} utilise d'une part la r\'esolution des  
singularit\'es, d'autre part que les sch\'emas soient d'\'egale 
caract\'eristique (pour prouvoir d\'eduire de la r\'esolution le ``th\'eor\`eme 
de pureté''). 






\subsection{Remarque}\label{VII:1-3}

Nous dirons qu'un complexe $K\in\ob\eD(X,A)$ est \emph{constructible} si ses 
faisceaux de cohomologie sont constructibles et nous noterons avec un indice 
$c$ la sous-cat\'egorie de $\eD(X,A)$ (ou $\eD^+$, $\eD^-$, $\eD^b$) form\'ee 
des complexes constructibles. Dans le langage des cat\'egories d\'eriv\'ees, 
que nous utiliserons librement, \ref{VII:1-1} dit que 
$\eR f_\ast:\eD^+(X,A) \to \eD^+(Y,A)$ envoie $\eD_c^+$ dans $\eD_c^+$. 
Explicitons: 
\begin{enumerate}[\indent a)]
  \item Pour $K$ r\'eduit \`a un faisceau $\sF$ en degr\'e $0$, on a 
    $\sH^i\eR f_\ast K = \eR^i f_\ast \sF$; l'\'enonc\'e d\'eriv\'e implique 
    donc le th\'eor\`eme. 
  \item Dans l'autre sens, on invoque la suite spectrale 
    \[
      E_1^{p q} = \eR^q f_\ast \sH^p(K) \Rightarrow \sH^{p+q} \eR f_\ast K \text{.} 
    \]
\end{enumerate}

Parler le langage des cat\'egories d\'eriv\'ees \`a l'avantage de remplacer par 
une simple formule de transitivit\'e $\eR(f g)_\ast = \eR f_\ast \eR g_\ast$ 
une suite spectrale de Leray 
$E_2^{p q} = \eR^p f_\ast \eR^q g_\ast \Rightarrow \eR^{p+q} (f g)_\ast$. 

Sous les hypoth\`eses de \ref{VII:1-1}, il r\'esulte de \cite{sga4} que 
$\eR f_\ast$ est de dimension cohomologique finie. Cela permet ci-dessus de 
remplacer $\eD^+$ par $\eD$, $\eD^-$ ou $\eD^b$. 





\subsection{Les id\'ees de la d\'emonstration}\label{VII:1-4}
\begin{enumerate}[\indent a)]
  \item La dualit\'e de Poincar\'e permet de traiter le cas o\`u $X$ est 
    lisse, $\sF$ localement constant, et $Y=S$. L'hypoth\`ese de dimension 
    appara\"it par calculer des $\rHom$ sur $S$.
  \item On factorise $f$ en $g j$: $g$ propre et $j$ plongement ouvert. On a 
    $\eR f_\ast = \eR g_\ast \eR j_\ast$ et le th\'eor\`eme de finitude pour 
    les morphismes propres contr\^ole $\eR g_\ast$. D\'evissant, on se ram\`ene 
    \`a supposer $x$ lisse, $\sF$ localement constant, $f$ un plongement ouvert 
    et $Y$ propre sur $S$. 
  \item Une r\'ecurrence sur $\dim X$ (avec changement de $S$) permet, grosso 
    modo, de supposer le th\'eor\`eme vrai en dehors d'une partie de $Y$ finie 
    sur $S$. Notant $a:X\to S$ et $b:Y\to S$ les morphismes structuraux, on 
    montre alors que si le th\'eor\`eme \'etait faux pour $\sF$ et 
    $\eR f_\ast$, il le serait aussi pour $\sF$ et 
    $\eR b_\ast \eR f_\ast = \eR a_\ast$: contradiction. 
\end{enumerate}





\begin{corollary_}\label{VII:1-5}
Sous les hypoth\`eses du th\'eor\`eme, les cat\'egories $\eD_c(X,A)$ et 
$\eD_c(Y,A)$ sont transform\'ees l'une en l'autre par les $4$ op\'erations 
$\eR f_\ast$, $\eR f_!$, $f^\ast$, $\eR f^!$. 
\end{corollary_}

$\eR f_\ast$ est trait\'e ci-dessus, $\eR f_!$ en \cite[XVII 5.3]{sga4}, 
$f^\ast$ est clair. Reste $\eR f^!$. Le probl\`eme est local, ce qui permet de 
supposer que $f$ se factorise en $X\xrightarrow i Z \xrightarrow g Y$ ($i$ 
plongement ferm\'e et $g$ lisse, purement de dimension relatif $n$). La 
dualit\'e de Poincar\'e $\eR g_! K(n)[2 n]$ et la transitivit\'e 
$\eR f^! = \eR i^! \eR g^!$ nous ram\`enent \`a prouver \ref{VII:1-5} pour $i$. 
Pour tout $L\in \eD(Z,A)$, si $j$ est l'inclusion dans $Z$ de $U=Z\setminus X$, 
$i_\ast \eR i^! L$ est le mapping cylinder de $K\to \eR j_\ast j^\ast K$ et 
\ref{VII:1-5} pour $i$ r\'esulte de \ref{VII:1-1} pour $j$. 





\begin{corollary_}\label{VII:1-6}
Soit $X$ comme dans le th\'eor\`eme, et supposons $A$ commutatif. Alors, si 
$\sF$ et $\sG$ sont des faisceaux constructibles de $A$-modules sur $X$, les 
$\Ext_A^i(\sF,\sG)$ sont constructibles: $\rHom$ envoie 
$\eD_c^-(X,A)\times \eD_c^+(X,A)$ dans $\eD_c^+(X,A)$. 
\end{corollary_}

Par d\'evissage de $\sF$, on se ram\`ene \`a supposer $\sF$ de la forme 
$j_!\sF_1$ ($j:Y\hookrightarrow X$) un plongement localement ferm\'e et $\sF_1$ 
localement constant sur $Y$). On a alors 
\[
  \rHom(j_! \sF_1,\sG) = \eR j_\ast \Hom(\sF_1,\eR j^! \sG) \text{:} 
\]
d'apr\`es \ref{VII:1-3}, nous sommes ramen\'es au cas o\`u $\sF$ est localement 
constant -- voire constant puisque le probl\`eme est local. Pour $\sF$ 
localement constant constructible, on a $\Hom(\sF,\sG)_x=\hom(\sF_x,\sG_x)$, 
et de m\^eme pour $\rHom$. Pour $\sF$ constant, on peut donc calculer les 
$\Ext^i$ \`a l'aide d'une r\'esolution projective de type fini de sa valeur 
constante, et les $\Ext^i$ sont constructibles si $\sG$ l'est -- d'o\`u le 
corollaire. 





\subsection{Remarque}\label{VII:1-7}

Puisque $\eR f_\ast$ est de dimension cohomologique finie, il transforme 
complexes de tor-dimension finie (resp. $\leqslant d$) en complexes de 
tor-dimension finie (resp. $\leqslant d$) \cite[XVII 5.2.11]{sga4}. 
\cite[XVII 5.2.10]{sga4} et la preuve de \ref{VII:1-5} montrent alors que la 
tor-dimension finie est stable par les $4$ op\'erations $\eR f_\ast$, 
$\eR f_!$, $f^\ast$, $\eR f^!$. Une variant de celle de \ref{VII:1-6} 
(d\'evisser $K$ selon une partition de $x$ pour supposer les faisceaux de 
cohomologie localement constants, puis se localiser pour remplacer $K$ par un 
complexe fini de faisceaux localement libres de type fini) montre alors que, 
pour $A$ commutatif, $\rHom$ induit 
\[
  \rHom:\eD_\text{ctf}^b(X,A)\times \eD_\text{tf}^b(X,A) \to \eD_\text{tf}^+(X,A) 
\]
(t.f.: tor-dimension finie). 





\subsection{}\label{VII:1-8}

Sous les hypoth\`eses du th\'eor\`eme, la m\^eme m\'ethode nous permet de 
prouver un th\'eor\`eme de bidualit\'e locale $K\iso D D K$ (paragraphe 
\ref{VII:4}). Nous prouvons aussi un th\'eor\`eme de finitude pous les 
faisceaux de cycles \'evanescents. Pour $S$ quelconque, on obtient encore un 
th\'eor\`eme ``g\'en\'erique'': 





\begin{theorem_}\label{VII:1-9}
Soit $f:X\to Y$ un morphisme de $S$-sch\'emas de type fini et $\sF$ un faisceau 
constructible de $A$-modules sur $X$. Il existe un ouvert dense $U$ de $S$ tel 
que 
\begin{enumerate}[\indent (i)]
  \item Au-dessus de $U$, les $\eR^i f_\ast \sF$ sont constructibles, et nuls 
    sauf un nombre fini d'entre eux. 
  \item La formation des $\eR^i f_\ast \sF$ est compatible \`a tout 
    changement de base $S' \to U\subset S$. 
\end{enumerate}
\end{theorem_}

Pour $S$ le spectre d'un corps, on a $U=S$. 

Si $S$ est le spectre d'un corps, un argument de passage \`a la limite fournit 
la compatibilit\'e aux $S$-changements de base des $\eR^i f_\ast$ pour tout 
morphisme quasi-compact quasi-s\'epar\'e de $S$-sch\'emas et tout faisceau 
$\sF$. 





\begin{corollary_}\label{VII:1-10}
Soient $x$ un sch\'ema de type fini sur $k$ s\'eparablement clos et $\sF$ un 
faisceau constructible de $A$-modules. Alors, les $\h^i(X,\sF)$ sont de type 
fini. 
\end{corollary_}

C'est le cas particulier $S=\spec(k)$, $Y=S$. 

\begin{corollary_}\label{VII:1-11}
Soient $X$ et $Y$ deux sch\'emas de type fini sur $k$ s\'eparablement clos et 
$K\in \ob\eD^-(X,A^\circ)$, $L\in \ob\eD^-(Y,A)$ des complexes de faisceaux de 
modules \`a droite et \`a gauche. La fl\`eche de K\"unneth 
\[
  \eR \Gamma(X,K) \lotimes \eR \Gamma(Y,L) \to \eR\Gamma(X\times Y,\operatorname{pr}_1^\ast K \lotimes \operatorname{pr}_2^\ast L) 
\]
est un isomorphisme. 
\end{corollary_}

On proc\`ede comme en \cite[XVII 5.4.3]{sga4}: 
\[\xymatrix{
  X\times Y \ar[r]^-{\text{pr}_1} \ar[d]^-{\text{pr}_2} 
    & X \ar[d]^-a \\
  Y \ar[r]^-b 
    & \spec(k) 
}\]
changeant de base par $b$, on trouve que 
$\eR\operatorname{pr}_{2,\ast}{} \operatorname{pr}_1^\ast K$ est 
$b^\ast \eR\Gamma(X,K)$. On a alors (cf. \cite[XVII 5.2.11]{sga4} et preuve) 

\begin{align*}
  \eR\Gamma(X\times Y,\operatorname{pr}_1^\ast K\lotimes \operatorname{pr}_2^\ast L) 
    &= \eR\Gamma(Y,\eR \operatorname{pr}_{2,\ast}(\operatorname{pr}_1^\ast K\lotimes \operatorname{pr}_2^\ast L)) \\
    &= \eR\Gamma(Y,(\eR\operatorname{pr}_{2,\ast}\operatorname{pr}_1^\ast K)\lotimes L) \\
    &= \eR\Gamma(Y,b^\ast \eR\Gamma(X,K)\lotimes L) \\
    &= \eR\Gamma(X,K)\lotimes \eR\Gamma(Y,L) 
\end{align*}





\subsection{}\label{VII:1-12}

Enfin, comme contre-partie locale de \ref{VII:1-9}(ii), nous prouverons un 
th\'eor\`eme d'acyclicit\'e locale g\'en\'erique (\ref{VII:2-13}). 










\section{Th\'eor\`emes g\'en\'eriques}\label{VII:2}

Dans ce paragraphe, nous prouvons \ref{VII:1-9} et le th\'eor\`eme 
d'acyclicit\'e locale g\'en\'erique \ref{VII:2-13}. Pour prouver \ref{VII:1-9}, 
on se ram\`ene aussit\^ot \`a supposer $S$ int\`egre. Soit $\eta$ son point 
g\'en\'erique. 





\subsection{}\label{VII:2-1}

Nous commencerons par prouver \ref{VII:1-9} sous les hypoth\`eses 
additionnelles suivantes: $X$ est lisse sur $S$, purement de dimension relative 
$n$, $A=\dZ/m$, $\sF$ est localement constant et $Y=S$. Soit 
$\sF'=\Hom(\sF,\dZ/m)$. Nous allons montrer que si les $\eR^i f_! \sF'$ sont 
localement constants, les conclusions de \ref{VII:1-9} valent pour $U=S$. 
Rappelons que si $\sL$ est un faisceau localement constant constructible, les 
$\Ext^i(\sL,\sG)$ se calculent fibre par fibre, et sont constructibles si $\sG$ 
l'est. Rappelons aussi que $\dZ/m$ est un $\dZ/m$-module injectif, et est le 
module dualisant, de sorte que $\sF=\rHom(\sF',\dZ/m)$. La dualit\'e de 
Poincar\'e 
\[
  \eR f_\ast \rHom(K,\eR f^! L) = \rHom(\eR f_! K,L) \text{,} 
\]
pour $K=\sF'$, $L=\dZ/m$, $\eR f^! L = \dZ/m[2n](n)$, fournit donc 
\[
  \eR^{2n-i} f_\ast \sF = \Hom(\eR^i f_! \sF',\dZ/m)(-n) \text{.} 
\]

Le faisceau au second membre est localement constant, de formation compatible 
\`a tout changement de base -- d'o\`u l'assertion. 





\subsection{}\label{VII:2-2}

Prouvons \ref{VII:1-9} sous les hypoth\`eses additionnelles: $X$ est lisse sur 
$S$, $\sF$ est localement constant et $Y=S$. 
\begin{enumerate}[a)]
  \item D\'ecomposant $X$ en composantes connexes, on se ram\`ene \`a supposer 
    qu'il est purement d'une dimension relative $n$ sur $S$. 
  \item D\'ecomposant $A$ en produit, on se ram\`ene \`a supposer que 
    $\ell^m A=0$, avec $\ell$ premier inversible sur $S$; on filtre alors $\sF$ 
    par les $\ell^k \sF$ pour se ramener, par la suite spectrale 
    correspondante, au cas o\`u $\ell\sF=0$. On peut alors remplacer $A$ par 
    $A/\ell A$ et supposer que $\ell A=0$. 
  \item Remplaçant $S$ par $U$ convenable, on peut supposer qu'il existe un 
    rev\^etement galoisien fini \'etale $X_1/X$, de groupe de Galois $G$, telle 
    que l'image r\'eciproque $\sF_1$ de $\sF$ sur $X_1$ soit un faisceau 
    constant, de valeur constante $F$. Notant $f_1$ la projection de $X_1$ sur 
    $S$, on a alors 
    \[
      \eR^i {f_1}_\ast \sF_1 = (\eR^i {f_1}_\ast \dZ/\ell)\otimes_{\dZ/\ell} F \text{.} 
    \]
    On dispose aussi de la suite spectrale de Hochschild-Serre (ou: de Leray 
    pour le recouvrement $X_1/X$) 
    \[
      \underline\h^p(G,\eR^q {f_1}_\ast \sF_1) \Rightarrow \eR^{p+q} f_\ast \sF \text{.} 
    \]
    D'apr\`es \ref{VII:2-1}, on peut supposer, quitte \`a r\'etr\'ecir $U$, que 
    les $\eR^i {f_1}_\ast \dZ/\ell$ sont localement constants, de formation 
    compatible \`a tout changement de base. La m\^eme propri\'et\'e vaut alors 
    pour les $\eR^i f_\ast \sF$. Enfin, si $\bar\eta$ est un point 
    g\'eom\'etrique localis\'e au point g\'en\'erique $\eta$ de $S$, les 
    $\eR^i f_\ast\sF$ sont presque tous nuls, car les 
    $(\eR^i f_\ast \sF)_{\bar\eta} = \h^i(X_{\bar \eta},\sF)$ le sont. 
\end{enumerate}





\subsection{}\label{VII:2-3}

Prouvons par r\'ecurrence sur $n$ que 
\begin{equation*}\tag{$\ast_n$}\label{VII:eq:2-3-1}
  \begin{array}{c}
    \text{Les conclusions de \ref{VII:1-9} sont vraies lorsque $\dim{X_\eta}\leqslant n$} \\
    \text{et que $f$ est un plongement ouvert d'image dense.}
  \end{array}
\end{equation*}

Pour $n=0$, quitte \`a r\'etr\'ecir $S$, on a $X=Y$: $(\ast_0)$ est \'evident. 
Supposons ($\ast_{n-1})$, et prouvons \eqref{VII:eq:2-3-1}. Dans 
($\ast_{n-1}$), on peut remplacer ``plongement ouvert'' par ``plongement'' 
comme on le voit en factorisant en plongement ouvert et plongement ferm\'e. 





\begin{lemma_}\label{VII:2-4}
Quitte \`a r\'etr\'ecir $S$, les conclusions de \ref{VII:1-9} valent au-
dessus de $Y'\subset Y$, le compl\'ement $Y_1$ de $Y'$ \'etant fini sur $S$. 
\end{lemma_}

L'assertion est locale sur $Y$, qu'on peut supposer affine: $Y\subset \dA_S^n$. 
L'hypoth\`ese de r\'ecurrence ($\ast_{n-1}$) s'applique \`a 
\[\xymatrix{
  X \ar[r]^-f \ar[dr] 
    & Y \ar[d]^-{\pr_i} \\
  & \dA_S^1 
}\]
Il existe donc pour chaque $i$ ouvert dense $U_i$ de $\dA_S^1$ tel que les 
conclusions de \ref{VII:1-9} valent au-dessus de $\pr_i^{-1}(U_i)$; elle valent 
au-dessus de la r\'eunion des $\pr_i^{-1}(U_i)$, et \ref{VII:2-4} en r\'esulte. 





\addtocounter{subsection}{1} % a section is skipped in the original
\subsection{}\label{VII:2-6}

Prouvons \eqref{VII:eq:2-3-1} pour $X$ lisse sur $S$ et $\sF$ localement 
constant sur $X$. Le probl\`eme \'etant local sur $Y$, on peut supposer $Y$ 
affine, puis projectif (remplacer $Y$ par son adh\'erence dans un espace 
projectif). Soient $i:Y_1\hookrightarrow Y$ et $j:Y' \to Y$ garantis par 
\ref{VII:2-4}. 
\[\xymatrix{
  X \ar@{^{(}->}[r] \ar[dr]_-a 
    & Y \ar[d]^-b 
    & \ar[l]_-i \ar[dl]^-{b_1} Y_1 \\
  & S 
}\] 
quitte \`a r\'etr\'ecir $S$, on sait que $j^\ast \eR f_\ast \sF$ est 
constructible, de formation compatible \`a tout changement de base en $S$; on 
sait aussi que $\eR a_\ast \sF = \eR b_\ast \eR f_\ast \sF$ est constructible, 
de formation compatible \`a tout changement de base. 

Appliquons $\eR b_\ast$ au triangle d\'efini par la suite exacate 
\begin{equation*}\tag{1}\label{VII:eq:2-6-1}
\xymatrix{
  0 \ar[r] 
    & j_! j^\ast \eR f_\ast \sF \ar[r] 
    & \eR f_\ast \sF \ar[r] 
    & i_! i^\ast \eR f_\ast \sF \ar[r] 
    & 0 \text{:} 
}
\end{equation*}
on obtient un triangle 
\begin{equation*}\tag{2}\label{VII:eq:2-6-2}
\xymatrix{
  \ar[r] 
    & \eR b_\ast j_! j^\ast \eR f_\ast \sF \ar[r] 
    & \eR a_\ast \sF \ar[r] 
    & {b_1}_\ast i^\ast \eR f_\ast \sF \ar[r] 
    & 
}
\end{equation*}
dans lequel les deux premiers termes sont constructibles, de formation 
compatible \`a tout changement de base en $S$ (pour le ler, d'apr\`es le 
th\'eor\`eme de finitude pour le morphisme propre $b$). Il en va donc de 
m\^eme pour le troisi\`eme. Puisque $b_1$ est fini, on en d\'eduit que 
$i^\ast \eR f_\ast \sF$ est constructible de formation compatible \`a tout 
changement de base en $S$, et de m\^eme pour $\eR f_\ast \sF$ par 
\eqref{VII:eq:2-6-1}. 





\subsection{}\label{VII:2-7}

Prouvons \eqref{VII:eq:2-3-1} en g\'en\'eral. On commence par se ramener au cas 
o\`u dans $X$ existe un ouvert dense $V$ lisse sur $S$. Pour $S$ spectre d'un 
corps parfait, il suffit de remplacer $X$ par $X_\text{red}$ (et $Y$ par 
$Y_\text{red}$). En g\'en\'eral, il faut rapetisser $S$, faire un changement de 
base fini radicel et surjectif $S' \to S$, et remplacer $X$ et $Y$ par 
$X_\text{red}$ et $Y_\text{red}$. La topologie \'etale \'etant insensible aux 
morphismes finis radiciels et surjectifs, ceci est innocent. Quitte \`a 
r\'etr\'ecir $V$, on peut supposer $\sF$ localement constant sur $V$ 
\[\xymatrix{
  V \ar@{^{(}->}[r]^i 
    & X \ar@{^{(}->}[r]^-f 
    & Y \text{.} 
}\]
D\'efinissons $\Delta$ par le triangle 
\begin{equation*}\tag{1}\label{VII:eq:2-7-1}
\xymatrix{
  \ar[r] 
    & \sF \ar[r] 
    & \eR j_\ast j^\ast \sF \ar[r] 
    & \Delta \ar[r] 
    & \text{.} 
}
\end{equation*}
Les faisceaux de cohomologie e $\Delta$ sont \`a support dans $X\setminus V$, 
et $\dim(X\setminus V)_\eta < n$. L'hypoth\`ese de r\'ecrrence permet donc de 
supposer que $\eR f_\ast \Delta$ est constructible. Appliquons $\eR f_\ast$ au 
triangle \eqref{VII:eq:2-7-1}; on trouve un triangle 
\[\xymatrix{
  \ar[r] 
    & \eR f_\ast \sF \ar[r] 
    & \eR(f j)_\ast j^\ast \sF \ar[r] 
    & \eR f_\ast \Delta \ar[r] 
    & 
}\]
dans lequel deux des sommets sont constructibles de formation compatible \`a 
tout changement de base en $S$. Le troisi\`eme l'est donc \'egalement. 





\subsection{}\label{VII:2-8}

Prouvons \ref{VII:1-9}. Le probl\`eme est local sur $Y$, qu'on peut supposer 
affine. Prenant un recouvrement affine de $X$ et invoquant la suite spectrale 
de Leray, on se ram\`ene \`a avoir $X$ \'egalement affine. Tout ceci pour 
assurer qu'on puisse factoriser $f$ en un plongement ouvert suivi d'un 
morphisme propre: $f=g j$, d'o\`u $\eR f_\ast = \eR g_\ast \eR j_\ast$. Le 
plongement ouvert est justiciable d'un \eqref{VII:eq:2-3-1}, le morphisme 
propre du th\'eor\`eme de finitude. 

Les corollaires suivants se prouvent comme au paragraphe 1. 





\begin{corollary_}\label{VII:2-9}
Sous les hypoth\`eses du th\'eor\`eme, pour $K$ dans $\eD_c^b(X,A)$ ou 
$\eD_c^b(Y,A)$ respectivement, il existe un ouvert non vide $U$ de $S$ 
au-dessus duquel $\eR f_\ast K$, $\eR f_! K$, $f^\ast K$, $\eR f^! K$ soient 
dans $\eD_c^b(Y,A)$ ou $\eD_c^b(X,A)$, et de formation compatible \`a tout 
changement de base $S' \to U\subset S$. 
\end{corollary_}





\begin{corollary_}\label{VII:2-10}
Soit $X$ comme dans le th\'eor\`eme, et supposons $A$ commutatif. Alors, si 
$\sF$ et $\sG$ sont des faisceaux constructibles de $A$-modules sur $X$, 
au-dessus d'un ouvert dense $U$ de $S$, les $\Ext_A^i(\sF,\sG)$ sont encore 
constructibles, de formation compatible \`a tout changement de base 
$S' \to U\subset S$. 
\end{corollary_}





\subsection{}\label{VII:2-11}

Si $x$ est un point g\'eo\'emetrique d'un sch\'ema $X$, nous noterons $X_x$ 
l'henselis\'e strict de $X$ en $x$. Pour $f:X\to S$ et $t$ un point 
g\'eom\'etrique de $S$, nous noterons $X_t$ la fibre g\'eom\'etrique de $X$ en 
$t$. Enfin, pour $x$ un point g\'eom\'etrique de $X$, et $t$ un point 
g\'eom\'etrique de $S_{f(x)}$, $(X_x)_t$ est la fibre en $t$ de 
$X_x \to S_{f(x)}$. 





\begin{definition_}\label{VII:2-12}
Soient $f:X\to S$ et $K\in \ob\eD^+(X,A)$. On dit que $f$ est \emph{localement 
acyclique en $x$, res. $K$}, si pour tout point g\'eom\'etrique $t$ de 
$S_{f(x)}$ on a $K_x=\eR\Gamma(X_x,K)\iso \eR\Gamma(X_{x,t},K)$. On dit que $f$ 
est \emph{localement acyclique, rel. $K$}, si c'est vrai pour tout $x$, et 
\emph{universellement localement acyclique, rel. $K$}, si cela reste vraie 
apr\`es tout changement de base $S' \to S$.
\end{definition_}





\begin{theorem_}\label{VII:2-13}
Soit $f:X\to S$ un morphisme de type fini et $\sF$ constructible sur $X$. Il 
existe un ouvert dense $U$ de $S$ au-dessus duquel $f$ est universellement 
localement acyclique, rel. $\sF$. 
\end{theorem_}

Nous admettrons le r\'esultat suivant, prouv\'e dans l'appendice. 





\begin{lemma_}\label{VII:2-14}
Soit $X\xrightarrow f S_1 \xrightarrow g S_2$. Si $g$ est lisse, et $f$ 
universellement localement acyclique rel. $K$, alors $g f$ l'est aussi. 
\end{lemma_}

On se ram\`ene \`a supposer $S$ int\`egre, de point g\'en\'erique $\eta$, et 
on proc\`ede par r\'ecurrence sur $\dim{X_\eta}$. On commence par d\'eduire de 
hypoth\`ese de r\'ecurrence que 
\begin{equation*}\tag{A}\label{VII:eq:2-14-1}
  \begin{array}{c}
    \text{Quitte \`a r\'etr\'ecir $S$, il existe $T\subset X$, fini sur $S$, tel que $f$} \\
    \text{soit universellement localement acyclique en dehors de $T$.} 
  \end{array}
\end{equation*}

Cette question \'etant locale, on se localise sur $X$ et on factorise $f$ en 
$X\xrightarrow u \dA_S^1 \xrightarrow S$. L'hypoth\`ese de r\'ecurrence 
s'applique \`a $u$ et on conclut par \ref{VII:2-14} et les arguments habituels. 

Pour prouver le th\'eor\`eme, on peut supposer $X$ propre, puisque le 
probl\`eme est local. On r\'etr\'ecit $S$ pour que les $\eR^i f_\ast \sF$ 
soient localement constants et que \eqref{VII:eq:2-14-1} soit applicable, et on 
utilise le lemme suivant. 





\begin{lemma_}\label{VII:2-15}
Soient $f:X\to S$, propre, et $\sF$ constructible. On suppose que les 
$\eR^i f_\ast \sF$ sont localement constants et que, en dehors de $T\subset X$ 
fini sur $S$, $f$ localement acyclique rel. \`a $\sF$. Alors, $f$ est 
localement acyclique rel. \`a $\sF$. 
\end{lemma_}

Soient $s$ un point g\'eom\'etrique de $S$, $S_s$ le localis\'e strict de $S$ 
en $s$ et $t$ un point g\'eom\'etrique de $S_s$. Soient $X_{(s)}$ l'image 
r\'eciproque de $X$ sur $S_s$, $\bar j:X_t \to X_{(s)}$ et 
$i:X_s \hookrightarrow X_{(s)}$. Notant encore $\sF$ l'image r\'eciproque de 
$\sF$ sur $X_{(s)}$ ou $X_t$, il faut prouver que 
$i^\ast \sF \iso i^\ast \eR \bar j_\ast \sF$. Soit $\Delta$ le mapping cylinder 
de cette application. Ses faisceaux de cohomologie sont \`a support dans $T_s$. 
Pour prouver que $\Delta=0$, il suffit donc de prouver que 
$\eR\Gamma(X_s,\Delta)=0$. C'est l\`a le mapping cylinder de 
$(\eR f_\ast \sF)_s = \eR\Gamma(X_s,\sF) \to \eR\Gamma(X_s,i^\ast \eR\bar j_\ast \sF) = \eR\Gamma(X_{(s)},\eR\bar j_\ast \sF) = \eR\Gamma(X_t,\sF) = (\eR f_\ast \sF)_t$. 
Ce morphisme de sp\'ecialisation est par hypoth\`ese un isomorphisme, et 
$\Delta=0$. 





\begin{corollary_}\label{VII:2-16}
Pour $S$ le spectre d'un corps, tout $S$-sch\'ema $X$ est universellement 
localement acyclique, rel. $K$, quel que soit $K\in\ob\eD^+(X,A)$. 
\end{corollary_}

Se d\'eduit de \ref{VII:2-13} par passage \`a la limite (possible, car $U$ de 
\ref{VII:2-13} est toujours \'egal \`a $S$. 










\section{Preuve de \texorpdfstring{\ref{VII:1-1}}{1.1} et constructibilit\'e des faisceaux de cycles \'evanescents}\label{VII:3}





\subsection{}\label{VII:3-1}

Soit $S$ un trait strictement local (le spectre d'un anneau de valuation 
discr\`ete henselien \`a corps r\'esiduel s\'eparablement clos). On note $s$ et 
$\eta$ ses points ferm\'e et g\'en\'erique, et $\bar\eta$ un point 
g\'eom\'etrique localis\'e et $\eta$ (une cl\^oture s\'eparable de $k(\eta)$). 
Soient $X$ sur $S$ et $\sF$ un faisceau sur $X_\eta$. Rappelons la 
d\'efinition des faisceaux de cycles \'evanescents $\eR^i \Psi_\eta(\sF)$ (des 
faisceaux sur $X_s$, muni d'une action de $\gal(\bar\eta/\eta)$). Soient 
$i:X_s\hookrightarrow X$, $j:X_\eta \hookrightarrow X$, et $\bar j$ le 
compos\'e $X_{\bar\eta}\to X_\eta \hookrightarrow X$; on a 
\[
  \eR^i \Psi_\eta(\sF) = i^\ast \eR^i \bar j_\ast (\bar j^\ast \sF) \text{.} 
\]

Voici le th\'eor\`eme principal de ce num\'ero. Il am\'eliore 
\cite[XIII 2.3.1, 2.4.2]{sga7}. 





\begin{theorem_}\label{VII:3-2}
On suppose que $X$ de type fini sur $S$ et $\sF$ constructible. Alors, les 
faisceaux de cycles \'evanescents $\eR^i \Psi_\eta(\sF)$ sont constructibles. 
\end{theorem_}

On proc\`ede par r\'ecurrence sur la dimension $n$ de $X_\eta$. On peut par 
ailleurs supposer -- et on suppose -- $X_\eta$ dense dans $X$ (remplacer $X$ 
par $\bar X_\eta$ ne change pas les $\eR^i \Psi$, qui sont \`a support dans 
$\bar X_\eta$). 





\begin{lemma_}\label{VII:3-3}
Si \ref{VII:3-2} est vrai en dimension de $X_\eta<n$, et que $\dim X_\eta=n$, 
il existe des sous-faisceaux constructibles $\sG_i$ des $\eR^i \Psi_\eta(\sF)$ 
tels que les supports des sections locales de $\eR^i\Psi_\eta(\sF)/\sG$ soient 
finis. 
\end{lemma_}

Soit $s'$ un point g\'en\'erique g\'eom\'etrique de la droite affine sur $s$, 
et soit $S'$ le localis\'e strict en $s'$ de la droite affine $\dA_S^1$ sur 
$S$. C'est encore un trait strictement local, et les uniformisants pour $S$ 
sont des uniformisantes sur $S'$. 
\[\xymatrix{
  & \spec(k') \ar[rr] \ar[dd] \ar[dr] 
    & & (S',\eta',s') \ar[dr] \ar[dd] \\
  \bar\eta' \ar[ur] \ar[dr] 
    & & \dA_{\bar S}^1 \ar[d]  
    & & \dA_S^1 \ar[dl] \\
  & \bar\eta\ar@{^{(}->}[r] 
    & \bar S \ar[r] 
    & (S,\eta,s) 
}\]
Soit $k'=k(\bar\eta)\otimes_{k(\eta)}k(\eta')$. Si $\bar S$ est le normalis\'e 
de $S$ dans $\bar\eta$, $k'$ est le corps des fractions du localis\'e strict 
de $\dA_{\bar S}^1$ en $s'$; c'est donc un corps, et 
$\gal(\bar\eta/\eta)=\gal(k'/\eta')$. Soit $\bar\eta'$ le spectre d'une 
cl\^oture alg\'ebrique de $k'$. Le groupe de Galois $P=\gal(\bar\eta'/k')$ est 
un pro-$p$-groupe, pour $p$ l'exposant caract\'eristique de $k(s)$. 





\begin{lemma_}\label{VII:3-4}
Pour tout sch\'ema $X'$ sur $S'$, et tout faisceau $\sF$ sur 
$X_\eta'=X_{\eta'}'$, on a entre les faisceaux de cycles \'evanescents pour 
$X'/S'$ et pour $X'/S$ la relation 
$\eR^i \Psi_\eta(\sF) =\eR^i\Psi_{\eta'}(\sF)^P$. 
\end{lemma_}

Le diagramme suivant compare les morphismes utilis\'es dans la d\'efinition de 
$\eR\Psi$ pour $X'/S'$ et $X'/S$: 
\[\xymatrix{
  X_{\bar\eta'} \ar[rr] \ar[d] 
    & & X_{\eta'}' \ar@{=}[dd] \ar@{^{(}->}[dr] 
    & & S' \\ 
  X_{\eta'}\times_{\eta'}\spec(k') \ar@{=}[d] \ar[urr] 
    & & & X \ar[ur] \ar[dr] \\
  X_\eta' \ar[rr] 
    & & X_\eta' \ar@{^{(}->}[ur] 
    & & S
}\]
et le lemme r\'esulte de la suite spectrale de Hochschild-Serre, compte tenu de 
ce que $P$ est un pro-$p$-groupe, avec $p$ inversible dans $A$. 

Prouvons \ref{VII:3-3}. La question est locale; ceci permet de supposer $X$ 
affine, $X\subset \dA_S^n$. Soient $f$ l'une des projection 
$X\subset \dA_S^n\to \dA_S^1$. $X'$ le ``localis\'e'' $X_{\dA_S^1} S'$ de $x$, 
et $\sF'$ l'image inverse de $\sF$ sur $x'$ 
\[\xymatrix{
  \sF'\text{ sur }X' \ar[r] \ar[d]^-\lambda 
    & S' \ar[d]^-\lambda \\
  \sF \text{ sur }X \ar[r]^-f 
    & \dA_S^1 \ar[r] 
    & S \text{.} 
}\]
On a, sur $X_s'=X_{s'}'$, 
\[
  \lambda^\ast \eR^i\Psi_\eta(\sF) = \eR^i \Psi_\eta(\sF') = \eR^i\Psi_{\eta'}(\sF')^P \text{:} 
\]
ce faisceau est constructible, car l'hypoth\`ese de r\'ecurrence s'applique \`a 
$X'/S'$, et on utilise le lemme suivant appliqu\'e \`a $X_s\subset \dA_S^n$ et 
aux faisceaux de cycles \'evanescents. 




