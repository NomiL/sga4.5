% !TEX root = sga4.5.tex

\chapter{Dualité}\label{V}




















On trouvera dans cet expos\'e quelques th\'eor\`emes et compatibilit\'es, tous 
relatifs \`a la dualit\'e de Poincar\'e. Au paragraphe 1, le th\'eor\`eme de 
bidualit\'e locale en dimension $1$ \cite[I.5.1]{sga5} et quelques calculus de 
deaux. Au paragraphe 2, une d\'emonstration tr\`es \'economique de la dualit\'e 
de Poincar\'e sur les courbes, que m'a apprise M.\ Artin. Au paragraphe 3, une 
compatibilit\'e qui fait le lien entre deux d\'efinitions de l'accouplement qui 
donne lieu \`a la dualit\'e de Poincar\'e pour les courbes: par cup-produit, ou 
par autodualit\'e de la jacobienne. Au paragraphe 4, enfin, la preuve de la 
compatibilit\'e du titre. 

Dans tout l'expos\'e, les sch\'emas seront noeth\'eriens et s\'epar\'es, et $n$ 
est un entier inversible sur tous les sch\'emas consid\'er\'es. 










\section{Bidualit\'e locale, en dimension \texorpdfstring{$1$}{1}}\label{V:1}





\subsection{}\label{V:1-1}

Soit $S$ un sch\'ema r\'egulier purement de dimension $1$. Nous nous proposons 
de montrer que le complexe r\'eduit \`a $\dZ/n$ en degr\'e $0$ est 
dualisant, i.e. que pour $\sK\in \ob\D_c^b(S,\dZ/n)$ ($(-)_c$ pour 
constructible), si on pose $D\sK=\rHom(\sK,\dZ/n)$, alors $D\sK$ est encore 
constructible \`a cohomologie born\'ee, et que le morphisme canonique $\alpha$ 
de $\sK$ dans $D D\sK$ est un isomorphisme. 

Pour $\sK$ dans $\D^-$, et $\sL$ quelconque, on a 
\[
  \hom(\sK\lotimes \sL,\dZ/n) = \hom(\sL,\Hom(\sK,\dZ/n)) \text{.}
\]
Le morphisme de $\sK$ dans $DD\sK$ est d\'efini en supposant $D\sK$ dans 
$\D^-$; si $\beta:D\sK\lotimes\sK\to \dZ/n$ est l'accouplement canonique, il 
est d\'efini par l'accouplement 
$\sK\lotimes D\sK=D\sK\lotimes \sK \xrightarrow\beta \dZ/n$. 





\subsection{}\label{V:1-2}

Soit $f:X\to S$ un morphisme s\'epar\'e de type fini. Posons 
$\sK_X=\R f^!\dZ/n$. Pour $\sK\in \ob\D^-(X,\dZ/n)$, on pose 
$D\sK = \rHom(\sK,\sK_X)$. L'adjonction entre $\R f_!$ et $\R f^!$ 
assure que 
\[\xymatrix{
  \R f_\ast \sK\lotimes \R f_\ast D\sK \ar[r] 
    & \R f_! (\sK\lotimes D\sK) \ar[r] 
    & \R f_! \R f^! \dZ/n \ar[r] 
    & \dZ/n \text{.}
}\]
La sym\'etrie de cette description montre que, pour $f$ propre, le diagramme 
\begin{equation}\label{V:eq:1-2-1}
\xymatrix{
  \R f_\ast \sK \ar[r] \ar[d] 
    & D D \R f_\ast \sK \ar[d]^-\sim \\
  \R f_\ast D D \sK \ar[r]^\sim 
    & D\R f_\ast D\sK
}
\end{equation}
est commutatif. 

Si $f$ est l'inclusion d'un point ferm\'e, on sait que 
$\R f^!\dZ/n=\dZ/n(-1)[-2]$ -- soit, \`a torsion et d\'ecalage pr\`es, $\dZ/n$. 
Sur le spectre d'un coprs, ce complexe est dualisant (dualit\'e de Pontrjagin 
pour les $\dZ/n$-modules). D'apr\`es \eqref{V:eq:1-2-1}, on a donc 
$\sK\iso D D\sK$ pour $\sK$ de la forme $\R f_!\sL$ -- et donc lorsque le 
support de $\underline\h^\bullet(\sK)$ est fini. 





\begin{theorem_}\label{V:1-3}
Soient $j:U\hookrightarrow S$ un ouvert dense de $S$ et $\sF$ un faisceau 
localement constant constructible de $\dZ/n$-modules sur $U$. On a 
$\D j_\ast \sF = j_\ast D \sF$, i.e. $\Hom(j_\ast \sF,\dZ/n) = j_\ast\Hom(\sF,\dZ/n)$ et $\Ext^i(j_\ast\sF,\dZ/n) = 0$ pour $i>0$.
\end{theorem_}

Sur $U$, $\Ext^i(j_\ast\sF,\dZ/n)=0$ pour $i>0$, car $\sF$ est localement 
constant (et $\dZ/n$ est un $\dZ/n$-module injectif), tandis que pour $i=0$ 
c'est le dual $\sF^\vee$ de $\sF$. On v\'erifie que 
$\hom(j_\ast\sF,\dZ/n)=j_\ast\sF^\vee$, et il reste \`a v\'erifier la nullit\'e 
des $\Ext^i$ ($i>0$) en les points de $S\setminus U$. Le probl\`eme est local 
en ces points. Ceci nous ram\`ene \`a supposer que $S$ est un trait strictement 
local et, que $U$ est r\'eduit \`a son point g\'en\'erique $\eta$. Soit 
$I=\gal(\bar\eta/\eta)$. Le faisceau $\sF$ s'identifie au module galoisien 
$\sF_{\bar\eta}$, et la fibre sp\'eciale de $j_\ast \sF$ \`a 
$\sF_{\bar\eta}^I$. 

Soit $i$ l'inclusion du point ferm\'e $s$, et appliquons $D$ aux suites 
exactes 
\[\xymatrix{
  0 \ar[r] 
    & j_! \sF \ar[r] 
    & j_\ast \sF \ar[r] 
    & i_\ast \sF_{\bar\eta}^I \ar[r] 
    & 0 \\
  0 \ar[r] 
    & j_! \sF_{\bar\eta}^I \ar[r] \ar[u] 
    & \sF_{\bar\eta}^I \ar[r] \ar[u] 
    & i_\ast \sF_{\bar\eta}^I \ar[r] \ar[u]
    & 0 \text{.}
}\]

On obtient un morphisme de triangles 
\[\xymatrix{
  i_\ast\left(\sF_{\bar\eta}^I\right)^\vee (-1)[-2] \ar[r] \ar@{=}[d] 
    & D j_\ast\sF \ar[r] \ar[d] 
    & \R j_\ast \sF^\vee \ar[d] \\
  i_\ast\left(\sF_{\bar\eta}^I\right)^\vee(-1)[-2] \ar[r] 
    & \left(\sF_{\bar\eta}^I\right)^\vee \ar[r] 
    & \R j_\ast \sF_{\bar\eta}^I \text{.}
}\]
La suite exacte longue d\'eduite de la premi\`er ligne fournit la nullit\'e des 
$\Ext^i$ ($i>2$) et, prenant la fibre en $s$, on trouve 
\[\xymatrix{
  0 \ar[r] 
    & \left(\underline\h^1(D j_\ast\sF\right)_s \ar[r] 
    & \h^1\left(I,\sF_{\bar\eta}^\vee\right) \ar[r]^-\partial \ar[d] 
    & \left(\sF_{\bar\eta}^I\right)^\vee(-1) \ar[r] \ar@{=}[d] 
    & \left(\underline\h^2 D j_\ast\sF\right)_s \ar[r] 
    & 0 \\
  & 0 \ar[r] 
    & \h^1\left(I,(\sF_{\bar\eta}^I)^\vee\right) \ar[r]^-\partial 
    & \left(\sF_{\bar\eta}^I\right)^\vee(-1) \ar[r] 
    & 0 \text{.}
}\]
Puisque 
$\left(\sF_{\bar\eta}^I\right)^\vee = \left(\sF_{\bar\eta}^\vee\right)_I$, 
posant $M=\sF_{\bar\eta}^\vee$, il faut finalement v\'erifier que 
\[\xymatrix{
  \h^1(I,M) \ar[r]^-\sim 
    & \h^1(I,M_I) \text{.}
}\]
Si $p$ est l'exposant caract\'eristique r\'esiduel, $I$ est extension d'un 
groupe isomorphe \`a $\widehat\dZ_{p'} = \varprojlim_{(m,p)=1} \dZ/m$ par un 
$p$-groupe $P$. Puisque $P$ est premier \`a l'ordre de $M$, on a 
$\h^1(P,M)=0$ ($i>0$) et $(M_I)^P=(M^P)_I$. Ceci permet de remplacer $M$ par 
$M^P$ et $I$ par $I/P$. On a enfin un isomorphisme functoriel 
$\h^1(\widehat\dZ_{p'},M)\sim ($coinvariants de $\widehat\dZ_{p'}$, dans $M)$, 
d'o\`u le th\'eor\`eme. 





\begin{theorem_}\label{V:1-4}
Pour $\sK\in \ob\D_c^b(S,\dZ/n)$, on a $\sK\iso D D \sK$.
\end{theorem_}

Par d\'evissage, on se ram\`ene \`a supposer que $\sK$ est r\'eduit \`a un 
faisceau constructible $\sF$ en degr\'e $0$, et que $\sF$ est soit \`a support 
fini (\ref{V:1-2}), soit de la forme $j_\ast\sF_1$ comme en \ref{V:1-3}. Dans 
ce second cas, \ref{V:1-3} nous ram\`ene \`a la bidualit\'e locale pour $\sF$ 
localement constant sur $U$. 










\section{La dualité de Poincaré pour les courbes}\label{V:2}





\subsection{}\label{V:2-1}

Soit $X$ une courbe projective et lisse sur $k$ alg\'ebriquement clos. On pose 
$\sK_X=\dZ/n(1)[2]$ et, pour $\sK\in\ob \D_c^b(X,\dZ/n)$, 
$D\sK=\rHom(\sK,\sK_X)$. Pour $M$ un $\dZ/n$-module, on pose aussi 
$D M = \hom(M,\dZ/n)$; de m\^eme pour les complexes de modules. Le morphisme 
trace $\h^0(X,\sK_X) \to \dZ/n$, ou $\R\Gamma(X,\sK_X)\to \dZ/n$, d\'efinit un 
accouplement $\R\Gamma(X,\sK)\lotimes\R\Gamma(X,D\sK) \to \dZ/n$. La dualit\'e 
de Poincar\'e entre cohomologie et cohomologie \`a supports propres d'un ouvert 
$j:U\hookrightarrow X$ de $X$ dit que, pour $\sK=j_!\dZ/n$, cet accouplement 
identifie chaque facteur au dual de l'autre. 

J'expose ci-dessous une d\'emonstration, qui m'a \'et\'e communiqu\'ee par 
M.\ Artin, de ce que pour $\sK\in\ob\D_c^b(X,\dZ/n)$, cet accouplement est 
toujours parfait, i.e. d\'efinit un isomorphisme
\begin{equation}\label{V:eq:2-1-1}
\xymatrix{
  \R\Gamma(X,\sK) \ar[r]^-\sim 
    & D \R\Gamma(X,D\sK) \text{.}
}
\end{equation}

Pour tout faisceau constructible $\sF$, posons 
\[
  '\h^i(X,\sF) = \text{$\dZ/n$-dual de $\h^{-i}(X,D\sF)$.}
\]
Que \eqref{V:eq:2-1-1} soit un isomorphisme \'equivaut au 





\begin{theorem_}\label{V:2-2}
Pour $\sF$ un faisceau constructible de $\dZ/n$-modules, on a 
\begin{equation}\label{V:eq:2-2-1}
\xymatrix{
  \h^i(X,\sF) \ar[r]^-\sim 
    & '\h^i(X,\sF) \text{.}
}
\end{equation}
\end{theorem_}





\begin{lemma_}\label{V:2-3}
Soit $f:X\to Y$ un morphisme g\'en\'eriquement \'etale entre courbes lisses sur 
$k$. On a $\sK_X=f^\ast\sK_Y=\R f^!\sK_Y$, avec $\tr_f$ pour fl\`eche 
d'adjonction $\R f_!\R f^!\sK_Y\to \sK_Y$.
\end{lemma_}

% NOTE: \rhom or \rHom ? originally \rhom
On se ram\`ene \`a v\'erifier que pour $f$ fini, on a 
$f_\ast \R f^!\sK_Y = \rHom(f_\ast\dZ/n,\sK_Y)$ est $f_\ast\dZ/n$, avec $\tr_f$ 
pour fl\`eche d'adjonction. La nullit\'e des $\Ext^i$ ($i>0$) r\'esulte de 
\ref{V:1-3}, et l'assertion en r\'esulte. 

On peut dire, plus explicitement, que $\R f^!$ est le foncteur d\'eriv\'e du 
foncteur $f^!:f^!\sF(U)=\hom(f_!\dZ_U,\sF)$ et \ref{V:1-3} implique la 
nullit\'e des $\R^i f^!\dZ/n$ pour $i>0$.





\begin{corollary_}\label{V:2-4}
Soit $f:X'\to X$ un morphisme g\'en\'eneriquement \'etale de courbes 
projectives et lisses sur $k$. On a $'\h^i(X,f_\ast \sF)='\h^i(X',\sF)$; cet 
isomorphisme est fonctoriel et le diagramme 
\[\xymatrix{
  \h^i(X,f_\ast\sF) \ar[r] \ar@{=}[d] 
    & '\h^i(X,f_\ast\sF) \ar@{=}[d] \\
  \h^i(X',\sF) \ar[r] 
    & '\h^i(X',\sF) 
}\]
est commutatif.
\end{corollary_}

L'isomorphisme est donn\'e par \ref{V:1-2}: $D f_\ast\sF=f_\ast D\sF$. 



